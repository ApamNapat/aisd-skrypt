\section{Algorytm znajdowania dwóch najbliższych punktów}
\sectionauthor{Tomasz Nanowski}

\label{sec:dwanajblizszepunkty}

Teraz zajmiemy się problemem znalezienia pary najmniej odległych punktów w zadanym zbiorze $Q = \{(x_1, y_1), \ldots, (x_n, y_n)\}$.
Interesować nas będzie odległość euklidesowa, czyli szukamy takich indeksów $i, j$, że $d(p_i, p_j) = \min\{d(p_k, p_l) \quad | \quad 1 \leq p < l \leq n\}$, gdzie $d(p_k, p_l) = \sqrt{(x_k - x_l)^2 + (y_k - y_l)^2}$

\subsection{Podejście siłowe}

W podejściu siłowym potrzebujemy wyznaczyć i porównać wszystkie odległości pomiędzy punktami. 
Jest ich ${|Q|}\choose{2}$, czyli w taki sposób problem można rozwiązać w czasie $O(n^2)$. 
W następnym rozdziale pokażemy jak zrobić to szybciej.

\subsection{Podejście Dziel i Zwyciężaj}

Wiemy już, że problem ten można naiwnie rozwiązać w czasie $\Theta(n^2)$.
Zastanówmy się jak wykorzystując strategię Dziel i Zwyciężaj zrobić to szybciej.
Już po chwili zastanowienia widać, że nie jest to takie trywialne, ponieważ po podziale zbioru na 2 części i znalezieniu dla nich rozwiązań, i tak musielibyśmy sprawdzić wszystkie odległości pomiędzy tymi zbiorami. 
Ale czy na pewno?
\\\\
Nasz algorytm będzie wywoływał się rekurencyjnie, więc w celu uniknięcia wielokrotnego sortowania, wykorzystamy dwie tablice $X$ i $Y$, które zawierać będą wszystkie punkty z Q posortowane odpowiednio po $x$'owej i $y$'owej współrzędnej.  
W tym miejscu istotnym jest, aby punkty ze wszystkich tablic były odpowiednio \textit{połączone} między sobą.  
Dzięki temu przy podziale zbioru na dwie części: $Q_L$ i $Q_R$ odtworzenie tablic $X_L$, $Y_L$ i $X_R$, $Y_R$ będzie możliwe w czasie $O(|Q|)$.  
Wystarczy przeglądać po kolei elementy z tablic $X$ oraz $Y$ i przerzucać je do mniejszych odpowiedników, w zależności czy punkt jest w części $L$ czy $R$.
\\\\
\begin{algorithm}[H]
 	\DontPrintSemicolon
  	\SetAlgorithmName{Algorytm}{}\\
  	\KwData{ $Q$ - zbiór punktów,  $X$, $Y$ }\\
  	\KwResult{ najmniejsza odległość między punktami }\\
    	\If{$|Q|  <  2$}  { \Return $\infty$ }
        \If {$|Q| = 2$} {\Return $Q[1] - Q[2]$}
		\texttt{wykorzystując tablicę $X$ znajdź prostą $l$ dzielącą zbiór Q na dwa prawie równoliczne zbiory}\;
		\texttt{podziel Q na zbiory $Q_L$ i $Q_R$ względem prostej $l$ odpowiednio po jej lewej i prawej stronie}\;
        \texttt{wyznacz tablice $X_L$, $Y_L$ i $X_R$, $Y_R$}\;
        $d_L$ $\leftarrow$ \texttt{NAJMNIEJ-ODLEGLA-PARA($Q_L$, $X_L$, $Y_L$)}\;
        $d_R$ $\leftarrow$ \texttt{NAJMNIEJ-ODLEGLA-PARA($Q_R$, $X_R$, $Y_R$)}\;
         $d$ $\leftarrow$ $\min{(d_L, d_R)}$\;
         $Y'$ $\leftarrow$ \texttt{punkty z $Y$ odległe o co najwyżej $d$ od prostej $l$}\;
         \For {$i = 1$ to  $|Y'|$ } {
         	\For {$j = 1$ to  $\min{(7, |Y'| - i)}$ } {
            	\If {$P[i] - P[i+j] < d$} {
                	$d$ $\leftarrow$ $|P[i] - P[i+j]|$
                }
            }
         }
        \Return $d$
  	\caption{Implementacja procedury NAJMNIEJ-ODLEGLA-PARA}
\end{algorithm}
