\section{Sieci przełączników Benesa-Waksmana}
\sectionauthor{Marcin Bartkowiak}

\label{sec:siecibenesa}

W tym rozdziale zajmiemy się sieciami przełączników Benesa-Waksmana. Moją one zastosowanie w sieciach komputerowych.

\subsection{Budowa}
Sieć składa się z przełączeników.
Każdy z przełączeników ma dwa możliwe stany.
\begin{itemize}
  \item W stanie 1 przełącznik przesyła dane z wejścia $i$ na wyjście $i$ ($i \in \left \{ 0, 1 \right \}$)
  \item W stanie 2 przełącznik przesyła dane z wejścia $i$ na wyjście $i + 1~mod~2$ ($i \in \left \{ 0, 1 \right \}$)
\end{itemize}
{\tiny<Tutaj wstawić obrazki ze stanami i jakąś przekładową sieć(wydaje mi się, że to lepiej objaśni, niż mój najlepszy opis)>}

\subsection{Kontrukcja sieci tworzącej wszystkie możliwe permutacje zbioru}
Dla ułatwienia będziemy się zajmować zbiorami w postaci $2^n$ ($n \in \mathbb{N}$).

Konstrukcja będzie oparta na zasadzie dziel i zwyciężaj i będzie sprowadzała problem do rekurencyjnego zbudowania sieci wielkości $2^{n-1}$,
a następnie odpowiedniego połączenia portów.
\subsubsection{$n = 1$}
Dla zbioru wielkości 2 jeden przełącznik generuje każdą możliwą permutacje; stan 1 generuje identycznośc; stan 2 drugą permutację.
\subsubsection{$n > 1$}
{\tiny<Tutaj znowu wstawiłbym rysunek idei algorytmu>}
\subsection{Własności wygenerowanej sieci}
Głębokość sieci wyraża się równaniem
$$
 G(2^n) = 
  \begin{cases} 
   1 & n = 1 \\
   G(2^{n - 1}) + 2 & n > 1
  \end{cases}
$$
z tego wynika, że $G(n) = 2 \log n - 1$


Ilość przełączeników w sieci wyraża się równaniem
$$
 P(2^n) = 
  \begin{cases} 
   1 & n = 1 \\
   2P(2^{n - 1}) + 2^n & n > 1
  \end{cases}
$$
Wykorzystując punkt 2. (a) z twierdzenia o rekurencji uniwersalnej możemy stwierdzić, że $P(n) = \Theta(n \log n)$.

\subsection{Dowód poprawności konstrukcji}
TODO
\subsection{Sortowanie}
TODO
