\section{Sortowanie ciągów różnej długości}
\sectionauthor{Sławomir Górawski}

\label{sec:sortowanieciagow}

Chcemy posortować leksykograficznie $k$ ciągów znaków (nad danym alfabetem $\Sigma$) różnej długości, o sumarycznej długości $n$, np.:

\begin{verbatim}
    1) b a
    2) x f a f g
    3) a
    4) a k j i c s f t a q
    5) a b c
    6) a b
\end{verbatim}

Moglibyśmy użyć do tego celu zwykłego algorytmu sortowania leksykograficznego, w którym zakładaliśmy równą długość sortowanych ciągów, przez dopisanie do każdego z nich pustych znaków, traktowanych w porównaniach jako mniejsze od każdego symbolu z $\Sigma$, tj.:

\begin{verbatim}
    1) b a _ _ _ _ _ _ _ _
    2) x f a f g _ _ _ _ _
    3) a _ _ _ _ _ _ _ _ _
    4) a k j i c s f t a q
    5) a b c _ _ _ _ _ _ _
    6) a b _ _ _ _ _ _ _ _
\end{verbatim}

Takie podejście jednak okazuje się być bardzo nieefektywne dla przypadków, w których występuje znacząca różnica długości sortowanych ciągów, gdyż złożoność rośnie wtedy do $O(kl_{\max})$, co może być znacząco gorsze od $O(n)$ ($l_{\max}$ - długość najdłuższego ciągu).
Aby osiągnąć złożoność linową względem sumy długości ciągów konieczna jest modyfikacja algorytmu.

Zauważmy, że gdy ciągi są różnej długości, na początku $i$-tej iteracji wszystkie słowa, których długości są mniejsze od $l_{\max} - i$, znajdują się na początkowych miejscach sortowanej kolekcji.
Aby tak było, nie jest wcale konieczne ich uczestnictwo w poprzednich iteracjach sortowania -- wystarczy znajomość długości każdego ciągu przed rozpoczęciem działania właściwego algorytmu.

Na początku policzmy długości wszystkich ciągów -- przy założeniu, że dla jednego ciągu odbywa się to w czasie liniowym względem jego długości, zajmie nam to $O(n)$ operacji.
Następnie możemy posortować ciągi względem ich długości -- używając np. RadixSorta (sortowania leksykograficznego dla liczb) jesteśmy w stanie zrobić to w czasie $O(k + l_{\max})$, a zatem $O(n)$, gdyż zarówno $k$, jak i $l_{\max}$ są mniejsze od $n$, jeżeli nie dopuszczamy pustych ciągów.

Teraz, mając posortowaną listę długości wszystkich ciągów, jesteśmy w stanie stwierdzić, które z nich muszą brać udział w danej iteracji sortowania -- bez wchodzenia w szczegóły implementacji, można np. trzymać tablicę list $T$, gdzie indeksy w tablicy wskazywałyby na iterację, od której wszystkie ciągi z listy pod tym adresem muszą dołączyć do sortowania.

Następnie możemy zacząć sortowanie -- załóżmy, że nasze ciągi zostały ustawione rosnąco względem długości, tj.:

\begin{verbatim}
    1) a
    2) b a
    3) a b
    4) a b c
    5) x f a f g
    6) a k j i c s f t a q
\end{verbatim}

Widzimy, że ciąg nr 4 nie musi brać udziału w sortowaniu aż do 6-tej iteracji, natomiast ciąg nr 3 -- do 8-mej itd. (technicznie ciągu nr 6 w ogóle nie musielibyśmy wcześniej sortować tylko z samym sobą, ale to szczegół niemający wpływu na złożoność asymptotyczną).

Zatem działanie algorytmu ostatecznie przedstawia się następująco (niech $U$ - ciągi biorące udział w sortowaniu, $T$ - tablica list ciągów biorących udział w sortowaniu od danej iteracji):

\begin{algorithm}[h]
  \DontPrintSemicolon
  \SetAlgorithmName{Algorytm}{}
  
  $U \leftarrow \varnothing$
  
  \For{$i \leftarrow 1$ to $l_{\max}$}
  {
    \If{not $T[i]$.empty}
    {
      $U \leftarrow U \cup T[i]$
    }
    Posortuj leksykograficznie ciągi z $U$ wg $i$-tej litery od końca.
  }
  
  \caption{Sortowanie ciągów różnej długości}
  \label{var-len-sort}
\end{algorithm}

Jeżeli chodzi o złożoność, możemy wyróżnić 3 fazy działania algorytmu:

\begin{enumerate}
    \item Liczenie długości ciągów -- $O(n)$,
    \item Sortowanie ciągów po długości -- $O(n)$,
    \item Iteracyjne sortowanie leksykograficzne ciągów -- łatwo zauważyć, że ilość operacji zależy liniowo od sumy mocy $U$ po każdej iteracji (lub inaczej -- wysokości kolumn znaków na przykładzie). Suma wysokości kolumn znaków to to samo, co suma długości ciągów, czyli $n$, zatem ten krok zajmuje $O(n)$ operacji, w związku z tym cały algorytm również.
\end{enumerate}