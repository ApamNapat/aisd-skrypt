% vim: set ts=2 sw=2 :
\section{Algorytm Shift-And}
\sectionauthor{Mateusz Urbańczyk}

\label{sec:shiftand}

Rozważmy następujący problem (ang. \textit{string searching}). Dla zadanych dwóch ciągów
znaków, tekstu i wzorca, odpowiedzieć na pytanie, czy wzorzec zawiera się w tekście.
Mówiąc inaczej, czy tekst zwiera spójny podciąg znaków, który jest równy wzorcowi.
Zdefiniujmy problem bardziej formalnie. \\

Niech $\Sigma$ będzie skończonym alfabetem. Dla zadanego $\mathcal{T},\mathcal{P} \in \Sigma^*$
takiego, że $|\mathcal{P}| \le |\mathcal{T}|$ (zwykle $|\mathcal{P}| \ll |\mathcal{T}|$), chcemy
odpowiedzieć na pytanie, czy

\begin{equation}
    \label{eq:stringSearchingDefinition}
    \mathlarger{\exists_{0 \leq j \leq n - m}} \enspace
    \textit{\textsf{t. że}} \enspace \mathcal{T}[j..j+m] = \mathcal{P}
\end{equation}

gdzie $m = |\mathcal{P}|$ oraz $n = |\mathcal{T}|$. Posłużymy się tymi oznaczeniami także w dalszej
części opisu. Algorytm \textit{Shift-And}, który zostanie omówiony, będzie obliczał wszystkie
prefiksy $\mathcal{P}$ które są suffiksami $\mathcal{T}[0..j]$ dla $j=0..n$. Wynik trzymany będzie
w masce bitowej $\mathcal{D} = d_m..d_1$ ($\forall_{1 \leq i \leq m} d_{i} \in \{0, 1\} $), która dla danej iteracji $j$ będzie spełniać niezmiennik:

\begin{equation}
 \label{eq:invariant}
 \mathcal{D}[i] =
  \begin{cases}
      1 & \textit{\textsf{jeśli}} \enspace \mathcal{P}[0..i] = \mathcal{T}[j-i..j]
            \quad\quad dla \enspace i=0..m \\
      0 & w.p.p
  \end{cases}
\end{equation}

Ponadto, zdefiniujmy sobie również tablicę $\mathcal{B}$, która dla każdego znaku będzie
trzymać maski bitowe wystąpień we wzorcu:

$$
    \forall x \in \mathcal{L} = \{c: c \in \mathcal{P} \wedge c \notin \mathcal{L}\}
$$
$$
    \mathcal{B}[x][i] =
     \begin{cases}
         1 & \textit{\textsf{jeśli}} \enspace \mathcal{P}[i] = c \quad\quad dla \enspace i=0..m \\
         0 & w.p.p
     \end{cases}
$$

Przejdźmy teraz do faktycznego pomysłu na algorytm. \textbf{Idea:} przechodząc od lewej do prawej,
dla każdego znaku w tekście szukamy najdłuższego prefiksu we wzorcu, który również jest
suffiksem w aktualnym oknie, gdzie okno jest ciągiem znaków z $\mathcal{T}$ o długości $m$,
kończące się na aktualnie rozważanym znaku. Prefiksy będziemy znajdować wykorzystując następujące
operacje bitowe na tablicy $\mathcal{D}$ zachowujące niezmiennik (\ref{eq:invariant}):

\begin{equation}
    \label{eq:resultUpdate}
    \mathcal{D'} \leftarrow ((\mathcal{D} \ll 1 ) \enspace | \enspace 1)
        \enspace \& \enspace \mathcal{B}[\mathcal{T}[j]]
\end{equation}

\begin{lemma}
    Dla $\mathcal{D}$ w iteracji $j$ spełniona jest równoważność:
    $$
        \mathcal{D'}[i+1] = 1 \Leftrightarrow{}
            \mathcal{D}[i] = 1 \wedge{} \mathcal{T}[j+1] = \mathcal{P}[i+1]
    $$
    gdzie $\mathcal{D'}$ to wynik w następnej iteracji algorytmu.
\end{lemma}
\begin{proof}
    Jeżeli lemat jest prawdziwy, to zachowany będzie również niezmiennik algorytmu. \\
    $(\Rightarrow)$
    Weźmy dowolne $i$, $j$ i rozważmy $\mathcal{D'}$. Dowód przeprowadzimy nie wprost.
    Załóżmy, że
        $\mathcal{D'}[i+1] = 1$
    $\wedge$
        ($\mathcal{D}[i] \neq 1 \vee{} \mathcal{T}[j+1] \neq \mathcal{P}[i+1]$)
    i rozważmy przypadki:
    \begin{itemize}
        \item $\mathcal{D}[i] \neq 1$. To implikuje, że $\mathcal{D}[i]$ = 0.
           W kolejnej iteracji wykonamy
           $\mathcal{D} \ll 1, zatem \enspace \mathcal{D'}[i+1] = 0$. Sprzeczność
        \item $\mathcal{T}[j+1] \neq \mathcal{P}[i+1]$.
            Zauważmy, że wtedy $\mathcal{B}[\mathcal{T}[j+1]][i+1] = 0$,
            a stąd bit $\mathcal{D'}[i+1]$ zostanie wyzerowany po operacji \textit{AND}. Sprzeczność.

    \end{itemize}
    $(\Leftarrow)$
    Analogicznie. Zostawiam jako proste ćwiczenie dla czytelnika.
\end{proof}

\begin{observation}
Jeżeli po wykonaniu iteracji $\mathcal{D}[m-1] = 1$, to nasz wzorzec występuje w tekście.
\end{observation}

Dokładny algorytm wynika wprost z powyższej obserwacji oraz z (\ref{eq:resultUpdate}):

\begin{algorithm}[h]
  \DontPrintSemicolon
  \SetAlgorithmName{Algorytm}{}

  \KwData{\enspace $\mathcal{T}, \mathcal{P} \in \Sigma, |\mathcal{P}| \le |\mathcal{T}|$}

  \KwResult{\enspace ewaluacja wyrażenia (\ref{eq:stringSearchingDefinition})}

  \For{$c \in \Sigma$} {
      $\mathcal{B}[c] \leftarrow 0$
  }
  \For{$i=1..m$} {
      $
        \mathcal{B}[\mathcal{P}[i]] \leftarrow
          \mathcal{B}[\mathcal{P}[i]] \enspace | \enspace (1 \ll (i-1))
      $
  }
  $\mathcal{D} \leftarrow 0$ \\
  \For{j=1..n} {
    $
      \mathcal{D} \leftarrow ((\mathcal{D} \ll 1 ) \enspace | \enspace 1)
        \enspace \& \enspace \mathcal{B}[\mathcal{T}[j]]
    $ \\
    \If{$\mathcal{D} \enspace \& \enspace (1 \ll (m-1))$}{
          \Return true
    }
  }
  \Return false

  \caption{Algorym Shift-And}
  \label{alg-wiesniakow}
\end{algorithm}


