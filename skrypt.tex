\documentclass[10pt,b5paper]{book}
\usepackage{polski}
\usepackage{amsthm}
\usepackage[T1]{fontenc}
\usepackage[utf8]{inputenc}
\usepackage{geometry}
\usepackage{caption}
\usepackage{tikz}
\usepackage{amsmath}
\usepackage{amssymb}
\usepackage{array}
\usepackage[titletoc]{appendix}
\usepackage{scrextend}
\usepackage[ruled,commentsnumbered]{algorithm2e}
\usepackage{adjustbox}
\usepackage[hidelinks]{hyperref}

%% Biblioteki TiKZa

\usetikzlibrary{shapes.arrows, shapes.multipart, calc}

%% Użyteczne komendy
\makeatletter
\newcommand{\sectionauthor}[1]{%
  {\parindent0pt\vspace*{-10pt}%
  \linespread{1.1}\large\scshape#1%
  \par\nobreak\vspace*{15pt}}
  \@afterheading%
}
\makeatother

\newtheorem{theorem}{Twierdzenie}
\newtheorem{definition}{Definicja}
\newtheorem{lemma}{Lemat}
\newtheorem{fact}{Fakt}
\newtheorem{observation}{Obserwacja}
\def\checkmark{\tikz\fill[scale=0.4](0,.35) -- (.25,0) -- (1,.7) -- (.25,.15) -- cycle;}
\newcommand{\tizkboxwithcaption}[2]{\noindent\begin{figure}[!h]\centering\begin{adjustbox}{min width=0.75\textwidth, max width=1\textwidth}\input{#1}\end{adjustbox}\caption{#2}\end{figure}}

\newcommand{\comment}[1]{\marginpar{\tiny\raggedright #1}}

\renewcommand{\KwData}{\textbf{Input:}}
\renewcommand{\KwResult}{\textbf{Output:}}

%% Strona tytułowa

\usepackage[load-headings]{exsheets}
\DeclareInstance{exsheets-heading}{mylist}{default}{
  runin = true ,
  attach = {
    main[l,vc]number[l,vc](-3em,0pt) ;
    main[r,vc]points[l,vc](\linewidth+\marginparsep,0pt)
  }
}

\SetupExSheets{
  headings = mylist ,
  headings-format = \normalfont ,
  counter-format = se.qu ,
  counter-within = section
}

\usepackage{etoolbox}
\AtBeginEnvironment{question}{\addmargin[3em]{0em}}
\AtEndEnvironment{question}{\endaddmargin}

\newcommand*{\titleGM}{\begingroup
\hbox{
\hspace*{0.2\textwidth}
\hspace*{0.05\textwidth}
\parbox[b]{0.75\textwidth}{
{\noindent\Huge\bfseries Skrypt \\[0.2\baselineskip] z Algorytmów \\[0.2\baselineskip] i~struktur danych}\\[2\baselineskip]
{\large \textit{Zbiór mniej lub bardziej ciekawych algorytmów i~struktur danych, jakie bywały omawiane na wykładzie (albo i nie).}}\\[3\baselineskip]
{\Large \textsc{praca zbiorowa pod redakcją \\[0.1\baselineskip] Krzysztofa Piecucha}}

\vspace{0.5\textheight}
{\noindent Korzystać na własną odpowiedzialność.}\\[\baselineskip]
}}
\endgroup}

\definecolor{titlepagecolor}{cmyk}{0.9,1,.6,.40}

\newcommand\titlepagedecoration{%
\begin{tikzpicture}[remember picture,overlay,shorten >= -10pt]

\coordinate (aux1) at ([yshift=-15pt]current page.north west);
\coordinate (aux2) at ([yshift=-450pt]current page.north west);
\coordinate (aux3) at ([xshift=+4.5cm]current page.north west);
\coordinate (aux4) at ([yshift=-150pt]current page.north west);

\begin{scope}[titlepagecolor!60,line width=14pt,rounded corners=12pt]
\draw
  (aux1) -- coordinate (a)
  ++(-45:5) --
  ++(225:5.1) coordinate (b);
\draw[shorten <= -10pt]
  (aux3) --
  (a) --
  (aux1);
\draw[opacity=0.4,titlepagecolor,shorten <= -10pt]
  (b) --
  ++(-45:2.2) --
  ++(225:2.2);
\end{scope}
\draw[titlepagecolor,line width=9pt,rounded corners=8pt,shorten <= -10pt]
  (aux4) --
  ++(-45:1.2) --
  ++(225:1.2);
\begin{scope}[titlepagecolor!70,line width=6pt,rounded corners=6pt]
\draw[shorten <= -10pt]
  (aux2) --
  ++(-45:3) coordinate[pos=0.45] (c) --
  ++(225:3.1);
\draw
  (aux2) --
  (c) --
  ++(45:2.5) --
  ++(135:2.5) --
  ++(225:2.5) coordinate[pos=0.3] (d);
\draw
  (d) -- +(135:1);
\end{scope}
\end{tikzpicture}%
}

\begin{document}
\pagenumbering{arabic}
\pagestyle{empty}

\titleGM
\titlepagedecoration

\tableofcontents
\pagestyle{plain}

\chapter{Zrobione}

\section{Twierdzenie o rekurencji uniwersalnej}

Popularną metodą rozwiązywania zadań jest metoda Dziel i Zwyciężaj.
Polega ona na podzieleniu problemu na mniejsze, rozwiązaniu ich w sposób rekurencyjny, a następnie na scaleniu wyniku w jeden.
Schemat tej metody jest przedstawiony jako Schemat \ref{recursion-example}.
\begin{algorithm}[h]
  \DontPrintSemicolon
  \SetAlgorithmName{Schemat}{}
    
  \If{$n \leq 1$}
  {
    rozwiąż trywialny przypadek
  }
  Stwórz $a$ podproblemów wielkości $n/b$ w czasie $D(n)$
  
  
  \For{$i \leftarrow 1$ to $a$}
  {
   	wykonaj procedurę \texttt{Dziel\_i\_zwyciezaj} rekurencyjnie dla $i$-tego podproblemu
  }
  Połącz wyniki w czasie $P(n)$
  \caption{Procedura \texttt{Dziel\_i\_zwyciezaj}}
  \label{recursion-example}
\end{algorithm}

Złożoność takiego algorytmu możemy zapisać zależnością rekurencyjną $T(n) = aT(n/b) + \Theta(n^k \log^{p} n)$ przy czym $P(n) + D(n) \in \Theta(n^k \log^{p} n)$.
Jednakże zależność rekurencyjna na czas działania algorytmu nie zawsze nas satysfakcjonuje.
Zazwyczaj chcielibyćmy uzyskać wzór zwarty.
Do tego celu służy poniższe twierdzenie, znane Twierdzeniem o rekurencji uniwersalnej.

\begin{theorem}
 Niech $T(n) = aT(n/b) + \Theta(n^k \log^{p} n)$ oraz $a \geq 1$, $b > 1$, $k \geq 0$ oraz $p$ liczby rzeczywiste.
 Wtedy
\begin{enumerate}
\item jeżeli $a > b^k$, to $T(n) \in \Theta(n^{\log_b a})$ \label{mt-1}
\item jeżeli $a = b^k$ oraz 
\begin{enumerate}
 \item $p > -1$ to $T(n) \in \Theta(n^{\log_b a} \log^{p + 1} n)$ \label{mt-2a}
 \item $p = -1$ to $T(n) \in \Theta(n^{\log_b a} \log \log n)$ \label{mt-2b}
 \item $p < -1$ to $T(n) \in \Theta(n^{\log_b a})$ \label{mt-2c}
\end{enumerate}
\item jeżeli $a < b^k$ oraz
\begin{enumerate}
 \item $p \geq 0$ to $T(n) \in \Theta(n^k \log^{p} n)$ \label{mt-3a}
 \item $p < 0$ to $T(n) \in O(n^k)$ \label{mt-3b}
\end{enumerate}
\end{enumerate}
 \label{Master}
\end{theorem}

\begin{proof}
 TODO TODO TODO.
\end{proof}

\subsection{Przykłady wykorzystania twierdzenia}

W rozdziale \ref{sec:merge-sort} zapoznamy się z algorytmem sortowania przez scalanie.
\comment{Nie podoba mi się to, że to jest osobny podrozdział.
Nie wiem czy to powinnien być osobny akapit, tabelka czy może coś jeszcze innego.}
Jego złożoność określona jest wzorem rekurencyjnym $T(n) = 2 \cdot T(n/2) + \Theta(n)$.
W tym przypadku $a = 2$, $b = 2$, $k = 1$ oraz $p = 0$.
Ponieważ $a = b^k$ sprawdzamy dodatkowo, że $p > -1$ i otrzymujemy, że w naszym przypadku powinniśmy skorzystać z pkt \ref{mt-2a}.
Otrzymujemy, że złożoność naszego algorytmu jest $\Theta(n^{\log_b a} \log^{p + 1} n)$ czyli $\Theta(n\log n)$.

W rodziale \ref{sec:bitoniczne} opisany został algorytm sortowania bitonicznego.
Jego złożoność określona jest wzorem $T(n) = 2 \cdot T(n/2) + \Theta(n \log n)$.
W tym przypadeku $a = 2$, $b = 2$, $k=1$ i $p = 1$.
Ponieważ $a = b^k$ i $p > -1$ korzystamy z punktu \ref{mt-2a}.
Otrzymujemy złożoność $\Theta(n \log^2 n)$.

\comment{Naprawić referencję.}
W rozdziale \ref{sec:karatsuba} opisany jest algorytm Karatsuby. Jego złożoność opisana jest rekurencyjnym wzorem $3 T(n/2) + \Theta(n)$.
W tym przypadku $a = 3$, $b = 2$, $k=1$ i $p=0$.
Ponieważ $a > b^k$ korzystamy z punktu \ref{mt-1}.
Otrzymujemy złożoność $\Theta(n^{\log_{2} 3})$ czyli około $\Theta(n^{1.585}).$

Wyszukiwanie binarne to algorytm klasy Dziel i Zwyciężaj wyszukujący element w posortowanym ciągu.
Ma złożoność opisaną rekurencyjnym wzorem $T(n) = T(n/2) + \Theta(1)$.
Mamy więc $a = 1$, $b = 2$, $k = p = 0$.
Ponieważ $a = b^k$ i $p > -1$ korzystamy z punktu \ref{mt-2a}.
Otrzymujemy złożoność $\Theta(\log n)$.

Algorytm Strassena jest opisany w rodziale \ref{sec:strassen}.
Jego złożoność opisana jest rekurencyjnym wzorem $T(n) = 7 \cdot T(n/2) + \Theta(n^2)$.
W tym przypadku mamy $a = 7$, $b = 2$, $k = 2$, $p = 0$.
Jako, że $a > b^k$ wykorzystamy punkt \ref{mt-1}.
Otrzymujemy złożoność $\Theta(n^{\log_{2} 7})$ czyli około $\Theta(n^{2.807})$


\section{Sortowanie bitoniczne}

\label{sec:bitoniczne}

W tym rozdziale przedstawimy algorytm sortowania bitonicznego.
Jest to algorytm działający w czasie $\Theta(n \log^2 n)$ czyli gorszym niż inne, znane algorytmy sortujące, takie jak sortowanie przez scalanie albo sortowanie szybkie.
Zaletą sortowania bitonicznego jest to, że może zostać uruchomiony równolegle na wielu procesorach.
Ponadto, dzięki temu, że algorytm zawsze porównuje te same elementy bez względu na dane wejściowe, istnieje prosta implementacja fizyczna tego algorytmu (np. w postaci tzw. sieci sortujących).
Algorytm będzie zakładał, że rozmiar danych $n$ jest potęgą dwójki.
Gdyby tak nie było, moglibyśmy wypełnić tablicę do posortowania nieskończonościami, tak aby uzupełnić rozmiar danych do potęgi dwójki.
Rozmiar danych zwiększyłby się wtedy nie więcej niż dwukrotnie, zatem złożoność asymptotyczna pozostałaby taka sama.

Sortowanie bitoniczne posługuje się tzw. ciągami bitonicznymi, które sobie teraz zdefiniujemy.
\begin{definition}
 \textbf{Ciągiem bitonicznym właściwym} nazywamy każdy ciąg powstały przez sklejenie ciągu niemalejącego z ciągiem nierosnącym.
\end{definition}
Dla przykładu ciąg $2$, $2$, $5$, $100$, $72$, $69$, $42$, $17$ jest ciągiem bitonicznym właściwym, gdyż powstał przez sklejenie ciągu niemalejącego $2$, $2$, $5$ oraz ciągu nierosnącego $100$, $72$, $69$, $42$, $17$.
Ciąg $1$, $0$, $1$, $0$ nie jest ciągiem bitonicznym właściwym, gdyż nie istnieją taki ciąg niemalejący i taki ciąg nierosnący, które w wyniku sklejenia dałyby podany ciąg.
\begin{definition}
 \textbf{Ciągiem bitonicznym} nazywamy każdy ciąg powstały przez rotację cykliczną ciągu bitonicznego właściwego.
\end{definition}
Ciąg $69$, $42$, $17$, $2$, $2$, $5$, $100$, $72$ jest ciągiem bitonicznym, gdyż powstał przez rotację cykliczną ciągu bitonicznego właściwego $2$, $2$, $5$, $100$, $72$, $69$, $42$, $17$.

Istnieje prosty algorytm sprawdzający, czy ciąg jest bitoniczny.
Należy znaleźć element największy oraz najmniejszy.
Następnie od elementu najmniejszego należy przejść cyklicznie w prawo (tj. w sytuacji gdy natrafimy na koniec ciągu, wracamy do początku) aż napotkamy element największy.
Elementy, które przeszliśmy w ten sposób powinny tworzyć ciąg niemalejący.
Analogicznie, idziemy od elementy największego cyklicznie w prawo aż do elementu najmniejszego.
Elementy, które odwiedziliśmy powinny tworzyć ciąg nierosnący.
W sytuacji w której mamy wiele elementów najmniejszych (największych), powinny one ze sobą sąsiadować (w sensie cyklicznym) i nie ma znaczenia, który z nich wybierzemy.
Dla przykładu w ciągu  $69$, $42$, $17$, $2$, $2$, $5$, $100$, $72$ idąc od elementu najmniejszego do największego tworzymy ciąg $2$, $2$, $5$, $100$ i jest to ciąg niemalejący.
Idąc od elementu największego do najmniejszego otrzymujemy ciąg $100$, $72$, $69$, $42$, $17$, $2$ i jest to ciąg nierosnący.

Jedyną procedurą, która będzie przestawiała elementy w tablicy, będzie procedura \texttt{bitonic\_compare} (Algorytm \ref{bitonic-compare}).
Jako dane wejściowe otrzymuje ona tablicę \texttt{A}, wielkość tablicy \texttt{n} oraz wartość logiczną \texttt{up}, która określa, czy ciąg będzie sortowany rosnąco czy malejąco.
\begin{algorithm}[h]
  \DontPrintSemicolon
  \SetAlgorithmName{Algorytm}{}

  \KwData{ A[0..n-1], up }
  
  \For{$i \leftarrow 0$ to $n-1$}
  {
    \If{$(A[i] > A[i+n/2]) = up$}
    {
      $A[i] \leftrightarrow A[i+n/2]$\;
    }
  }
  \caption{Procedura \texttt{bitonic\_compare}}
  \label{bitonic-compare}
\end{algorithm}
Procedura dzieli zadaną na wejściu tablicę na dwie równe części.
Następnie porównuje pierwszy element z pierwszej części z pierwszym elementem z drugiej części.
Jeśli te elementy nie znajdują się w pożądanym porządku, to je przestawia.
Następnie powtarza tą czynność z kolejnymi elementami.

Dla przykładu, jeśli procedurę uruchomimy z tablicą \texttt{A = [2, 8, 7, 1, 4, 3, 5, 6]}, wartością \texttt{n = 8} oraz \texttt{up = true}, w wyniku otrzymamy tablicę \texttt{A = [2, 3, 5, 1, 4, 8, 7, 6]}.
W pierwszym kroku wartość $2$ zostanie porównana z wartością $4$.
Ponieważ chcemy otrzymać porządek rosnący (wartość zmiennej \texttt{up} jest ustawiona na \texttt{true}), to zostawiamy tą parę w spokoju.
W następnym kroku porównujemy wartość $8$ z wartością $3$.
Te wartości są w złym porządku, dlatego algorytm zamienia je miejscami.
Dalej porównujemy $7$ z $5$ i zamieniamy je miejscami i w końcu porównujemy $1$ z $6$ i te wartości zostawiamy w spokoju, gdyż są w dobrym porządku.

Procedura \texttt{bitonic\_compare} ma bardzo ważną własność, którą teraz udowodnimy.

\begin{theorem}
 Jeżeli elementy tablicy \texttt{A[0..n-1]} tworzą ciąg bitoniczny, to po zakończeniu procedury \texttt{bitonic\_compare} elementy tablicy \texttt{A[0..n/2-1]} oraz tablicy \texttt{A[n/2..n-1]} będą tworzyły ciągi bitoniczne.
 Ponadto jeśli wartość zmiennej \texttt{up} jest ustawiona na \texttt{true} to każdy element tablicy \texttt{A[0..n/2-1]} będzie niewiększy od każdego elementu tablicy \texttt{A[n/2..n-1]}.
 W przeciwnym przypadku będzie niemniejszy.
 \label{bitonic-theorem}
\end{theorem}

Weźmy dla przykładu ciąg bitoniczny $69$, $42$, $17$, $2$, $2$, $5$, $100$, $72$.
Po przejściu procedury \texttt{bitonic\_compare} z ustawioną zmienną \texttt{up} na wartość \texttt{true} otrzymamy ciąg $2$, $5$, $17$, $2$, $69$, $42$, $100$, $72$.
Ciągi $2$, $5$, $17$, $2$ oraz $69$, $42$, $100$, $72$ są ciągami bitonicznymi.
Ponadto każdy element ciągu $2$, $5$, $17$, $2$ jest niewiększy od każdego elementu ciągu $69$, $42$, $100$, $72$.

Przejdźmy do dowodu powyższego twierdzenia.
Przyda nam się do tego poniższy lemat:

\begin{lemma}[zasada zero-jeden]
 Twierdzenie \ref{bitonic-theorem}. jest prawdziwe dla dowolnych tablic wtedy i tylko wtedy, gdy jest prawdziwe dla tablic zero-jedynkowych.
 \label{zero-one-lemma}
\end{lemma}

\begin{proof}
 Jeśli twierdzenie jest prawdziwe dla każdej tablicy to w szczególności jest prawdziwe dla tablic złożonych z zer i jedynek.
 Dowód w drugą stronę jest dużo ciekawszy.
 
 Weźmy dowolną funkcję niemalejącą $f$.
 To znaczy funkcję $f: \mathbb{R} \rightarrow \mathbb{R}$ taką, że $\forall_{a, b \in \mathbb{R}} \enspace a \leq b \Rightarrow f(a) \leq f(b)$.
 Dla tablicy \texttt{T} przez $f(T)$ będziemy rozumieli tablicę powstałą przez zaaplikowanie funkcji $f$ do każdego elementu tablicy \texttt{T}.
 Niech \texttt{A} oznacza tablicę wejściową do procedury \texttt{bitonic\_compare} i niech \texttt{B} oznacza tablicę wyjściową.
 Udowodnimy, że karmiąc procedurę \texttt{bitonic\_compare} tablicą $f(A)$ otrzymamy tablicę $f(B)$.
 W kroku $i$-tym procedura rozważa przestawienie elementów $t_i$ oraz $t_{i+n/2}$.
 Jeśli $f(a_i) = f(a_{i+n/2})$ to nie ma znaczenia czy elementy zostaną przestawione.
 Z kolei jeśli $f(a_i) < f(a_{i+n/2})$ to $a_i < a_{i+n/2}$ zatem jeśli procedura przestawi elementy $f(a_i)$ oraz $f(a_{i+n/2})$ to również przestawi elementy $a_i$ oraz $a_{i+n/2}$.
 Analogicznie gdy $f(a_i) > f(a_{i+n/2})$.
 Zatem istotnie: dla każdej funkcji niemalejącej $f$, procedura \texttt{bitonic\_compare} otrzymując na wejściu tablicę $f(A)$ zwróci na wyjściu tablicę $f(B)$.
 
 Wróćmy do dowodu lematu.
 Dowód nie wprost.
 Załóżmy, że twierdzenie jest prawdziwe dla wszystkich tablic zero-jedynkowych i nie jest prawdziwe dla pewnej tablicy \texttt{T[0..n-1]}.
 Niech \texttt{S[0..n-1]} oznacza zawartość tablicy po zakończeniu procedury \texttt{bitonic\_compare}.
 Jeśli twierdzenie nie jest prawdziwe, oznacza to, że albo któraś z tablic \texttt{S[0..n/2-1]}, \texttt{S[n/2..n-1]} nie jest bitoniczna albo, że element pierwszej z nich jest większy od któregoś elementu z drugiej tablicy.
 Rozważmy dwa przypadki.
 
 Załóżmy, że tablica \texttt{S[0..n/2-1]} nie jest bitoniczna (przypadek kiedy druga z tablic nie jest bitoniczna, jest analogiczny).
 Załóżmy, że ciąg powstały przez przejście od najmniejszego elementu w tej tablicy do największego nie tworzy ciągu niemalejącego (przypadek gdy ciąg powstały przez przejście od największego elementu do najmniejszego nie tworzy ciągu nierosnącego jest analogiczny).
 Zatem istenieje w tablicy element \texttt{S[i]} większy od elementu \texttt{S[i+1]}.
 Rozważmy następują funkcję:
 \[   
  f(a) = 
     \begin{cases}
       1 &\quad\textsf{jeśli} \enspace a \leq S[i]\\
       0 &\quad\textsf{wpp.}
     \end{cases}
 \]
 Zwróćmy uwagę, że w takiej sytuacji twierdzenie nie byłoby prawdziwe dla tablicy $f(T)$, zatem dla tablicy zero-jedynkowej.
 Gdyż ponownie - element $f(S[i]) = 1$ byłby większy od elementu $f(S[i+1]) = 0$.
 
  \begin{center}
  \begin{tabular}{ccccccccr}
  1 & 2 & 3 & 5          & 4            & 6 & 7 & 1 & \texttt{S[]} \\
  0 & 0 & 0 & 1          & 0            & 1 & 1 & 0 & $f(S)$ \\
    &   &   & $i$ & $i+1$ &   &   &   &
  \end{tabular}
  \end{center}
 
 Drugi przypadek.
 Załóżmy, że zmienna \texttt{up} ustawiona jest na \texttt{true} (przypadek drugi jest analogiczny).
 Załóżmy, że element \texttt{S[i]} jest mniejszy od elementu \texttt{S[j]} gdzie $j < n/2$ oraz $i \geq n/2$.
 Rozważmy funkcję:
  \[   
  f(a) = 
     \begin{cases}
       1 &\quad\textsf{jeśli} \enspace a \leq S[j]\\
       0 &\quad\textsf{wpp.}
     \end{cases}
 \]
 Wtedy twierdzenie nie byłoby prawdziwe dla tablicy $f(T)$ (zero-jedynkowej).
 Ponownie - element $f(S[i]) = 0$ byłby mniejszy od elementu $f(S[j]) = 1$.
 
  \begin{center}
  \begin{tabular}{ccccccccr}
  1 & 2 & 100        & 2 & 102 & 99         & 103 & 107 & \texttt{S[]} \\
  0 & 0 & 1          & 0 & 1   & 0          & 1   & 1   & $f(S)$ \\
    &   & $j$ &   &     & $i$ &     &     &
  \end{tabular}
  \end{center}
 
 
\end{proof}

Do pełni szczęścia potrzebujemy udowodnić, że Twierdzenie \ref{bitonic-theorem} jest prawdziwe dla wszystkich ciągów zero-jedynkowych.

\begin{lemma}
 Twierdzenie \ref{bitonic-theorem}. jest prawdziwe dla wszystkich ciągów zero-jedynkowych.
 \label{zero-one-cases-lemma}
\end{lemma}

\begin{proof}
 Zakładać będziemy, że zmienna \texttt{up} jest ustawiona na \texttt{true} (dowód dla sytuacji przeciwnej jest analogiczny).
 \comment{Ten dowód jest nudny.}
 Istnieje sześć rodzai bitonicznych ciągów zero-jedynkowych : $0^n$, $0^k1^l$, $0^k1^l0^m$, $1^n$, $1^k0^l$, $1^k0^l1^m$ z czego trzy ostatnie są symetryczne do trzech pierwszych (więc zostaną pominięte w dowodzie).
 Rozważmy wszystkie interesujące nas przypadki:
 \begin{itemize}
  \item $0^n$.
   Po wykonaniu procedury \texttt{bitonic\_compare} otrzymamy $0^{n/2}$ oraz $0^{n/2}$.
   Oba ciągi są bitoniczne i każdy element z pierwszego ciągu jest niewiększy od każdego elementu z ciągu drugiego.
  \item $0^k1^l$ oraz $k < n/2$.
   Wtedy po wykonaniu procedury \texttt{bitonic\_compare} otrzymamy ciągi $0^k1^{l-n/2}$ oraz $1^{n/2}$.
   Oba ciągi są bitoniczne i każdy element z pierwszego ciągu jest niewiększy od każdego elementu z ciągu drugiego.
  \item $0^k1^l$ oraz $k > n/2$.
   Otrzymamy ciągi $0^{n/2}$ oraz $0^{k-n/2}1^l$.
   Znowu - oba są bitoniczne i każdy element z pierwszego jest niewiększy od każdego z drugiego.
  \item $0^k1^l0^m$ oraz $k > n/2$.
   Wtedy otrzymujemy ciągi $0^{n/2}$ oraz $0^{k-n/2}1^l0^m$.
   Spełniają one tezę twierdzenia.
  \item $0^k1^l0^m$ oraz $m > n/2$.
   Ciągi, które otrzymamy wyglądają tak: $0^{n/2}$ oraz $0^k1^l0^{m-n/2}$.
   Są to ciągi, które nas cieszą.
  \item $0^k1^l0^m$ oraz $l > n/2$.
   Dostaniemy wtedy ciągi $0^k1^{l-n/2}0^m$ oraz $1^{n/2}$.
   Są to ciągi, które spełniają naszą tezę.
  \item $0^k1^l0^m$ oraz $k, l, m < n/2$.
   Ciągi, które uzyskamy to $0^{n/2}$ oraz $1^{n/2-m}0^{n/2-l}$ $1^{n/2-k}$.
   Spełniają one naszą tezę.
 \end{itemize}
\end{proof}

Na mocy Lematów \ref{zero-one-lemma} i \ref{zero-one-cases-lemma} Twierdzenie \ref{bitonic-theorem} jest prawdziwe dla wszystkich tablic \texttt{T[0..n-1]}.
Mając tak piękne twierdzenie, możemy napisać prosty algorytm sortujący ciągi bitoniczne (Algorytm \ref{bitonic-merge}).

\begin{algorithm}[h]
  \DontPrintSemicolon
  \SetAlgorithmName{Algorytm}{}
  
  \KwData{ A - tablica bitoniczna, n, up }
  
  \KwResult{ A - tablica posortowana }
  
   \If{$n > 1$}
   {
     bitonic\_compare(A[$0$..$n-1$], $n$,   up)\;
     bitonic\_merge(A[$0$..$n/2-1$], $n/2$, up)\;
     bitonic\_merge(A[$n/2$..$n-1$], $n/2$, up)\;
   }
  \caption{Procedura \texttt{bitonic\_merge}}
  \label{bitonic-merge}
\end{algorithm}

Algorytm zaczyna od wywołania procedury \texttt{bitonic\_compare}.
Dzięki niej, wszystkie elementy mniejsze wrzucane są do pierwszej połowy tablicy, a elementy większe do drugiej połowy.
Ponadto \texttt{bitonic\_compare} gwarantuje, że obie podtablice pozostają bitoniczne (jakie to piękne!).
Możemy zatem wykonać całą procedurę ponownie na obu podtablicach rekurencyjnie.

Złożoność algorytmu wyraża się wzorem rekurencyjnym $T(n) = 2 \cdot T(n/2) + \Theta(n)$.
Rozwiązując rekurencję otrzymujemy, że złożoność algorytmu to $\Theta(n \log n)$.

Mamy algorytm sortujący ciągi bitoniczne.
Jak uzyskać algorytm sortujący dowolne ciągi?
Zrealizujemy to w najprostszy możliwy sposób!
Posortujemy (rekurencyjnie) pierwszą połowę tablicy rosnąco, drugą połowę tablicy malejąco (dlatego potrzebna nam była zmienna \texttt{up}!) i uzyskamy w ten sposób ciąg bitoniczny.
Teraz wystarczy już uruchomić algorytm sortujący ciągi bitoniczne i voilà.

\begin{algorithm}[h]
  \DontPrintSemicolon
  \SetAlgorithmName{Algorytm}{}
  
  \KwData{ A, n, up }
  
  \KwResult{ A - tablica posortowana }
  
   \If{$n > 1$}
   {
     bitonic\_sort(A[$0$..$n/2-1$], $n/2$, true)\;
     bitonic\_sort(A[$n/2$..$n-1$], $n/2$, false)\;
     bitonic\_merge(A[$0$..$n-1$], $n$, up)\;
   }
  \caption{Procedura \texttt{bitonic\_sort}}
  \label{bitonic-sort}
\end{algorithm}

Złożoność algorytmu wyraża się wzorem rekurencyjnym $T(n) = 2 \cdot T(n/2) + \Theta(n \log n)$.
Rozwiązaniem tej rekurencji jest $\Theta(n \log^2 n)$.

\section{Algorytm macierzowy wyznaczania liczb Fibonacciego}
\sectionauthor{Rafał Florczak}

\label{sec:fibonacci}

W tym rozdziale opiszemy algorytm obliczania liczb Fibonacciego, który wykorzystuje szybkie potęgowanie.
Algorytm działa w czasie $\Theta(\log{n})$, co sprawia, że jest znacznie atrakcyjniejszy (gdy pytamy tylko o jedną liczbę) od algorytmu dynamicznego, który wymaga czasu $\Theta(n)$.
Zacznijmy od zdefiniowania ciągu Fibonacciego:
\begin{equation*}
  F_n=\begin{cases}
    n, & \text{jeśli $n \leq1$}\\
    F_{n-1} + F_{n-2}, & \text{wpp.}
  \end{cases}
\end{equation*}

Teraz, znajdźmy taką macierz $M$, która po wymnożeniu przez transponowany wektor wyrazów $F_{n}$ i $F_{n - 1}$ da nam wektor, w którym otrzymamy wyrazy $F_{n + 1}$ oraz $F_{n}$. 
Łatwo sprawdzić, że dla ciągu Fibonacciego taka macierz ma postać:
\begin{equation*}
  M = \begin{bmatrix}1 & 1 \\ 1 & 0\end{bmatrix}
\end{equation*}
bo:
\begin{equation}
  \label{eq:fibonacci-m}
  M
  \times
  \begin{bmatrix}F_n \\ F_{n - 1}\end{bmatrix}
  =
  \begin{bmatrix}1 & 1 \\ 1 & 0\end{bmatrix}
  \times
  \begin{bmatrix}F_n \\ F_{n - 1}\end{bmatrix}
  =
  \begin{bmatrix}F_n + F_{n - 1} \\ F_{n}\end{bmatrix}
  =
  \begin{bmatrix}F_{n + 1} \\ F_{n}\end{bmatrix}
\end{equation}
Wynika to wprost z definicji mnożenia macierzy oraz definicji ciągu Fibonacciego.
Wykonajmy mnożenie z równania \ref{eq:fibonacci-m} $n$ razy:
\begin{equation*}
  \underbrace{M \times \bigg(M \times \bigg(M \times ...\,\bigg(M}_\text{n razy}
  \times
  \begin{bmatrix}F_1 \\ F_0\end{bmatrix}\bigg)\,...\,\bigg)\bigg)
\end{equation*}
Z faktu, że mnożenie macierzy jest łączne oraz powyższego wyrażenia otrzymujemy:
\begin{align*}
  M^n \times \begin{bmatrix}F_1 \\ F_0\end{bmatrix}
\end{align*}
Pokażemy, że powyższa macierz ma zastosowanie w obliczaniu n-tej liczby Fibonacciego.
\begin{lemma}
  \begin{equation*}
    M^{n} \times \begin{bmatrix}F_1 \\ F_0\end{bmatrix} = \begin{bmatrix}F_{n + 1} \\ F_{n}\end{bmatrix}
  \end{equation*}
\end{lemma}

\begin{proof}[Dowód przez indukcję]
  Sprawdźmy dla $n = 0$. Mamy:
  \begin{equation*}
    \begin{bmatrix}1 & 1 \\ 1 & 0\end{bmatrix}^0
    \times
    \begin{bmatrix}1 \\ 0\end{bmatrix}
    =
    I
    \times
    \begin{bmatrix}1 \\ 0\end{bmatrix}
    =
    \begin{bmatrix}1 \\ 0\end{bmatrix}
    =
    \begin{bmatrix}F_1 \\ F_0\end{bmatrix}
  \end{equation*}
  Rozważmy $n + 1$ zakładając poprawność dla $n$.
  \begin{equation*}
    \begin{bmatrix}
      1 & 1 \\
      1 & 0
    \end{bmatrix}^{n + 1}
    \times
    \begin{bmatrix}1 \\ 0\end{bmatrix}
    =
    \begin{bmatrix}
      1 & 1 \\
      1 & 0
    \end{bmatrix}
    \times
    \begin{bmatrix}
      1 & 1 \\
      1 & 0
    \end{bmatrix}^{n}
    \times
    \begin{bmatrix}1 \\ 0\end{bmatrix}
    \stackrel{Z.I.}{=}
    \begin{bmatrix}
      1 & 1 \\
      1 & 0
    \end{bmatrix}
    \times
    \begin{bmatrix}F_{n+1} \\ F_{n}\end{bmatrix}
    \stackrel{\ref{eq:fibonacci-m}}{=}
    \begin{bmatrix}F_{n + 2} \\ F_{n + 1}\end{bmatrix}
  \end{equation*}
\end{proof}

\begin{algorithm}[h]
  \DontPrintSemicolon
  \SetAlgorithmName{Algorytm}{}

  \KwData{ n }

  \KwResult{ $n$-ta liczba Fibonacciego }

  $M \leftarrow \begin{bmatrix}
                  1 & 1 \\
                  1 & 0
                \end{bmatrix}$\;
  $M' \leftarrow \texttt{exp\_by\_squaring(M, n - 1)}$\;
  $M'' \leftarrow M' \times \begin{bmatrix}1 \\ 0\end{bmatrix}$\;
  \KwRet{$M''_{1,1}$}\;

  \caption{Procedura \texttt{get\_fibonacci}}
\end{algorithm}
Mimo że powyższy algorytm działa w czasie $\Theta(\log{n})$, warto mieć na uwadze fakt, że liczby Fibonacciego 
rosną wykładniczo. W praktyce oznacza to pracę na liczbach przekraczających długość słowa maszynowego.

Zaprezentowaną metodę można uogólnić na dowolne ciągi, które zdefiniowane są przez liniową 
kombinację skończonej liczby poprzednich elementów. Wystarczy znaleźć odpowiednią macierz $M$. 
Dla ciągów postaci:
\begin{equation*}
  G_{n + 1} = a_n G_n + a_{n - 1} G_{n - 1} + ... + a_{n - k} G_{n - k}
\end{equation*}
wygląda ona następująco:
\begin{align*}
  M
  =
  \begin{bmatrix}
    a_n    & a_{n - 1} & a_{n - 2} & \dots  & a_{n - k + 1} & a_{n - k} \\
    1      & 0         & 0         & \dots  & 0             & 0 \\
    0      & 1         & 0         & \dots  & 0             & 0 \\
    0      & 0         & 1         & \dots  & 0             & 0 \\
    \vdots & \vdots    & \vdots    & \ddots & \vdots        & \vdots \\
    0      & 0         & 0         & \dots  & 1             & 0
  \end{bmatrix}
\end{align*}
Dowód tej konstrukcji pozostawiamy Czytelnikowi jako ćwiczenie.


\section{Algorytm Strassena}
\sectionauthor{Krzysztof Piecuch}

\label{sec:strassen}

Z mnożeniem macierzy mieliście już prawdopodobnie do czynienia na Algebrze.
\comment{Symbol mnożenia macierzy jest inny niż w rozdziale o liczbach fibonacciego.}
Mając dane macierze $A$ (o rozmiarze $n \times m$) oraz $B$ (o rozmiarze $m \times p$) nad ciałem liczb rzeczywistych, chcemy policzyć ich iloczyn:
\[
 A \cdot B = C
\]
gdzie elementy macierzy $C$ (o rozmiarze $n \times p$) zadane są wzorem:
\[
 c_{i,j} = \sum_{r=1}^{m} a_{i,r} \cdot b_{r,j}
\]
Przykładowo:
\[
  \begin{bmatrix}
    1 & 0 & 2 \\
    0 & 3 & 1 \\
  \end{bmatrix}
\cdot
  \begin{bmatrix}
    3 & 1 \\
    2 & 1 \\
    1 & 0
  \end{bmatrix}
=
  \begin{bmatrix}
     (1 \cdot 3  +  0 \cdot 2  +  2 \cdot 1) & (1 \cdot 1   +   0 \cdot 1   +   2 \cdot 0) \\
     (0 \cdot 3  +  3 \cdot 2  +  1 \cdot 1) & (0 \cdot 1   +   3 \cdot 1   +   1 \cdot 0) \\
  \end{bmatrix}
=
  \begin{bmatrix}
    5 & 1 \\
    7 & 3 \\
  \end{bmatrix}
\]
Korzystając prosto z definicji możemy napisać następujący algorytm mnożenia dwóch macierzy:

\begin{algorithm}[H]
  \DontPrintSemicolon
  \SetAlgorithmName{Algorytm}{}
  
  \KwData{ $A$, $B$ - macierze o rozmiarach $n \times m$ oraz $m \times p$ }
  
  \KwResult{ $C = A \cdot B$ }
  
  \For{$i \leftarrow 1$ to $n$}
  {
     \For{$j \leftarrow 1$ to $p$}
     {
	$C[i][j] \leftarrow 0$\;
	\For{$r \leftarrow 1$ to $m$}
	{
	  $C[i][j] \leftarrow C[i][j] + A[i][r] \cdot B[r][j]$\;
	}
     }
  }
  
  \caption{Naiwny algorytm mnożenia macierzy}
  \label{alg-mnozenie-macierzy}
\end{algorithm}
Powyższy algorytm działa w czasie $\Theta(n \cdot p \cdot m)$ ($\Theta(n^3)$ dla macierzy kwadratowych o boku $n$).
Korzystając ze sprytnej sztuczki, jesteśmy w stanie zmniejszyć złożoność naszego algorytmu.

Zacznijmy od założenia, że rozmiar macierzy jest postaci $2^k \times 2^k$.
Jeśli macierze nie są takiej postaci, to możemy uzupełnić brakujące wiersze i kolumny zerami.
Następnie podzielmy macierze na cztery równe części:
\[
  A = 
  \begin{bmatrix}
    A_{1,1} & A_{1,2} \\
    A_{2,1} & A_{2,2} \\
  \end{bmatrix} \enspace  
  B = 
  \begin{bmatrix}
    B_{1,1} & B_{1,2} \\
    B_{2,1} & B_{2,2} \\
  \end{bmatrix} \enspace
  C = 
  \begin{bmatrix}
    C_{1,1} & C_{1,2} \\
    C_{2,1} & C_{2,2} \\
  \end{bmatrix}
\]
Każda z części jest rozmiaru $2^{k-1} \times 2^{k-1}$.
Ponadto wzór na każdą część macierzy $C$ wyraża się wzorem:
\[
 C_{i,j} = A_{i,1} \cdot B_{1,j} + A_{i,2} \cdot B_{2,j}
\]

Czy wzór ten umożliwia nam ułożenie efektywnego algorytmu mnożenia macierzy?
Nie.
W algorytmie mamy do policzenia $4$ podmacierze macierzy $C$.
Każda podmacierz wymaga $2$ mnożeń oraz jednego dodawania.
Dodawanie macierzy możemy w prosty sposób zrealizować w czasie $\Theta(n^2)$.
Mnożenie podmacierzy możemy wykonać rekurencyjnie.
Taki algorytm będzie działał w czasie $T(n) = 8\cdot T(n/2) + \Theta(n^2)$ czyli $\Theta(n^3)$.
Osiągnęliśmy tą samą złożoność czasową jak w przypadku algorytmu liczącego iloczyn wprost z definicji.

Algorytm Strassena osięga lepszą złożoność asymptotyczną przez pozbycie się jednego z mnożeń.
Algorytm ten liczy następujące macierze:
\begin{align*}
 M_1 &= (A_{1,1} + A_{2,2}) \cdot (B_{1,1} + B_{2,2}) \\
 M_2 &= (A_{2,1} + A_{2,2}) \cdot B_{1,1} \\
 M_3 &= A_{1,1} \cdot (B_{1,2} - B_{2,2}) \\
 M_4 &= A_{2,2} \cdot (B_{2,1} - B_{1,1}) \\
 M_5 &= (A_{1,1} + A_{1,2}) \cdot B_{2,2} \\
 M_6 &= (A_{2,1} - A_{1,1}) \cdot (B_{1,1} + B_{1,2}) \\
 M_7 &= (A_{1,2} - A_{2,2}) \cdot (B_{2,1} + B_{2,2})
\end{align*}
Do policzenia każdej z tych macierzy potrzebujemy jednego mnożenia i co najwyżej dwóch dodawań/odejmowań.
Podmacierze macierzy $C$ możemy policzyć teraz w następujący sposób:
\begin{align*}
 C_{1,1} &= M_1 + M_4 - M_5 + M_7 \\
 C_{1,2} &= M_3 + M_5 \\
 C_{2,1} &= M_2 + M_4 \\
 C_{2,2} &= M_1 - M_2 + M_3 + M_6
\end{align*}
Wykonując proste przekształcenia arytmetyczne, możemy dowieść poprawności powyższych równań.

Używając powyższych wzorów, możemy skonstruować algorytm rekurencyjny.
Będzie on dzielić macierze $A$ oraz $B$ o rozmiarze $2^k \times 2^k$ na cztery równe części.
Następnie policzy on macierze $M_i$.
Tam, gdzie będzie musiał dodawać/odejmować użyje on algorytmu działającego w czasie $\Theta(n^2)$.
Tam, gdzie będzie musiał mnożyć - wywoła się on rekurencyjnie.
Na podstawie macierzy $M_i$ policzy macierz $C$.
Ponieważ wykona dokładnie $7$ mnożeń oraz stałą ilość dodawań, jego złożoność obliczeniowa będzie wyrażała się wzorem rekurencyjnym $T(n) = 7\cdot T(n/2) + \Theta(n^2)$.
Korzystając z twierdzenia o rekurencji uniwersalnej otrzymujemy złożoność $\Theta(n^{\log_2 7})$ czyli około $\Theta(n^{2.81})$.

\section{Model afinicznych drzew decyzyjnych}
\sectionauthor{Krzysztof Piecuch}

\label{sec:elementuniqness}

Zdefiniujmy następujący problem (ang. element uniqness).
Mając daną tablicę \texttt{T[0..n-1]} liczb rzeczywistych, odpowiedzieć na pytanie czy istnieją w tablicy dwa elementy, które są sobie równe.
Pierwsze rozwiązanie jakie przychodzi wielu ludziom do głowy, to posortować tablicę \texttt{T} a następnie sprawdzić sąsiednie elementy.
Algorytm ten rozwiązuje nasz problem w czasie $\Theta(n \log n)$.
Pytanie - czy da się szybciej?
W niniejszym rozdziale udowodnimy, że w modelu afinicznych drzew decyzyjnych problemu nie da się rozwiązać lepiej.

W modelu afinicznych drzew decyzyjnych, w każdym zapytaniu możemy wybrać sobie $n+1$ liczb: $c$, $a_0$, $a_1$, $\dots$, $a_{n-1}$, a następnie zapytać czy
\[
 c + \sum_{i=0}^{n-1} a_i t_i \geq 0
\]
gdzie $t_i$ to elementy tablicy \texttt{T}.
Gdybyśmy użyli terminologii algebraicznej, to powiedzielibyśmy, że $t$ jest punktem w przestrzeni $\mathbb{R}^n$, lewa strona powyższej nierówności to przekształcenie afiniczne,
a zbiór wszystkich punktów z $\mathbb{R}^n$, które spełniają tą nierówność to półprzestrzeń afiniczna.
Jeśli na Algebrze nie wyrobiliście sobie jeszcze intuicji, to w $\mathbb{R}^2$ półprzestrzeń afiniczną otrzymujemy przez narysowanie dowolnej prostej i wzięcie wszystkich elementów z jednej ze stron.
Podobnie w $\mathbb{R}^3$ półprzestrzeń afiniczną otrzymujemy poprzez narysowanie dowolnej płaszczyzny, a następnie wzięcia wszystkich elementów z jednej ze stron.
W wyższych wymiarach wygląda to analogicznie.
\tizkboxwithcaption{tikz/uniqnesstree.tikz}{
Przykład afinicznego drzewa decyzyjnego.
W wierzchołkach wewnętrznych mamy zapytanie $(c^j, a_i^j)$.
W zależności od tego czy $c^k + \sum_{i=0}^{n-1} a^k_i t_i \geq 0$ czy też nie, idziemy odpowiednio w lewo lub w prawo.
Liście zawierają odpowiedź naszego algorytmu.
}

Algorytm używający tego typu porównań można zapisać za pomocą drzewa (rys. \ref{uniqness-tree}).
Zaczynamy z korzenia tego drzewa.
W każdym wierzchołku wewnętrznym zadajemy zapytanie.
W zależności od tego czy odpowiedż na pytanie była pozytywna czy negatywna, idziemy w drzewie w lewo lub w prawo.
Gdy dojdziemy do liścia w drzewie otrzymujemy nasze rozwiązanie (tak lub nie).
Takie drzewo będziemy nazywać afinicznym drzewem decyzyjnym.

Aby było nam łatwiej zdefiniować główny lemat naszego rozdziału, zdefiniujemy sobie dwa pojęcia.
\begin{definition}
 Mówimy, że punkt $t \in \mathbb{R}^n$ \textbf{osiąga} liść $l$ w afinicznym drzewie decyzyjnym, jeśli algorytm uruchomiony dla punktu $t$ dochodzi do liścia $l$.
\end{definition}
\begin{definition}
 Mówimy, że podzbiór $C \subseteq \mathbb{R}^n$ jest \textbf{zbiorem wypukłym}, jeśli dla dowolnych punktów $u, v \in C$ oraz dowolnej liczby rzeczywistej $0 \leq \alpha \leq 1$ punkt $\alpha \cdot u + (1 - \alpha) v$ także należy do $C$.
\end{definition}
Pierwsza z definicji pozwala nam mówić o elementach, które trafiają do tego samego liścia, a druga to sformalizowane pojęcie wypukłości znane z liceum.
Uzbrojeni w nowe definicje, możemy przejść do obiecanego lematu:

\begin{lemma}
 Zbiór punktów osiągających liść $l$ w afinicznym drzewie decyzyjnym, jest zbiorem wypukłym.
 \label{uniqness-lemma}
\end{lemma}

\begin{proof}
 Weźmy dowolne afiniczne drzewo decyzyjne i wybierzmy w nim dowolny liść $l$.
 Dobrze wiemy, że istnieje dokładnie jedna ścieżka prosta z korzenia do tego liścia.
 Weźmy dowolny wierzchołek wewnętrzny $w$ na tej ścieżce i dowolne punkty $u$ oraz $v$, które osiągają liść $l$.
 W końcu weźmy dowolną liczbę rzeczywistą $0 \leq \alpha \leq 1$.
 Załóżmy ponadto, że ścieżka z korzenia do liścia $l$ w wierzchołku $w$ skręca w lewo (zatem zapytanie zadane w wierzchołku $w$ punkty $u$ oraz $v$ otrzymały odpowiedź twierdzącą).
 Przypadek przeciwny jest analogiczny.
 Wiemy zatem, że
\[
 c^w + \sum_{i=0}^{n-1} a_i^w u_i \geq 0
\]
 oraz, że
\[
 c^w + \sum_{i=0}^{n-1} a_i^w v_i \geq 0
\]
 Ponieważ $\alpha \geq 0$ możemy przemnożyć pierwsze równanie przez $\alpha$:
\[
 \alpha c^w + \sum_{i=0}^{n-1} a_i^w \alpha u_i \geq 0
\]
 a ponieważ $1 - \alpha \geq 0$ możemy drugie równanie przemnożmyć przez $1 - \alpha$:
\[
 (1 - \alpha) c^w + \sum_{i=0}^{n-1} a_i^w (1 - \alpha) v_i \geq 0
\]
Teraz sumując oba równania otrzymujemy:
\[
 c^w + \sum_{i=0}^{n-1} a_i^w (\alpha u_i + (1 - \alpha) v_i) \geq 0
\]
Zatem punkt $\alpha \cdot u + (1 - \alpha) v$ również w wierzchołku $w$ skręci w tą samą stronę co punkty $u$ oraz $v$.
Ponieważ wybraliśmy dowolny wierzchołek $w$, to punkt $\alpha \cdot u + (1 - \alpha) v$ osiągnie liść $l$.
Stąd zbiór wszystkich punktów, które osiągają liść $l$ w afinicznym drzewie decyzyjnym, jest zbiorem wypukłym.
\end{proof}

Wyobraźmy sobie, że na płaszczyznę $\mathbb{R}^2$ kładziemy półpłaszczyzny.
Wtedy przecięcie dowolnej liczby półpłaszczyzn jest zbiorem wypukłym.
Podobnie jeśli w przestrzeni $\mathbb{R}^3$ wyznaczymy sobie półprzestrzenie, to ich przecięcie będzie tworzyło zbiór wypukły.
Lemat \ref{uniqness-lemma} mówi, że tak samo się dzieje w każdej przestrzeni $\mathbb{R}^n$.

Ten lemat za chwilę okaże się dla nas kluczowy, gdyż za jego pomocą udowodnimy, że jeśli afiniczne drzewo decyzyjne poprawnie rozwiązuje problem element uniqness to musi posiadać conajmniej $n!$ liści.
Oznacza to, że wysokość takiego drzewa musi wynosić conajmniej $O(n \log n)$.

\begin{lemma}
 \label{permutation-lemma}
 Niech $\{r_0, r_1, \ldots, r_{n-1}\}$ będzie $n$ elementowym zbiorem liczb rzeczywistych i niech $(a_0, a_1, \ldots, a_{n-1})$ oraz $(b_0, b_1, \ldots, b_{n-1})$ będą dwoma różnymi permutacjami liczb z tego zbioru.
 W każdym afinicznym drzewie decyzyjnym poprawnie rozwiązującym problem element uniqness, punkty $a$ oraz $b$ osiągają różne liście w drzewie.
\end{lemma}

\begin{proof}
 Dowód niewprost.
 Załóżmy, że $a$ oraz $b$ osiągają ten sam liść w drzewie.
 Liść ten musi odpowiadać przecząco na zadany problem, gdyż ani $a$ ani $b$ nie zawierają dwóch tych samych elementów.
 Ponieważ $a$ oraz $b$ składają się z tych samych liczb rzeczywistych i różnie się od siebie permutacją, to muszą istnieć takie indeksy $i$ oraz $j$, że $a_i > a_j$ oraz $b_i < b_j$.
 Weźmy następującą wartość $\alpha$:
 \[
  \alpha = \frac{b_j - b_i}{(a_i - a_j) + (b_j - b_i)}
 \]
 Wykonując proste przekształcenia arytmetyczne, możemy przekonać się, że $0 < \alpha < 1$.
 Oznacza to, na mocy lematu \ref{uniqness-lemma}, że punkt $\alpha a + (1 - \alpha)b$ również osiąga ten sam liść co punkty $a$ i $b$.
 Ponieważ jednak zachodzi:
 \[
  \alpha a_i + (1 - \alpha) b_i = \alpha a_j + (1 - \alpha) b_j
 \]
 (o czym można się przekonać wykonując proste przekształcenia arytmetyczne), odpowiedź algorytmu dla tego punktu powinna być twierdząca.
 Zatem afiniczne drzewo decyzyjne dla tego punktu zwraca złą odpowiedź.
 Sprzeczność z założeniem, że drzewo rozwiązywało problem poprawnie.
\end{proof}

Weźmy dowolny $n$ elementowy zbiór liczb rzeczywistych.
Na mocy Lematu \ref{permutation-lemma} każda permutacja tych liczb musi osiągać inny liść w afinicznym drzewie decyzyjnym poprawnie rozwiązującym problem element uniqness.
Oznacza to, że liczba liści w takim drzewie musi wynosić przynajmniej $n!$.
Zatem wysokość takiego drzewa musi wynosić conajmniej $\Omega(n \log n)$.

\section{Problem plecakowy}

\label{sec:plecaki}

W tym rozdziale opiszemy jeden z najbardziej znanych problemów optymalizacyjnych, czyli tak zwany problem plecakowy.

Załóżmy, że jesteśmy złodziejem, który w trakcie włamania chce wypełnić swój plecak wielkości $W$ przedmiotami o jak największej wartości.
Mamy $M$ typów przedmiotów i plecak wielkości $W$.
Każdy z przedmiotów typu $i$ ma swoją wartość $v_i> 0$ i wielkość $w_i> 0$.
To ile przedmiotów typu $i$ zabierzemy będziemy oznaczać przez $x_i\epsilon N$.
Dyskretny plecakowy posiada trzy popularne wersje.


\begin{itemize}
  \item 0-1 dyskretny problem plecakowy, w którym $x_i \epsilon \left \{ 0, 1 \right \}$.
  Tę wersję możemy interpretować, jako okradanie galerii sztuki.
  \item Ograniczony problem placakowy, w którym, dla każdego $i$ mamy podaną ilość przedmiotów danego typu - $c_i$, czyli $x_i \leq c_i$.
  Tę wersję możemy interpretować, jak zwykłą kradzież ze sklepu rtv.
  \item Nieograniczony problem plecakowy, w którym $x_i$ nie ma górnego ograniczenia.
  Tę wersję możemy intepretować, jako kradzież kopii Windowsa, gdzie plecakiem będzie dysk.
\end{itemize}

\subsection{Ciągły problem plecakowy}
Problem plecakowy posiada także wersję ciągłą, w której możemy zabrać część, zamiast całości przedmiotu.
Specyfikacja jest analogiczna do wariantu dyskretnego, z tą różnicą, że $x_i \epsilon R_+ \cup \left \{ 0 \right \}$.
O tym wariancie możemy myśleć jak o kradzieży płynnych chemikaliów.

\subsection{Próba rozwiązania zachłannego}
Mimo, że intuicyjnym byłoby dla każdego przedmiotu wyliczyć stosunek wartości do wielkości i branie przedmiotów
o najlepszym współczynniku rozwiązanie takie jest nieprawidłowe.
Prostym kontrprzykładem będzie sytuacja, w której plecak ma wielkość $w = 10$ i do dyspozycji mamy $3$ przedmioty;
$v_1=9$, $v_2=v_3=5$, $w_1=6$, $w_2=w_3=5$. W każdym z wariantów problemu postępując w ten sposób zachłannie wybierzemy przedmiot $1$,
otrzymując plecak o wartości $9$, zamiast wybierając $2$ i $3$ otrzymując plecak o wartości $10$.
Co ciekawe podejście zachłanne sprawdza się w przypadku problemu ciągłego.

\subsection{Rozwiązanie dynamiczne}
\subsubsection{0-1}
Stwórzmy dwuwymiarową tablicę A.
$A[i, w]$ będzie oznaczała maksymalną wartość plecaku wielkości $w$, rozważając $i$ pierwszych przedmiotów.
Uznajmy przy tym, że dla indeksów ujemnych A będzie zwracać $-\infty$.

Stwórzmy zależność rekurencyjną.
\[A[0, w] = 0\]
Gdyż biorąc zero przedmiotów nie otrzymamy żadnej wartości dodanej.
\[A[i, 0] = 0\]
\[A[i, w] = max(A[i - 1, w], A[i - 1, w - w_i] + v_i)\]
Czyli rozważając dodanie następnego przedmiotu sprawdzamy, czy najbardziej wartościowy po jego dołożeniu jest wart więcej, niż najbardziej wartościowy plecak bez tego elementu.

Wynik będzie w komórce $A[M, W]$.

\subsubsection{Ograniczony}
\[A[0, w] = 0\]
\[A[i, 0] = 0\]
\[A[i, w] = \max_{0 \leq n \leq c_i}(A[i - 1, w - w_i \cdot n] + v_i \cdot n)\]
Podobnie, do wariantu 0-1, z tym, że tutaj przy i-tym elemencie rozważamy jego każdą możliwą ilość.

Wynik będzie w komórce $A[M, W]$.

\subsubsection{Nieograniczony}
W wariance nieograniczonym; jako, że nie musimy kontrolować, ile elementów wzięliśmy, sytuacja się upraszcza.
Wystarczy nam jednowymiarowa tablica $A$; $A[w]$ będzie oznaczała wartość najcennieszego plecaka o wielkości $w$.
\[A[0] = 0\]
\[A[w] = \max_{w_i \le w}(A[w - w_i] + c_i) \]

Wynik będzie w komórce $A[W]$.

\subsection{Złożoność}
\begin{itemize}
  \item 0-1
  
  Złożoność pseudowielomioanowa $O(M \cdot W)$.
  \item Ograniczony
  
  Złożoność pseudowielomioanowa $O(M \cdot W)$.
  \item Nieograniczony
  
  Złożoność pseudowielomioanowa $O(M \cdot W \cdot \max\limits_{1 \le i \le n}(c_i))$.
\end{itemize}
\comment{Wydaje mi się, że na wykładzie było użyte O, a nie $\Theta$, więc tutaj zrobiłem tak samo}


\section{Lazy Select}
\sectionauthor{Krzysztof Piecuch}

\label{sec:lazy}

Naszym celem w tym rozdziale będzie znalezienie mediany w nieuporządkowanej tablicy w czasie liniowym.
Przedstawimy algorytm zrandomizowany, który albo (z dużym prawdopodobieństwem) zwróci poprawną medianę, albo stwierdzi, że poszukiwania zakończyły się fiaskiem.
Proszę mieć na uwadze fakt, że algorytm może posłużyć do znalezienia $k$-tego elementu, dla dowolnej wartości $k$.

Mediany będziemy szukać w zbiorze $A$ i tradycyjnie załóżmy, że ma on $n$ elementów.
Ponadto oznaczmy sobie szukaną medianę jako $m$.
Na początku załóżmy, że mamy pewne dodatkowe informacje zesłane nam przez Boga, wyrocznię, wróżkę albo super-doktoranta (kto co woli).
Super-doktorant pokazuje nam dwa elementy ze zbioru $A$: $l$ oraz $u$.
Ponadto wyrocznia gwarantuje nam, że te dwa elementy mają następujące własności:
\begin{itemize}
 \item $l \leq m \leq u$
 \item $|C| \leq 4 \cdot n^{3/4} \enspace gdzie \enspace C = \{a \in A: l \leq a \leq u\}$
\end{itemize}
Czyli mediana zbioru $A$ znajduje się między podanymi elementami, oraz ilość elementów pomiędzy $l$ a $u$ jest nie większa od $4 \cdot n^{3/4}$ (elementy te oznaczamy jako zbiór $C$).
Czy z taką boską pomocą student poradzi sobie ze znalezieniem mediany w zbiorze?
Niewykluczone.
Wystarczy bowiem tylko posortować teraz elementy leżące pomiędzy $l$ a $u$.
Jeśli wybierzemy swoją ulubioną efektywną metodę sortowania, to zajmie nam to $\Theta(n^{3/4} \log n^{3/4})$ czasu.
Czyli $o(n)$ (małe $o(n)$ oznacza czas ostro mniejszy od liniowego - algorytm działa szybciej niż liniowo).
Następnie z posortowanego zbioru wybierzemy odpowiedni element.
W tym celu policzmy $d_l$ - liczbę elementów mniejszych od $l$ w zbiorze $A$ oraz $d_u$ - liczbę elementów większych od $u$ w zbiorze $A$.
Mediana zbioru $A$ to $n/2 - d_l + 1$-szy element posortowanego zbioru $C$.

W swoim (słynnym) skrypcie do programowania funkcyjnego Marcin Kubica napisał, że informatyka to dziedzina magii.
W ``Nowej encyklopedii powszechnej PWN'' możemy znaleźć następujące określenie magii: ``zespół działań zasadniczo pozaempirycznych, symbolicznych, których celem jest bezpośrednie osiągnięcie (...) pożądanych skutków (...)''.
Wyróżniamy przy tym następujące składniki działań magicznych:
\begin{itemize}
 \item zrytualizowane działania (manipulacje)
 \item materialne obiekty magiczne (amulety, eliksiry itp.)
 \item reguły obowiązujące przy praktykach magicznych (zasady czystości, reguły określające czas i miejsce rytuałów)
 \item formuły tekstowe mające moc sprawczą (zaklęcia)
\end{itemize}
Programowanie mieści się w ramach ostatniego z powyższych punktów.
Programy komputerowe są zapisanymi w specjalnych językach zaklęciami.
Zaklęcia te są przeznaczone dla specjalnego rodzaju duchów żyjących w komputerach, zwanymi procesami obliczeniowymi.
Ze względu na to, że komputery są obecnie produkowane seryjnie, stwierdzenie to może budzić kontrowersyjne.
Zastanówmy się jednak chwilę, czym charakteryzują się duchy.
Są to obiekty niematerialne, zdolne do działania.
Procesy obliczeniowe ewidentnie spełniają te warunki: nie można ich dotknąć, ani zobaczyć, nic nie ważą, a można obserwować skutki ich dziłania, czy wręcz uzyskać od nich interesujące nas informacje.
Nota bene, w trakcie zajęć można się spotkać również z przejawami pozostałych wymienionych składników działań magicznych.
``Logowanie'' się do sieci oraz wyłączanie komputera to działania o wyraźnie rytualnym charakterze.
Przedmioty takie jak indeks, czy karta zaliczeniowa wydają się mieć iście magiczną moc.

Czemu wam o tym pisze?
Bo teraz zacznie się magia :)
Jeśli bowiem nikt nam nie poda wartości $l$ oraz $u$ to będziemy musieli zabawić się w wróżkę i sami te wartości wyczarować.
Zaczniemy od utworzenia zbioru $R$ do którego zaczniemy wrzucać losowo z powtórzeniami elementy zbioru $A$ z jednakowym prawdopodobieństwem.
Będziemy robić to tak długo, aż zbiór $R$ będzie miał $n^{3/4}$ elementów.
Następnie zbiór ten posortujemy w czasie $o(n)$.
Jako wartość $l$ weźmiemy $n^{3/4}/2 - \sqrt{n}$-szy element zbioru $R$, a jako wartość $r$ weźmiemy $n^{3/4}/2 + \sqrt{n}$-szy element zbioru $R$.

\begin{algorithm}
  \DontPrintSemicolon
  \SetAlgorithmName{Algorytm}{}
  
  \KwData{ A, n }
  
  \KwResult{ m - mediana tablicy A }
  
   \For{$ i \leftarrow 0$ to $n^{3/4}$}
   {
     $R[i] \leftarrow A[random(n)]$\;
   }
   sort(R, $n^{3/4}$)\;
   $l \leftarrow R[n^{3/4}/2 - \sqrt{n}]$\;
   $u \leftarrow R[n^{3/4}/2 + \sqrt{n}]$\;
   $s \leftarrow d_l \leftarrow d_r \leftarrow 0$\;
   \For{$ i \leftarrow 0$ to $n$}
   {
     \lIf{$A[i] < l$}
     {
      $d_l \leftarrow d_l + 1$
     }
     \lElseIf{$A[i] > u$}
     {
      $d_u \leftarrow d_u + 1$
     }
     \Else
     {
      $C[s] = A[i]$\;
      $s \leftarrow s + 1$\;
     }
   }
   \lIf{$d_l > n/2$ or $d_u > n/2$ or $s > 4 * n^{3/4}$}
   {
     \Return{fail}
   }
   sort(C, $s$)\;
   \Return $C[n/2 - d_l + 1]$\;
  \caption{Procedura \texttt{lazy\_select}}
  \label{lazy-select}
\end{algorithm}

Co mogło pójść źle?
Nasze wartości $l$ oraz $u$ miały spełniać dwie własności.
Mediana miała znajdować się pomiędzy tymi wartościami oraz liczba elementów pomiędzy tymi wartościami miała być mniejsza od $4 \cdot n^{3/4}$.
Pierwszy warunek nie będzie spełniony jeśli którakolwiek z liczb $d_l$ albo $d_u$ jest większa od $n/2$.
Policzmy prawdopodobieństwo tego, że $d_l > n/2$.
Drugie prawdopodobieństwo liczy się symetrycznie.

Przez $Y$ oznaczmy sobie zbiór elementów $R$ mniejszy bądź równy medianie.
Formalnie: $Y = \{ x \in R: x \leq m\}$.
Skoro $d_l> n/2$, to $|Y| < n^{3/4}/2 - \sqrt{n}$.
Niech $X_i$ będzie zmienną losową oznaczającą, że i-ty element wybrany do zbioru $R$ był mniejszy bądź równy medianie.
Wtedy $P(X_i = 1) = E[X_i] = ((n+1)/2)/n = 1/2 + 1/2\cdot n$.
Oczywiście $|Y| = \sum X_i$. Można również zauważyć, że zmienne $X_i$ są niezależne.
Stąd łatwo możemy policzyć wartość oczekiwaną oraz wariancję $|Y|$.
\begin{align*} 
    E[|Y|] & = E[\mathsmaller{\sum X_i}] = \sum E[X_i] = \frac{1}{2} \cdot n^{\frac{3}{4}} + \frac{1}{2}n^{-\frac{1}{4}} \\
    Var[X_i] & = E[X_i^2] - (E[X_i])^2 = \frac{1}{2} + \frac{1}{2 \cdot n}- (\frac{1}{2} + \frac{1}{2 \cdot n})^2 \\
             & = \frac{1}{4} - \frac{1}{4 \cdot n^2} < \frac{1}{4} \\
 Var[|Y|] & = n^{\frac{3}{4}} \cdot Var[X_i] < \frac{1}{4} \cdot n^{\frac{3}{4}}
\end{align*}
Przy założeniu, że $n \neq 0$. Teraz korzystając z nierówności Czybyszewa otrzymujemy interesującą nas wartość:
\begin{align*}
 P\left(fail_1\right) = & P\left(|Y| < \left(\frac{1}{2} \cdot n^{\frac{3}{4}} - \sqrt{n}\right)\right) \\
                      = & P\left(\left(\frac{1}{2} \cdot n^{\frac{3}{4}} - |Y|\right) > \sqrt{n}\right) \\
                      \leq & P\left(\left(\frac{1}{2} \cdot n^{\frac{3}{4}} + \frac{1}{2} \cdot n^{-\frac{1}{4}} - |Y|\right) > \sqrt{n}\right) \\
                      = & P\left(\left(E[|Y|] - |Y|\right) > \sqrt{n}\right)\\
                      \leq & P\left(|E[|Y|] -|Y| | > \sqrt{n}\right) \\
                      \leq & Var[|Y|] / (\sqrt{n})^2 \\
                      \leq & \frac{1}{4} \cdot n^{\frac{3}{4}} / n \\
                      = & \frac{1}{4} \cdot n^{-\frac{1}{4}}
\end{align*}

Teraz rozważymy drugą sytuację, która mogła pójść źle - zbiór $C$ okazał się za duży.
Oznacza to, że albo conajmniej $2 \cdot n^{3/4}$ elementów $C$ jest większych od mediany albo, że conajmniej $2 \cdot n^{3/4}$ elementów $C$ jest mniejszych od mediany.
Mamy zatem tak samo jak w poprzednim akapicie dwie symetryczne sytuacje.
Rozważymy pierwszą z nich.

Dowód będzie analogiczny do dowodu poprzedniego.
Przez $Y$ oznaczmy sobie zbiór elementów $R$ większych od $n/2 + 2\cdot n^{3/4}$-szego elementu zbioru $A$ (w zbiorze posortowanym).
Skoro conajmniej $2 \cdot n^{3/4}$ elementów $C$ jest większych od mediany to znaczy, że $|Y| \geq n^{3/4} - (n^{3/4}/2 + \sqrt{n}) = n^{3/4}/2 - \sqrt{n}$.
Niech $X_i$ będzie zmienną losową oznaczającą, że i-ty element wybrany do zbioru $R$ jest większy od $n/2 + 2\cdot n^{3/4}$-szego elementu zbioru $A$.
Wtedy $P(X_i = 1) = (n/2 - 2 \cdot n^{3/4}) / n = 1/2 - 2 \cdot n^{-1/4}$.
I tak jak ostatnio  $|Y| = \sum X_i$.
Teraz wartość oczekiwana oraz wariancja:
\begin{align*}
 E[|Y|] & = \frac{1}{2} \cdot n^{\frac{3}{4}} - 2 \cdot \sqrt{n} \\
 Var[|Y|] & = n^{\frac{3}{4}} Var[X_i] < \frac{1}{4} \cdot n^{\frac{3}{4}}
\end{align*}
Wykonując podobne obliczenia jak ostatnio otrzymujemy wartość:
\begin{align*}
 P\left(fail_2\right) \leq \frac{1}{4} \cdot n^{-\frac{1}{4}}
\end{align*}
Ostatecznie
\begin{align*}
 P\left(fail\right) \leq 2 \cdot P\left(fail_1\right) + 2 \cdot P\left(fail_2\right) \leq n^{-\frac{1}{4}}
\end{align*}

Otrzymaliśmy algorytm, który działa w czasie $\Theta(n)$ i zwraca poprawną medianę lub z prawdopodobieństwem mniejszym od $n^{-1/4}$ zwraca, że się pomylił.



\section{Counting sort}
\sectionauthor{Dominika Wójcik}

\label{sec:countingsort}

W tym rozdziale można zapoznać się z algorytmem sortowania przez zliczanie (ang. \textit{counting sort}). 
W metodzie tej zakładamy, że każdy z $n$ elementów ciągu jest liczbą całkowitą z przedziału od 0 do k.
Sortowanie przez zliczanie działa w czasie $\Theta(n+k)$.
Jeśli jednak $k \in O(n)$, to wtedy jest to czas liniowy.

Być może niektórych niepokoi fakt, że osiągnęliśmy lepszą złożoność czasową niż mówi o tym dolna granica sortowania, czyli $\Theta(n \log n)$.
Zauważmy jednak, że w tym wypadku mamy dodatkowe założenie.
Do posortowania mamy liczby całkowite z danego zakresu.
Ponadto dolna granica mówi nam o wykonywaniu algorytmu sortowaniu za pomocą porównań.
W sortowaniu przez zliczanie nie korzystamy z porównań, a zatem ta granica nie dotyczy tego algorytmu.

\textbf{Idea.} W sortowaniu przez zliczanie głównym pomysłem jest wyznaczenie dla każdego $x$ z ciągu wejściowego liczby elementów mniejszych od $x$.
Dlaczego?
Załóżmy, że policzyliśmy, że jest 10 elementów mniejszych od $x$.
Z tego możemy już wywnioskować, że element $x$ będzie na 11 miejscu w tablicy.
Jeśli chcemy dopuścić powtórzenia w ciągu wejściowym musimy dokonać drobnej modyfikacji i zamiast pamiętać liczby mniejsze od $x$, pamiętamy te mniejsze lub równe.
Spójrzmy na pseudokod algorytmu.

\begin{algorithm}[h]

  \DontPrintSemicolon
  
  \SetAlgorithmName{Algorytm}{}
  
  \KwData{ tablica $A$, liczba $k$}

  \KwResult{ posortowana tablica $B$ }
  
  \For{$i \leftarrow 0 .. k$}   {
    $C[i] \leftarrow 0$;
  }
   \For{$j \leftarrow 1 .. A.length$}   {
    $C[A[j]] \leftarrow C[A[j]]+1$;
  }
   \For{$i \leftarrow 1 .. k$}   {
    $C[i] \leftarrow C[i]+C[i-1]$;
  }
   \For{$j \leftarrow A.length .. 1$}   {
    $B[C[A[j]]] \leftarrow A[j]$\;
    $C[A[j]] \leftarrow C[A[j]]-1$;
  }
  
  \caption{Procedura \texttt{counting sort}}
  \label{alg-count}
\end{algorithm}

A teraz pokażemy, czemu dokładnie służą kolejne pętle for w pseudokodzie.

\begin{enumerate}
\item Na początku inicjujemy tablicę $C$.
\item W następnej pętli \texttt{for} przechodzimy po kolei po wszystkich elementach i dla wartości $x$ danego elementu, wartość tablicy $C[x]$ zwiększamy o jeden.
      A zatem po wykonaniu drugiej pętli tablica $C$ zawiera liczbę wystąpień elementów np. $x$ występuje $C[x]$ razy.
\item Obliczamy ile jest elementów mniejszych lub równych $x$. Odbywa się to przez sumowanie komórek $C[x]$ z wartościami mniejszych indeksów tablicy.
\item W ostatniej pętli \texttt{for} umieszczamy elementy na właściwych pozycjach w tablicy wyjściowej $B$.
      Jeśli w tablicy wejściowej A nie ma powtórzeń, to dla każdego $A[j]$ wartość $C[A[j]]$ jest poprawnym ostatecznym numerem pozycji w tablicy $B$.
      Jeśli jednak dopuszczamy powtórki musimy tę wartość dekrementować za każdym razem gdy wartość $A[j]$ jest wstawiana do tablicy $B$.
      Dzięki temu następny element o tej samej wartości (jeśli istnieje) zostanie wstawiony do tablicy $B$ na pozycję o numerze o jeden mniejszym.
\end{enumerate}

Gdy przyjrzymy się dokładnie algorytmowi sortowania przez zliczanie możemy zastanawiać się, dlaczego postępujemy w tak dziwny sposób?
Przecież mając zliczone wystąpienia każdej wartości w licznikach, możemy je od razu przepisać do zbioru wyjściowego.
Istotnie tak by było, gdyby chodziło jedynie o posortowanie liczb.
Jest jednak inaczej.
Celem nie jest posortowanie jedynie samych wartości elementów.
Sortowane wartości są zwykle tzw. kluczami, czyli wartościami skojarzonymi z elementami, które wyliczono na podstawie pewnego kryterium.
Sortując klucze chcemy posortować zawierające je elementy.
Dlatego do zbioru wynikowego musimy przepisać całe elementy ze zbioru wejściowego, gdyż w praktyce klucze stanowią jedynie część (raczej małą) danych zawartych w elementach.
Zatem algorytm sortowania przez zliczanie wyznacza docelowe pozycje elementów na podstawie reprezentujących je kluczy, które mogą się wielokrotnie powtarzać.
Następnie elementy są umieszczane na właściwym miejscu w zbiorze wyjściowym.

\textbf{Złożoność}

Będziemy określać złożoność czasową poprzez oszacowanie każdej kolejnej pętli.
Pierwsza działa w czasie $\Theta(k)$, druga w czasie $\Theta(n)$, kolejna w czasie $\Theta(k)$ a ostatnia w $\Theta(n)$.
Całkowity czas działania procedury to $\Theta(k+n)$ ale w praktyce najczęściej $k \in O(n)$, a wtedy złożoność czasowa to \textbf{$\Theta(n)$}.

\textbf{Stabilność}

Stabilność to własność sortowania, dzięki której elementy o tych samych wartościach występują w tablicy wynikowej w takiej samej kolejności jak w ciągu wejściowym.
Sortowanie przez zliczanie jest stabilne, co jest ważną zaletą pozwalającą wykorzystać tę procedurę w sortowaniu pozycyjnym.


\chapter{Under construction}

\section{Problemy NP} %albo NP-zupełność, nie jestem pewien co do nazwy
\sectionauthor{Wojciech Balik}

\label{sec:problemynp}

Na wstępie chcielibyśmy zaznaczyć że tematyka NP-zupełności nie będzie poruszana dogłębnie. 
Nie będziemy np. wprowadzać definicji maszyny Turinga(lub innego modelu) oraz przeprowadzać skomplikowanych dowodów, gdyż wymagałoby to wiedzy z zakresu teorii języków formalnych i złożoności obliczeniowej. 
Zamiast tego będziemy chcieli nabrać trochę intuicji co do tego czy w ogóle warto starać się rozwiązywać dany problem czy może jest to strata naszego czasu.
\\
\\ \noindent
\begin{definition}
Problemem decyzyjnym nazywamy problem którego rozwiązanie przyjmuje jedną z dwóch wartości - $TAK, NIE$
\end{definition}
\noindent
Zauważmy że problemy decyzyjne możemy utożsamiać z podzbiorami pewnego uniwersumi, w ten sposób że problem jest zbiorem wartości, dla których odpowiedź to $TAK$.
\\ \noindent
\b{PRZYKŁAD}. Problem polegający na rozstrzygnięciu czy dana liczba naturalna $p$ jest liczbą pierwszą utożsamilibyśmy ze zbiorem liczb pierwszych.

\begin{definition}
Dla danej funkcji kosztu $f$, problemem optymalizacyjnym nazywamy problem którego rozwiązaniem jest wartość z danego uniwersum, minimalizująca wartość funkcji kosztu.
\end{definition}
\noindent
\b{PRZYKŁAD}. Dla danego grafu $G$ oraz wierzchołków $u$ i $v$, wyznaczyć najkrótszą drogę między $u$ i $v$. 
Naszą funkcją kosztu jest długość drogi,a uniwersum to wszystkie drogi łączące $u$ i $v$.

% TODO napisać cos o związkach między problemami decyzyjnymi i optymalizacyjnymi

\begin{definition}[Klasa $NP$]
Klasą $NP$ nazywamy zbiór problemów decyzyjnych $L$ t. że istnieje algorytm wielomianowy $A$ dla którego prawdziwe jest następujące zdanie:
\begin{center}
$x \in L \iff $ istnieje $y$ t. że $|y| < |x|^c$ oraz $A$ akceptuje $(x,y)$
\end{center}
\end{definition}
\noindent
W powyższej definicji możemy myśleć o $y$ jako o podpowiedzi dla algorytmu, lub nawet gotowym rozwiązaniu, natomiast $A$ jest weryfikatorem, który używając podpowiedzi próbuje udzielić odpowiedzi, gdzie $A$ akceptuje $(x,y)$ oznacza że odpowiedź dla danych wejściowych $x$ to $TAK$.
% TODO dodać cos o długości, bądź też rozmiarze wejścia (|.|)
\\
\\ \noindent
\b{PRZYKŁAD}. Pokażmy że problem decyzyjny, polegający na sprawdzeniu czy w grafie $G$ istnieje cykl Hamiltona jest w $NP$.
\begin{proof}
Najpierw musimy wymyślić weryfikator, tzn. wielomianowy algorytm sprawdzający istnienie cyklu Hamiltona w grafie. 
Nasz będzie dość prosty - dla danego grafu $x$, oraz $y$ - pewnej drogi w $x$, zwyczajnie sprawdzimy czy $y$ jest cyklem hamiltona, a jeśli tak to zwrócimy $TAK$(to znaczy nasz algorytm będzie akceptować parę ($x$, $y$)). 
Łatwo zauważyć że czas działania algorymtu jest ograniczony przez $O(|V| + |E|)$.
\\
\\ \noindent
Dowód $\Rightarrow$ \\ \noindent
Niech $x$ będzie grafem zawierającym cykl Hamiltona. 
Wówczas naszym $y$ bedzie właśnie ten cykl. 
Wtedy $|y|$ jest ograniczona przez $O(|V| + |E|)$ oraz zgodnie z opisem działania naszego algorytmu, $A$ zaakceptuje $(x, y)$. 
\\ \noindent
Dowód w drugą stronę jest równie prosty, więc go pominiemy.
\end{proof}

\noindent
Zauważmy że podpowiedź nie zawsze jest potrzebna. 
Np. w przypadku problemu polegającym na sprawdzeniu czy liczba jest podzielna przez $2$, możemy wziąć jakąkolwiek podowiedź, zignorować ją a następnie udzielić odpowiedzi w czasie wielomianowym. O takich problemach mówimy że są klasy $P$. 
\\
\\ \noindent

\begin{definition}[Redukcje wielomianowe]
Mówimy że problem $L$ jest wielomianowo redukowalny do problemu $Q$ jeśli:
\begin{enumerate}
\item $\exists f \forall x \quad  x \in L \Rightarrow f(x) \in Q$
\item Istnieje wielomianowy algorytm obliczający funkcję $f$
\end{enumerate}
\end{definition}
\noindent
Mimo że na pierwszy rzut oka może się to wydać niezrozumiałe, sens intuicyjny jest bardzo prosty. 
Chcielibyśmy "przetłumaczyć" problem $L$ na problem $Q$, to znaczy użyć rozwiązania problemu $Q$ tak abyśmy mogli za jego pomocą rozwiązać problem $L$. 
Jeśli uda nam się znaleźć taką funkcję(o której myślimy jak o algorytmie) to znaczy że znaleźliśmy redukcję problemu $L$ do problemu $Q$. 
Aby redukcja była wielomianowa, musi być spełniony drugi warunek, tzn. musimy umieć obliczyć tę funkcję w czasie wielomianowym.
% TODO pokazać jakiś przykład redukcji

\begin{definition}[Problem $NP$-zupełny]
Problem $L$ jest problemem $NP$-zupełnym jeśli:
\begin{enumerate}
\item $L \in NP$
\item Każdy problem z klasy $NP$ jest wielomianowo redukowalny do $L$
\end{enumerate}
\end{definition}
\noindent
O ile udowodnienie pierwszego warunku nie wydaje się specjalnie trudne(już raz to zrobiliśmy), tak drugi warunek może już sprawiać kłopoty. 
Zazwyczaj jednak nie dowodzi się tego bezpośrednio. 
Zamiast tego korzysta się z następującego faktu:
\begin{lemma}
Jeśli $L$, jest problemem $NP$, $Q$ jest problemem $NP$-zupełnym oraz $L$ jest redukowalny wielomianowo do $Q$ to $L$ jest problemem $NP$-zupełnym
\end{lemma}
\begin{proof}
Weźmy dowolny problem $A \in NP$, zredukujmy go do $Q$ a następnie do $L$. Czas obliczeń jest wówczas ograniczony przez sumę wielomianów, a więc przez wielomian.
\end{proof}
\noindent
Jednak aby skorzystać z tego lematu trzeba najpierw znaleźć problem który jest $NP$-zupełny. 
Jednym z pierwszych takich problemów którego $NP$-zupełność udowodniono był problem $SAT$(spełnialność formuł logicznych), a jak się później okazało, 
wiele innych problemów możana do niego zredukować. 
My zostawimy to bez dowodu. 
\\ 
\\ \noindent
To czy $P = NP$ jest wciąż otwartym, problemem milenijnym, którego rozwiązanie jest warte 1 000 000\$. 
Mimo że nikomu jeszcze nie udało się tego udowodnić, cały(duża większość) świat informatyki/matematyki wierzy że $P \neq NP$. 
Wiara jest na tyle mocna że wydawane są prace naukowe oraz dowodzone są twierdzenia przy założeniu że $P \neq NP$.
\\ 
\\ \noindent
O problemach $NP$-zupełnych można myśleć jak o takich problemach które są co najmniej tak samo trudne jak wszystkie inne problemy z klasy $NP$. 
Ogólnie rzecz biorąc, są to problemy bardzo trudne obliczeniowo, dla których nie istnieje żaden algorytm wielomianowy rozwiązujący je, co więcej prawdopodobnie nigdy nie będzie istniał, no chyba że $P = NP$.
Morał z tego jest taki, że jeśli wiemy że dany problem jest $NP$-zupełny, to raczej nie warto tracić czasu na rozwiązywanie go.
\\ 
\\ \noindent
Jako dodatek, poniżej prezentujemy liste najbardziej znanych problemów $NP$-zupełnych(lub $NP$-trudnych):
\begin{enumerate}
\item Problem SAT
\item Problem cyklu Hamiltona
\item Problem trójkolorowalności grafu
\item Problem komiwojażera($NP$-trudny)
\item Problem kliki
\item Problem plecakowy($NP$-trudny)
\item Problem sumy podzbioru
\item Problem minimalnego pokrycia zbioru($NP$-trudny)
\end{enumerate}

 
% TODO dodać coś o problemach NP-trudnych, może dodać jakieś obrazki
\noindent


\section{Kopce binarne}
\sectionauthor{Mateusz Ciesiółka}

\label{sec:kopiec}

\comment{Chciałbym, aby skrypt był skryptem i żebyśmy tam pisali formalnie.
Dlatego definicję kopca chciałbym mieć zapisaną formalnie (bez przypisów) i w znacznikach \\begin\{definition\} \\end\{definition\}.
Teraz śmieszna rzecz jest taka, że kopiec trudno ładnie formalnie zdefiniować na drzewach.
W sensie najpierw musielibyśmy powiedzieć co to jest poziom w drzewie, co to jest pełny poziom w drzewie a następnie .... w sumie to nawet ja nie wiem jak to ładnie zdefiniować na drzewach :P
Więc zamiast mówić, że kopiec to drzewo które można reprezentować w tablicy, lepiej zdefiniować kopiec jako tablicę na którą możemy patrzeć jako na drzewo.
Wtedy musimy zdefiniować co to jest lewy syn elementu w tablicy, prawy syn oraz ojciec.
Na tej podstawie będzie łatwiej nam zdefiniować własność kopca jako, że dla każdego wierzchołka wartość elementu jest mniejsza od wartości elementów jego dzieci.
Jeśli wolimy zamiast tego powiedzieć że ciąg elementów na ścieżce od liścia do korzenia tworzy ciąg malejący to musimy zdefiniować co to jest liść, korzeń i ścieżka.
Co da się zrobić ale nie wiem czy jest to warte świeczki, gdyż to będzie bodajże jedyne miejsce w których użyjemy tych definicji).}

Kopiec binarny to struktura danych, która reprezentowana jest jako prawie pełne drzewo binarne\footnote{To znaczy wypełniony na wszystkich poziomach (poza, być może, ostatnim).} i na której zachowana jest własność kopca.
Kopiec przechowuje klucze, które tworzą ciąg uporządkowany.
W przypadku kopca typu \emph{min} ścieżka prowadząca od dowolnego liścia do korzenia tworzy ciąg malejący.

Kopce można w prosty sposób reprezentować w tablicy jednowymiarowej \textendash kolejne poziomy drzewa zapisywane są po sobie.

\tizkboxwithcaption{tikz/kopiec-warstwy.tikz}{Reprezentacja kolejnych warstw kopca w tablicy jednowymiarowej.}

Warto zauważyć, że tak reprezentowane drzewo pozwala na łatwy dostęp do powiązanych węzłów.
Synami węzła o indeksie $i$ są węzły $2i+1$ oraz $2i + 2$, natomiast jego ojcem jest $\left\lfloor \frac{i-1}{2} \right\rfloor$.

Kopiec powinien udostępniać trzy podstawowe funkcje: \texttt{zamien\_element}, która podmienia wartość w konkretnym węźle kopca, \texttt{przesun\_w\_gore} oraz \texttt{przesun\_w\_dol}, które zamieniają odpowiednie elementy pilnując przy tym, aby własność kopca została zachowana.

\tizkboxwithcaption{tikz/kopiec-przenoszenie.tikz}{Przykład działania funkcji \texttt{zamien\_element}. a) Oryginalny kopiec. b) Zmiana wartości w wyróżnionym węźle. c) Ponieważ nowa wartość jest większa od wartości swoich dzieci, należy wykonać wywołanie funkcji \texttt{przesn\_w\_dol}. d) Po zmianie własność kopca nie jest zachowana, dlatego należy ponownie wywołać funkcję \texttt{przesn\_w\_dol}. To przywraca kopcowi jego własność.}

\comment{Zamień element jest fajną funkcją, ale funkcje przesun w dol i przesun w gore są ważniejsze.
Ponadto nie chcemy nazywać funkcje w języku polskim.}
\begin{algorithm}[ht]
  \If{k[i] < v}
  {
    k[i] = v\;
	przesun\_w\_dol(k, i)\;
  }
  \Else
  {
    k[i] = v\;
	przesun\_w\_gore(k, i)\;
  }  
  \caption{Implementacja funkcji \texttt{zamien\_element}}
  \label{kopiec-zamien-element}
\end{algorithm}

\section{Algorytm rosyjskich wieśniaków}
\sectionauthor{Przemysław Joniak}


Algorytm rosyjskich wieśniaków jest przypisywany sposobowi mnożenia liczb używanemu w XIX-wiecznej Rosji.
Aktualnie jest on stosowany w niektórych układach mnożących.

W celu obliczenia $a \cdot b$ tworzymy tabelkę i liczby $a$ i $b$ zapisujemy w pierwszym jej wierszu.
Kolumnę $a$ wypełniamy następująco: w $i+1$ wierszu wpisujemy wartość z wiersza $i$ podzieloną całkowicie przez 2.
W kolumnie $b$ kolejne wiersze tworzą ciąg geometryczny o ilorazie równym 2.
Wypełnianie tabelki kończymy wtedy, gdy w kolumnie $a$ otrzymamy wartość 1.
Na koniec sumujemy wartości w kolumine $b$ z tych wierszy dla których wartości w kolumine $a$ są nieparzyste.
Uzyskany wynik to $a \cdot b$.

W poniższym przykładzie obliczymy $42 \cdot 17$.

\begin{center}
\begin{tabular}{ c|c } 

 a & b \\ 
 \hline
 42 & 17 \\ 
 \textbf{21} & $17 \cdot 2 = 34$ \\
 10 & $17 \cdot 2^2 = 68$ \\
 \textbf{5}  & $17 \cdot 2^3 = 136$ \\
 2  & $17 \cdot 2^4 = 272$ \\
 \textbf{1}  & $17 \cdot 2^5 = 544$ \\

\end{tabular}
\end{center}
\comment{Jak usuniecie enter po tabelce, to to zdanie nie dostanie w pdfie tabulatora.
Tabulator powinien się znajdować przed akapitami.
To zdanie jest częścią poprzedniego akapitu.}
Wartości $a$ są nieparzyste w wierszach 2, 4 oraz 6. Zatem będziemy sumować wartości $b$ z wierszy 2, 4 i 6.

\begin{equation*} 
\begin{split}
a \cdot b &= 17\cdot2 + 17\cdot2^3 + 17\cdot2^5 \\
&= 34 + 136 + 544 \\
&= 714
\end{split}
\end{equation*}
Faktycznie, otrzymaliśmy wynik poprawny. Spójrzmy raz jeszcze na na tę sumę:

\begin{equation*} 
\begin{split}
a \cdot b &= 17\cdot2 + 17\cdot2^3 + 17\cdot2^5 \\
&= 17 \cdot ( 2^5 + 2^3 + 2) \\
&= 17 \cdot ( 1 \cdot 2^5 + 0 \cdot 2^4 + 1 \cdot 2^3 + 0 \cdot 2^2 + 1 \cdot 2^1 + 0 \cdot 2^0 ) \\
&=17_{10} \cdot 101010_2 \\
&=17 \cdot 42 = 714
\end{split}
\end{equation*}
Przypomnij sobie algorytm zamiany liczby w systemie dziesiętnym na system binarny.
Okazuje się, że algorytm rosyjskich wieśniaków "po cichu" wylicza tę reprezentację $a$.

W kolejnych paragrafach podamy algorytm i w celu jego udowodnienia sformułujmy \textit{niezmiennik} oraz wykażemy jego prawdziwośc.

\begin{algorithm}[h]
  \DontPrintSemicolon
  \SetAlgorithmName{Algorytm}{}
  
  \KwData{ $a$, $b$ - liczby naturalne }
  
  \KwResult{ $wynik = a \cdot b$ }
  
  $a' \leftarrow a$\;
  $b' \leftarrow b$\;
  $wynik \leftarrow 0$\;
  \While{\upshape $a' > 0$}
  {
    \If{\upshape $a' \textsf{ mod } 2 = 1$}
    {
      $wynik \leftarrow wynik + b'$\;
    }
    $a' \leftarrow a' \textsf{ div } 2$\;
    $b' \leftarrow b' \cdot 2$\;
  }
  
  \caption{Algorytm rosyjskich wieśniaków}
  \label{alg-wiesniakow}
\end{algorithm}

\begin{theorem}
Niech $a'_i$ (kolejno: $b'_i$, $wynik_i$) będzie wartością zmiennej \texttt{a'} (\texttt{b'}, \texttt{wynik}) w $i-tej$ iteracji pętli \texttt{while}. Zachodzi następujący niezmiennik pętli:
\[
a'_i \cdot b'_i + wynik_i = a \cdot b \enspace.
\]
\end{theorem}

\begin{lemma}
Przed wejściem do pętli \texttt{while} niezmiennik jest prawdziwy.
\end{lemma}
\begin{proof}
Skoro przed przed wejściem do pętli mamy: $a'_0 = a$, $b'_0 = b$ oraz $wynik_0 = 0$, to oczywiście: $a'_0 \cdot b'_0 + wynik_0 = a \cdot b + 0 = a \cdot b$.
\end{proof}

\begin{lemma}
Po $i-tym$ obrocie pętli niezmiennik jest spełniony.
\end{lemma}
\begin{proof}
Załóżmy, że niezmiennik zachodzi w $i-tej$ iteracji i sprawdźmy co dzieje się w $i+1$ iteracji.
\comment{Nie możesz czegoś takiego założyć :)}
Rozważmy dwa przypadki.


\begin{itemize}
    \item $a'_i$ parzyste. Instrukcja \texttt{if} się nie wykona, w $i+1$ iteracji $wynik_i$ pozostanie niezmieniony, $a'_i$ zmniejszy się o połowę, a $b'_i$ zwiększy dwukrotnie. 
    \[
      wynik_{i+1} = wynik_i
    \]
    \[
      a'_{i+1} = a'_i \textsf{ div } 2 = \frac{a'_i}{2}
    \]
    \[
      b'_{i+1} = b'_i \cdot 2
    \]
    W tym przypadku otrzymujemy:
    \[
      a'_{i+1} \cdot b'_{i+1} + wynik_{i+1} = \frac{a'_i}{2} \cdot 2 b'_i + wynik_i = a'_i \cdot b'_i + wynik_i = a \cdot b
    \]

    \item $a'_i$ nieparzyste:
    \[
      wynik_{i+1} = wynik_i + b'_i
    \]
    \[
      a'_{i+1} = a'_i \textsf{ div } 2 = \frac{a'_i-1}{2}
    \]
    \[
      b'_{i+1} = b'_i \cdot 2
    \]
    Ostatecznie otrzymujemy:
    \[
      a'_{i+1} \cdot b'_{i+1} + wynik_{i+1} = \frac{a'_i-1}{2} \cdot 2 b'_i + wynik_i +b'_i = a'_i \cdot wynik_i + b'_i= a \cdot b
    \]

\end{itemize}

\end{proof}

\begin{lemma}
Po zakończeniu algorytmu $wynik = a \cdot b$

\end{lemma}
\begin{proof}
Wystarczy zauważyć, że tuż po wyjściu z pętli \texttt{while} wartość zmiennej $a'$ wynosi $0$.
Podstawiając do niezmiennika okazuje się, że faktycznie algorytm rosyjskich wieśniaków liczy $a \cdot b$.
\end{proof}

\begin{lemma}
Algorytm sie kończy.
\end{lemma}
\begin{proof}
Skoro $a_i \in \mathbb{N} $ oraz $\mathbb{N}$ jest dobrze uporządkowany, to połowiąc $a_i$ po pewnej liczbie iteracji otrzymamy 0.
\end{proof}

Z powyższych lematów wynika, że niemiennik spełniony jest zarówno przed, w trakcie jak i po zakończeniu algorytmu. 
Algorytm rosyjskich wieśniaków jest poprawny.

\paragraph{Złożoność}

Z każdą iteracją połowimy $a'$. 
Biorąc pod uwagę kryterium jednorodne pozostałe instrukcje w pętli nic nie kosztują. 
Stąd złożoność to $O(\log a)$.

W kryterium logarytmicznym musimy uwzględnić czas dominującej instrukcji: dodawania  $wynik \leftarrow wynik + b'$. 
W najgorszym przypadku zajmuje ono $O(\log ab)$. Zatem złożoność to $O(\log a \cdot \log ab)$.


\section{Sortowanie topologiczne}
\sectionauthor{Mateusz Ciesiółka, Krzysztof Piecuch}

\label{sec:sortowanietopologiczne}

\tizkboxwithcaption{tikz/topologic-sort.tikz}
{
Przykładowy graf z ubraniami dla bramkarza hokejowego.
Krawędź między wierzchołkami $a$ oraz $b$ istnieje wtedy i tylko wtedy, gdy gracz musi ubrać $a$ zanim ubierze $b$.
Pytanie o to w jakiej kolejności bramkarz powinien się ubierać, jest pytaniem o posortowanie topologiczne tego grafu.
}


\section{Algorytmy sortowania}
\sectionauthor{Anna Karaś}

\label{sec:merge-sort}

W tym rozdziale zapoznamy się z algorytmem sortowania przez scalanie (ang. \textit{merge sort}). 
Wykorzystuje on metodę "dziel i zwyciężaj" - problem jest dzielony na kilka mniejszych podproblemów podobnych do początkowego problemu, problemy te są rozwiązywane rekurencyjnie, a następnie rozwiązania otrzymane dla podproblemów scala się, uzyskując rozwiązanie całego zadania.

\textbf{Idea.} Algorytm sortujący dzieli porządkowany $n$-elementowy zbiór na kolejne połowy, aż do uzyskania $n$ jednoelementowych zbiorów - każdy taki zbiór jest już posortowany. 
Uzyskane w ten sposób części zbioru sortuje rekurencyjnie - posortowane części łączy ze sobą za pomocą scalania tak, aby wynikowy zbiór był posortowany.

\textbf{Scalanie.} Podstawową operacją algorytmu jest scalanie dwóch uporządkowanych zbiorów w jeden uporządkowany zbiór.
W celu wykonania scalania skorzystamy z pomocniczej procedury  \texttt{merge}$(A,p,q,r)$, gdzie $A$ jest tablicą, a $p, q, r$ są indeksami takimi, że $p \leq q < r$. 
W procedurze zakłada się, że tablice $A[p..q]$ oraz $A[q+1..r]$ (dwie przyległe połówki zbioru, który został przez ten algorytm podzielony) są posortowane. 
Procedura \texttt{merge} scala te tablice w jedną posortowaną tablicę $A[p..r]$. 
Ogólna zasada działania jest następująca:
\begin{enumerate}
\item Przygotuj pusty zbiór tymczasowy.
\item Dopóki żaden ze scalanych zbiorów nie wyczerpał elementów, porównuj ze sobą pierwsze elementy każdego z nich i w zbiorze tymczasowym umieszczaj mniejszy z elementów.
\item W zbiorze tymczasowym umieść zawartość tego scalanego zbioru, który zawiera niewykorzystane jeszcze elementy.
\item Zawartość zbioru tymczasowego przepisz do zbioru wynikowego i zakończ algorytm.
\end{enumerate}
Zapis algorytmu scalania dwóch list w pseudokodzie podano niżej.
\begin{algorithm}[h]

  \DontPrintSemicolon
  
  \SetAlgorithmName{Algorytm}{}
  
  \KwData{ tablica $A$, liczby $p$, $q$, $r$}

  \KwResult{ posortowana tablica $A[p..r]$ }
  
  $C \leftarrow$ pusta tablica\;
  
  $i \leftarrow p$, $j \leftarrow q+1$, $k \leftarrow 0$\;
  
  \While{$i \leq q$ oraz $j \leq r$} {
  	\eIf{$A[i] \leq A[j]$} {
    	$C[k] \leftarrow A[i]$, $i \leftarrow i+1$\;
    } { 
    	$C[k] \leftarrow A[j]$, $j \leftarrow j+1$\;
    
    }
    $k \leftarrow k+1$\;
  
  }

  \While{$i \leq q$} {
    $C[k] \leftarrow A[i]$, $i \leftarrow i+1$, $k \leftarrow k+1$\;
  }
  
  \While{$j \leq r$} {
    $C[k] \leftarrow A[j]$, $j \leftarrow j+1$, $k \leftarrow k+1$\;
  }
  
  \caption{Procedura \texttt{merge}}
  \label{alg-merge}
\end{algorithm}

Scalanie wymaga $O(n+m)$ operacji porównań elementów i wstawienia ich do tablicy wynikowej. 

\textbf{Sortowanie.} Algorytm sortowania przez scalanie jest algorytmem rekurencyjnym. 
Wywołuje się go z zadanymi wartościami indeksów wskazujących na początek i koniec sortowanego zbioru, zatem początkowo indeksy obejmują cały zbiór. 
Algorytm wyznacza indeks elementu połowiącego przedział, a następnie sprawdza, czy połówki zbioru zawierają więcej niż jeden element. 
Jeśli tak, to rekurencyjnie sortuje je tym samym algorytmem.
Po posortowaniu obu połówek zbioru scalamy je za pomocą opisanej wcześniej procedury scalania podzbiorów uporządkowanych i kończymy algorytm. 
Zbiór jest posortowany.
\begin{algorithm}[h]

  \DontPrintSemicolon
  
  \SetAlgorithmName{Algorytm}{}
  
  \KwData{ tablica $A$, liczby $p$, $r$}

  \KwResult{ posortowana tablica $A[p..r]$ }
  
  $q \leftarrow 0$\;
  
  \If{$p < r$} {
  	$q \leftarrow \left \lfloor{\frac{p+r}{2}}\right \rfloor $\;
    \texttt{merge sort}$(A,p,q)$\;
 	\texttt{merge sort}$(A,q+1,r)$\;
    \texttt{merge}$(A,p,q,r)$\;
  }
  \caption{Procedura \texttt{merge sort}}
  \label{alg-merge-sort}
\end{algorithm}

\textbf{Złożoność.} Chociaż algorytm sortowania przez scalanie działa poprawnie nawet wówczas, gdy $n$ jest nieparzyste, dla uproszczenia analizy załóżmy, że $n$ jest potęgą dwójki. 
Dzielimy wtedy problem na podproblemy rozmiaru dokładnie $\frac{n}{2}$. 
Rekurencję określającą czas $T(n)$ sortowania przez scalanie otrzymujemy, jak następuje. \\\\
Sortowanie przez scalanie jednego elementu wykonuje się w czasie stałym. Jeśli $n > 1$, to czas działania zależy od trzech etapów:\\
\textbf{Dziel:} podczas tego etapu znajdujemy środek przedziału, co zajmuje czas stały, zatem $D(n) = \theta(1)$.\\
\textbf{Zwyciężaj:} rozwiązujemy rekurencyjnie dwa podproblemy, każdy rozmiaru $\frac{n}{2}$, co daje czas działania $2T(\frac{n}{2})$.\\
\textbf{Połącz:} procedura \texttt{merge}, jak wspomniano wyżej, działa w czasie liniowym, a więc $P(n) = \theta(n)$.\\\\
Funkcje $D(n)$ i $P(n)$ dają po zsumowaniu funkcję rzędu $\theta(n)$. 
Dodając do tego $2T(\frac{n}{2})$ z etapu "zwyciężaj", otrzymujemy następującą rekurencję dla $T(n)$:
$$
 T(n) = 
  \begin{cases} 
   \theta(1) & n = 1 \\
   2T(\frac{n}{2}) + \theta(n) & n > 1
  \end{cases}
$$

Poniższy przykład ilustruje zasadę działania sortowania przez scalanie:\\
{\tiny<tu obrazek, ale nie umiem w obrazki, na dniach ogarnę>}

\textbf{Podsumowanie.} Sortowanie przez scalanie należy do algorytmów szybkich, posiada klasę złożoności równą $\theta(n\log n)$. Jest oparty na metodzie dziel i zwyciężaj, która powoduje podział dużego problemu na mniejsze, łatwo rozwiązywane podproblemy. Sortowanie nie odbywa się w miejscu, potrzebujemy dodatkowej struktury. Algorytm jest stabilny.

\subsection{Quick sort}

In progress


\section{Minimalne drzewa rozpinające}

Todo, todo, todo...

\subsection{Cut Property i Circle Property}

Udowodnijmy dwie własności, które okazują się być niezwykle przydatne w dowodach dotyczących minimalnych drzew rozpinających – MST.

\subsection{Cycle property}
    Niech $C$ będzie dowolnym cyklem w~ważonym grafie $G$. Załóżmy, że wszystkie wagi są różne.
\begin{theorem}
   Jeżeli krawędź $e_k \in C$ jest najcięższą spośród krawędzi z~$C$, to $e \notin \text{MST}\left( G \right)$.
\end{theorem}
\begin{proof}
    Załóżmy nie wprost, że utworzyliśmy drzewo $\text{MST}\left( G \right)$ w~którym znajduje się krawędź $e$. Usuńmy ją. W~ten sposób otrzymaliśmy dwa rozłączne drzewa, nazwijmy je $T_1$ i $T_2$.
    
    Krawędź $e$ należała do cyklu $C$. Stąd wynika, że istniała druga krawędź $f$, która tworzyła „most” między $T_1$ a $T_2$. Ponadto, z~założenia, ma ona mniejszą wagę od $e$. 
    
    Dodajmy krawędź $f$ do naszego lasu $ \left\{ T_{1},T_{2}\right\} $. Otrzymaliśmy spójne drzewo MST o~koszcie mniejszym od pierwotnego drzewa MST. Sprzeczność.
\end{proof}

\subsection{Cut property}
    Założenia takie same, jak w przypadku Cycle property: $C$ – dowolny cykl w~ważonym grafie $G$ o różnych wagach.
\begin{theorem}
   Podzielmy wszystki wierzchołki cyklu na dwa rozłączne zbiory $C_1$ i $C_2$ (czyli dokonajmy cięcia). Jeżeli $e$ jest najlżejszą krawędzią spośród łączących te dwa zbiory, to znajdzie się ona w~$\text{MST}\left( G \right) $.
\end{theorem}
\begin{proof}
    Załóżmy nie wprost, że mamy drzewo $T = \text{MST}\left( G \right) $, które nie zawiera $ e $. Dodanie tej krawędzi utworzy cykl. Zatem istnieje druga krawędź $ f $, która znajduje się między podzbiorami $ C_{1} $ oraz $ C_{2} $.
    
    Rozważmy drzewo $ T \setminus \left\{ f \right\} \cup \left\{ e \right\} $. Tym sposobem otrzymaliśmy drzewo MST o~mniejszej wadze. Sprzeczność.
\end{proof}

\subsection{Algorytm Prima}

\subsection{Algorytm Kruskala}

\subsection{Algorytm Borůvki}

\section{Algorytm Dijkstry}
\sectionauthor{Mikołaj Słupiński}

\label{sec:dijkstra}

\paragraph{}Powstało wiele algorytmów pozwalających wyznaczyć najkrótszą ścieżkę w grafie z
krawedziami ważonymi. Wśród nich na szczególną uwagę zasługuje algorytm Dijkstry.

\subsection{Działanie}

\paragraph{}Niech $G = (V, E)$ będzie grafem ważonym. Dodatkowo musimy założyć, że waga $w(u,v) \ge 0$
dla wszystkich krawędzi $(u, v) \in E$.

Niech $S$ będzie takim zbiorem wierzchołków, których najkrótsza odległość od źródła
s została już określona. Algorytm Dijkstry wybiera kolejne wierzchołki $u \in V - S$
z minimalnym oszacowaniem najkrótszej ścieżki, dodaje u do S, i rozluźnia wszystkie ścieżki pozostawiając u.

\paragraph{Pseudokod:}
TODO

Algorytm zachowuje niezmiennik, że $Q = V - S$ na początku każdej iteracji pętli while.

Algorytm Dijkstry stosuje zachłanne podejście zawsze wybierając najbliższy wierzchołek
w $V - S$, który dodaje do zbioru S.

\section{Dowód poprawności algorytmu}

\paragraph{}Aby dowieść poprawności algorytmu Dijkstry skorzysamy z następującego niezmiennika pętli:
Na począrku każdej iteracji pętli while $d[v] = \delta(s,v)$ dla każdego wierzchołka $v \in S$.

Wystarczy udowodnić, że dla każdego wierzchołka $u \in V$ mamy $d[u] = \delta(s, u)$ w momencie
kiedy $u$ jest dodane do zbioru $S$. Kiedy już udowodnimy, że $d[u] = \delta(s, u)$,
polegamy na ograniczeniu górnym własności, aby pokazać, że równość jest potem zachowana.

\paragraph{Inicializacja:} Na samym początku, $S$ jest zbiorem pustym, więc niezmiennik jest oczywiście
zachowany.

\paragraph{Utrzymanie:} Chcemy pokazać, że z każdą iteracją zachowana jest równość $d[u] = \delta(s, u)$
dla wszystkich wierzchołków dodanych do zbioru $S$. Załóżmy nie wprost, że u jest
pierwszym takim wierzchołkiem, że $d[u] \ne \delta(s,u)$ w momencie go do zbioru
$S$. Weźmy dowolny $u \ne s$, ponieważ  $s$ jest pierwszym wierzcholkiem dodanym
do $S$ i E$d[s] = \delta(s,s) = 0$. Skoro $u \ne s$, to $S$ nie jest zbiorem pustym
zaraz przed dodaniem $u$ do S. Musi istnieć ścieżka od $s$ do $u$ gdyż inaczej
$d[u] = \delta(s,u) = \inf$ i doszlibyśmy do sprzeczności z naszym założeniem, że
$d[u] \ne \delta(s, u)$. Skoro istnieje conajmniej jedna ścieżka, istnieje też ścieżka
najkrótsza. Przed dodaniem $u$ do $S$, ścieżka p łączy wierzchołek w $S$, powiedzmy s,
z wierzchołkeim w $V - S$, powiedzmy $u$. Rozważmy pierwszy wierzchołek y należący do p
t. że $y \in V - S$ i niech $x \in S$ będzie poprzednikiem y. Możemy podzielić ścieżkę
$p$ na dwie podścieżki, $p_1$ łączącą $s$ z $x$ oraz $p_2$ łączącą $y$ z $u$
(ścieżki te mogą być pozbawione krawędzi).

Chcemy udowodnić, że $d[y] = \delta(s,y)$ gdy $u$ jest dodane do $S$. Aby to zrobić
zauważmy, że $x \in S$. Skoro $u$ jest pierwszym wierzchołkiem, dla którego $d[u] \ne \delta(s,u)$,
to $d[x] \delta(s,x)$ w momencie kiedy $x$ został dodany do S. Krawędź $(x, y)$
została wtedy zrelaksowana, z czego wynika powyższa równość.

Możemy teraz uzyskać sprzeczność pozwalającą nam udowodnić, że $d[u] = \delta(s,u)$.
Skoro $y$ występuje przed $u$ na najkrótszej ścieżce od $s$ do $u$, a wszystkie
wagi krawędzi są nieujemne (w szczególności krawędzi należących do $p_2$). Otrzymujemy
$\delta(s, y) \le \delta(s, u)$, więc
TODO

Ale skoro oba wierzchołki u i y były w $V - S$ gdy ustaliliśmy $u$ mamy $d[u] \le d[y]$.
Zatem, obie nierówności są tak na prawdę równościami, dzięki którym $d[y] = \delta(s, y) = \delta(s, u) = d[u]$.
W rezultacie $d[u] = \delta(s,u)$, co przeczy naszemu wyborowi u. Wnioskujemy, że
$d[u] = \delta(s, u)$ gdy dodamy $u$ do $S$, a własność ta jest zachowana od tego momentu.

\paragraph{Zakończenie:} Na końcu, $Q$ jest puste, co w połączeniu z naszym niezmiennikiem, że $Q = V - S$ implikuje,
że $S = V$, zatem $d[u] = \delta(s,u)$ dla każego wierzchołka $u \in V$.

\section{Analiza}

Skoro wiemy już jak działa algorytm Dijkstry oraz wiemy, że działa poprawie należy
zastanowić się z jaką prędkością on działa.
TODO
\section{Problemy}
TODO


\section{Sieci przełączników Benesa-Waksmana}
\sectionauthor{Marcin Bartkowiak}

\label{sec:siecibenesa}

W tym rozdziale zajmiemy się sieciami przełączników Benesa-Waksmana. Moją one zastosowanie w sieciach komputerowych.

\subsection{Budowa}
Sieć składa się z przełączników.
Każdy z przełączników ma dwa możliwe stany.
\begin{itemize}
  \item W stanie 1 przełącznik przesyła dane z wejścia $i$ na wyjście $i$ ($i \in \left \{ 0, 1 \right \}$)
  \item W stanie 2 przełącznik przesyła dane z wejścia $i$ na wyjście $i + 1~mod~2$ ($i \in \left \{ 0, 1 \right \}$)
\end{itemize}
{\tiny<Tutaj wstawić obrazki ze stanami i jakąś przekładową sieć(wydaje mi się, że to lepiej objaśni, niż mój najlepszy opis)>}

\subsection{Konstrukcja sieci tworzącej wszystkie możliwe permutacje zbioru}
Dla ułatwienia będziemy się zajmować zbiorami w postaci $2^n$ ($n \in \mathbb{N}$).

Konstrukcja będzie oparta na zasadzie dziel i zwyciężaj i będzie sprowadzała problem do rekurencyjnego zbudowania sieci wielkości $2^{n-1}$,
a następnie odpowiedniego połączenia portów.
\subsubsection{$n = 1$}
Dla zbioru wielkości 2 jeden przełącznik generuje każdą możliwą permutacje; stan 1 generuje identyczność; stan 2 drugą permutację.
\subsubsection{$n > 1$}
{\tiny<Tutaj znowu wstawiłbym rysunek idei algorytmu>}
\subsection{Własności wygenerowanej sieci}
Głębokość sieci wyraża się równaniem
$$
 G(2^n) = 
  \begin{cases} 
   1 & n = 1 \\
   G(2^{n - 1}) + 2 & n > 1
  \end{cases}
$$
z tego wynika, że $G(n) = 2 \log n - 1$


Ilość przełączników w sieci wyraża się równaniem
$$
 P(2^n) = 
  \begin{cases} 
   1 & n = 1 \\
   2P(2^{n - 1}) + 2^n & n > 1
  \end{cases}
$$
Wykorzystując punkt 2. (a) z twierdzenia o rekurencji uniwersalnej możemy stwierdzić, że $P(n) = \Theta(n \log n)$.

\subsection{Dowód poprawności konstrukcji}
TODO
\subsection{Sortowanie}
TODO


\section{Pokrycie zbioru}
\sectionauthor{Michał Wierzbicki}

\label{sec:pokrycie}

Problem pokrycia zbioru jest problemem optymalizacyjnym związany z problemem alokacji zasobów. 
Przedstawimy zachłanny algorytm o logarytmicznym współczynniku aproksymacji rozwiązujący ten problem.\\

Dane dla problemu pokrycia zbioru to para $(U,\mathcal{S})$ oraz funkcja kosztu $c$. $U$, zwane uniwersum, jest skończonym zbiorem elementów, a $\mathcal{S}$ jest rodziną podzbiorów $U$, taką że:\\
\[ \bigcup\limits_{i=1}^{n} S_{i} = U \] \\
Mówimy, że podzbiór $S_{i} \in \mathcal{S}$ pokrywa elementy należące do $S_{i}$. 
Z kolei $c: S_{i} \rightarrow \mathbb{R} $, każdemu podzbiorowi $S_{i}$ określa cenę pokrycia swoich elementów.

Problem polega na znalezieniu podrodziny $\mathcal{T} \subseteq \mathcal{S}$, której elementy pokrywają cały zbiór $U$:
\[ U = \bigcup_{T \in \mathcal{T}} T \]  
Spośród wszystkich takich rozwiązań interesuje nas to, którego koszt $c(\mathcal{T})$ jest minimalny:
\[ c(\mathcal{T}) = \sum\nolimits_{T \in \mathcal{T}} c(T) \]  \\

Mając zdefiniowaną cenę podzbior możemy zdefiniować cenę rynkową elementu, która będzie naszym kryterium wyboru podzbiorów.
Niech $e_{1}, e_{2}, ... , e_{n}$ będą elementami $U$ w porządku pokrycia przez algorytm.
Przez cenę rynkową $f_{i}$ elementu będziemy rozumieć średni koszt nowo pokrywanego elementu $e_{i}$ przez rozpartywany podzbiór $S_{i}$: \\
\begin{align*}
f_{i} &= \frac{c(S_{j_i})}{\left|S_{j_i} \setminus \bigcup_{j_{i} < k} S_{k} \right|},\\
\end{align*}
gdzie $S_{i}$ jest to $i-ty$ zbiór wybierany przez algorytm, $S_{j_i}$ jest pierwszym zbiore m pokrywającym $e_{i}$, czyli\\ $j_{i} = min \bigg\{ 1 \leq k < n : e_i \in S_k \setminus \bigcup\limits_{l=1}^{k-1} S_{l}  \bigg\}$.\\
Mówiąc cena rynkowa zbioru mamy na myśli cenę rynkową elementów tego zbioru.\\


\begin{algorithm}[H]
  \DontPrintSemicolon
  \SetAlgorithmName{Algorytm}{}\\
  \KwData{ $U$ - uniwersum, $\mathcal{S}$ - rodzina podzbiorów $U$ }\\
  \KwResult{ $\mathcal{T}$, takie że $c(\mathcal{T})$ jest minimalne }\\
$\mathcal{T} \leftarrow \emptyset$  \\
  \While{$\mathcal{T} \neq U$}
  {
     Oblicz cenę rynkową dla wszystkich zbiorów\\
     Wybierz zbiór $A$ o najniższej cenie rynkowej\\
     $\mathcal{T} \leftarrow \mathcal{T} \cup A$ 
  }
  \caption{Zachłanny algorytm dla problemu pokrycia zbioru}
  \label{alg-pokrycie}
\end{algorithm}

\begin{lemma}
$f_{i} \leq \frac{c(OPT)}{n-i+1}$, gdzie $OPT$ jest rozwiązaniem optymalnym.\\
\end{lemma}
\begin{proof}
Gdyby do pokrycia elementu $e_i$ oraz wszystkich pozostałych elementów, czyli $e_{i+1}, e_{i+2}, ... , e_{n}$ użyłby rodziny zbiorów $OPT$, to cena rynkowa dla każdego z tych elementów wyniosłaby $\frac{c(OPT)}{libcza \ nowo \ pokrytych \ elementów}$, czyli $\frac{c(OPT)}{n-i+1}$.\\
W szczególności istnieje zbiór $Y \in OPT$ taki, że cena rynkowa pokrywanego elementu jest nie większa niż dla całego $OPT$.
Zatem algorytm zachłanny wybiera do pokrycia $e_i$ zbiór o cenie rynkowej pokrywanych elementów $\leq \frac{c(OPT)}{n-i+1}$.
\end{proof}
Koszt algorytmu zachłannego ALG \\
\begin{align*}
c(ALG) &= \sum_{i=1}^{n} f_i \\ &\leq  
\sum_{i=1}^{n} \frac{c(OPT)}{n-i+1}\\ &= 
c(OPT) \cdot \sum_{i=1}^{n} \frac{1}{n-i+1}\\ &=
 c(OPT) \cdot \sum_{i=1}^{n} \frac{1}{i}\\ &=
  c(OPT) \cdot H_n \\ &\leq c(OPT) \cdot \log(n+1)\\
\end{align*}


\section{Przynależność słowa do języka}
\sectionauthor{Przemysław Joniak}

TODO: co oznacza gwiazdka nad zbiorem?

W tym rozdziale przedstawimy algorytm sprawdzające czy dane słowo należy do języka generowanego przez gramatykę bezkontekstową.
Algorytm ten działa w czasie $\Theta(n^3)$ względem długości słowa i jego idea jest oparta na technice programowania dynamicznego.
Na początek wprowadźmy parę definicji i oznaczeń.

\begin{definition}
\textbf{Gramatyka bezkontekstowa} to taka czwórka $\langle N, T, P, S \rangle$, że:
    \begin{itemize}
    	\item \textbf{N} i \textbf{T} to skończone zbiory rozłączne. 
        $N$ nazywamy zbiorem \textbf{nieterminali}, a $T$ zbiorem \textbf{terminali}.
        \item \textbf{P} - zbiór \textbf{produkcji} - to podzbiór $(N \times (N\cup T ))^*$
        \item \textbf{S} to \textbf{symbol startowy} - wyróżniony element ze zbioru produkcji.
    \end{itemize}
\end{definition}
Będziemy przyjmować, że elementy zbioru terminali to duże litery alfabetu angielskiego, np. $N = \{ A, B, S\}$, a elementy zbioru terminali to małe litery, np. $T = \{ a, b, c, \epsilon\}$ (z wyjątkiem epsilonu - jest to \textit{znak pusty}.
Zbiór produkcji to zbiór par $(L, R)$, gdzie L jest terminalem, a R jest ciągiem złożonym z terminiali i nieterminali. 
Parę $(L, R)$ będziemy zapisywali w postaci $L \rightarrow R$, np.: $A \rightarrow aAb$, $A \rightarrow c$, $B \rightarrow b$,$A \rightarrow aSbB$, $S \rightarrow A$. 
Jeżeli w zbiorze produkcji jeden terminal pojawia się "po prawej strzałki" więcej niż raz, np. $B \rightarrow ab$, $B \rightarrow b$, to zapisujemy te dwie produkcje w skrócie: $B \rightarrow ab |  b$, 
Poprawną produkcją nie jest $AB \rightarrow b$, ponieważ z lewej strony strzałki znajdują się dwa nieterminale.

Aby wyprowadzić konkretne słowo, np. $aacbb$, to musimy zacząć od symbolu startowego $S$ i kolejno "podmieniać" terminale zgodnie ze zbiorem produkcji:
 \[ S \Rightarrow A \Rightarrow aAb \Rightarrow aaAbb \Rightarrow aacbb \]
W powyższym przykładzie kolejno skorzystaliśmy z produkcji: $S \rightarrow A$, $A \rightarrow aAb$ (dwukrotnie),   $A \rightarrow c$. Słowo $aabb$ zostało \textit{wyprowadzone} w gramatyce G.

\begin{definition}
Niech $G = \langle N, T, P, S \rangle$, $a,b,c \in (N \cup T)^*$ oraz $A \in N$.
Ze słowa $aAb$ można w G \textbf{wyprowadzić} słowo $acb$ jeżeli $A \rightarrow c$ jest produkcją z P.
Zapisujemy to: $aAb \Rightarrow acb$.
\end{definition}

Jeżeli rozważymy zbiór wszystkich słów, które da się wyprowadzić gramatyce $G$, to zbiór ten nazywamy \textit{językiem $L(G)$ nad gramatyką $G$}:

\begin{definition}
Niech $G = \langle N, T, P, S \rangle$. Język $L(G)$ generowany przez gramatykę $G$ to:
\[ L(G) = \{w: w\in T^* \land S \Rightarrow^* w \}\]
\end{definition}

O $\Rightarrow^*$, czyli o tranzytywnym domknięciu relacji $\Rightarrow$, możemy myśleć jako wielokrotnym zaaplikowaniu (iterowaniu) relacji $\Rightarrow$.

W naszym algorytmie będziemy używać gramatyk bezkontekstowych, w których po prawej stronie każdej produkcji znajdują się albo dwa terminale albo jeden nieterminal:

\begin{definition}
Gramatyka jest w \textbf{postaci Chomsky'ego} jeżeli każda produkcja jest w jednej z poniższych postaci:
	\begin{itemize}
		\item $A \rightarrow BC$ (typ I)
        \item $a$ (typ II)
	\end{itemize}
gdzie $A,B,C$ są nieterminalami i $a$ jest terminalem.
\end{definition}

Zauważmy, że każdą gramatykę bezkontekstową można przedstawić w postaci Chomsky'ego.
Wystarczy każdą produkcję niebędącej w pożądanej postaci rozpisać na kilka innych, poprawnych produkcji. 

Mając daną gramatykę $G$ w postaci Chomsky'ego oraz słowo $w$ możemy zadać pytanie: czy słowo $w = a_1a_2...a_n$ należy do języka generowanego przez tę gramatykę? 
Jeżeli $w$ jest długości jeden, to znaczy, że $w$ jest nieterminalem i wystarczy sprawdzić, czy w $G$ istnieje produkcja $S \rightarrow w$. 
Jeżeli długość $w$ jest większa od $1$ i jeżeli $w$ należałoby do języka, to ostatnia produkcja w wyprowadzeniu $w$ musiała mieć postać $X \rightarrow YZ$ (bo gramatyka jest w postaci Chomsky'ego).
W takim razie słowo $w$ da się podzielić na dwie części $a_1a_2...a_i$ oraz $a_{i+1}...a_n$, takie, że pierwszą da się wyprowadzić z $Y$, a drugą z $Z$. 
Nie znamy indeksu $i$, więc musimy sprawdzić wszystkie możliwe jego wartości.
Następnie tę procedurę powtarzamy zarówno dla $Y$ jak i $Z$.

Niestety, może się okazać, ze wiele takich samych fragmentów słowa $w$ możemy obliczać wielokrotnie.
Takie podejście rekurencyjne skutkuje wykładniczym czasem działania.
Aby temu zapobiec zastosujemy programowanie dynamiczne. 
Zaczniemy analogicznie jak w algorytmie obliczania $n-tej$ liczby Fibonacciego: rozpoczniemy od małych fragmentów, a pośrednie wyniki obliczane iteracyjne będziemy spamiętywać.

\paragraph{Szkic algorytmu} Mamy daną gramatykę  $G = \langle N, T, P, S \rangle$ oraz słowo $w = a_1a_2...a_n$. Chcemy się dowiedzieć czy $w \in L(G)$.
\begin{itemize}
\item Na początku przeglądamy fragmenty $w$ długości jeden, czyli $a_i$ dla $i=1..n$.
Dla każdego $a_i$ musimy sprawdzić czy istnieje taka produkcja, w której po prawej stronie występuje $a_i$. Jeżeli istnieje, to zapamiętujemy nieterminal po lewej stronie produkcji.
\item Teraz będziemy rozważać kolejno wszystkie fragmenty $w$ długości $2,3,...n$:
	\begin{itemize}
		\item Fragmentów długości $2$ jest $n-1$: $a_1a_2$, $a_2a_3$, ..., $a_{n-1}a_n$. 
        \item Fragmentów długości $3$ jest $n-2$: $a_1a_2a_3$, $a_2a_3a_4$, ..., $a_{n-2}a_{n-1}a_n$; itd.
 		\item Będą dwa fragmenty długości $n-1$ ($w$ bez kolejno: ostatniego i pierwszego znaku) oraz jeden długości $n$ - całe słowo $w$.
	\end{itemize}

Weźmy fragment $a_ia_{i+1}...a_j$ ($1 \leq i < j \leq n$).
Gdyby ten fragment należał do języka, to dało by się go wyprowadzić pewną produkcją $X \rightarrow YZ$.
Aby to sprawdzić, musimy ciachnąć $a_ia_{i+1}...a_j$ na dwie niepuste połowy.
Możemy to zrobić na $j-i$ sposobów:
\begin{align*}
&a_i|a_{i+1}a_{i+2}...a_j  &a_ia_{i+1}|a_{i+2}...a_j \\ &a_i...a_k|a_{k+1}...a_j &a_i...a_{n-1}|a_n
\end{align*}

Dla każdego podziału sprawdzamy czy zarówno prawa strona jak i lewa strona dała się wcześniej wyprowadzić - takie sprawdzenie jest darmowe, wcześniej już to policzyliśmy (albo i nie).
Jeżeli te części istnieją, to wystarczy sprawdzić czy istnieje taka produkcja $X \rightarrow YZ$, że z $X$ możemy wyprowadzić pierwszą część fragmentu: $a_i...a_k$, a z $Y$ można wyprowadzić drugą: $a_{k+1}...a_j$. Jak istnieje, to zapamiętujemy nieterminale po prawej stronie produkcji.
\item $w$ należy do języka, jeżeli okaże się, że symbol startowy wyprowadza $w$.

\end{itemize}
Wszystkich fragmentów słowa $w$ jest rzędu $n^2$ i dla każdego fragmentu wykonujemy operacji proporcjonalnie do jego długości. Stąd wykonamy $\Theta(n^3)$ operacji.

Zauważmy, że obliczenia możemy zorganizować w tabeli $n \times n$.
W komórkach przekątnej wpisujemy wyniki kroku pierwszego: nieterminale, które wyprowadzają pojedynczy znak.
W kolejnych przyprzekątnych obliczamy fragmenty długości $2,3...n$.
Jeżeli w komórce [1,n] znajdzie się nieterminal $S$, to dane słowo jest wyprowadzalne.

\paragraph{Przykład}
Niech dana będzie  gramatyka $G$, w której: $T = \{a,b\}$, $N = \{S, A, B\}$, a zbiór produkcji wygląda następująco:
\[
	P = \{S\rightarrow SS|AB, A\rightarrow AS|AA|a, B\rightarrow SB|BB|b \}
\]
oraz dane słowo $w = aabbab$.

Najpierw rozważamy wszystkie fragmenty długości $1$: $a,a,b,b,a,b$. 
Każdy z nich da się wyprowadzić albo z produkcji $A\rightarrow a$ albo z $B\rightarrow b$. 
Skoro $w_{1,1}=a$ oraz $A \rightarrow a$, to do komórki ($1,1$) tabeli wpisujemy $A$. Analogicznie wypełniamy resztę przekątnej:
\begin{table}[h]
\centering
\label{my-label}
\begin{tabular}{ccccccc}
                       & 1                          & 2                          & 3                          & 4                          & 5                          & 6                          \\ \cline{2-7} 
\multicolumn{1}{l|}{1} & \multicolumn{1}{l|}{\{A\}} & \multicolumn{1}{l|}{}      & \multicolumn{1}{l|}{}      & \multicolumn{1}{l|}{}      & \multicolumn{1}{l|}{}      & \multicolumn{1}{l|}{}      \\ \cline{2-7} 
\multicolumn{1}{l|}{2} & \multicolumn{1}{l|}{}      & \multicolumn{1}{l|}{\{A\}} & \multicolumn{1}{l|}{}      & \multicolumn{1}{l|}{}      & \multicolumn{1}{l|}{}      & \multicolumn{1}{l|}{}      \\ \cline{2-7} 
\multicolumn{1}{l|}{3} & \multicolumn{1}{l|}{}      & \multicolumn{1}{l|}{}      & \multicolumn{1}{l|}{\{B\}} & \multicolumn{1}{l|}{}      & \multicolumn{1}{l|}{}      & \multicolumn{1}{l|}{}      \\ \cline{2-7} 
\multicolumn{1}{l|}{4} & \multicolumn{1}{l|}{}      & \multicolumn{1}{l|}{}      & \multicolumn{1}{l|}{}      & \multicolumn{1}{l|}{\{B\}} & \multicolumn{1}{l|}{}      & \multicolumn{1}{l|}{}      \\ \cline{2-7} 
\multicolumn{1}{l|}{5} & \multicolumn{1}{l|}{}      & \multicolumn{1}{l|}{}      & \multicolumn{1}{l|}{}      & \multicolumn{1}{l|}{}      & \multicolumn{1}{l|}{\{A\}} & \multicolumn{1}{l|}{}      \\ \cline{2-7} 
\multicolumn{1}{l|}{6} & \multicolumn{1}{l|}{}      & \multicolumn{1}{l|}{}      & \multicolumn{1}{l|}{}      & \multicolumn{1}{l|}{}      & \multicolumn{1}{l|}{}      & \multicolumn{1}{l|}{\{B\}} \\ \cline{2-7} 
\end{tabular}
\end{table}

Teraz fragmenty długości $2$: $aa, ab, bb, ba, ab$.
\begin{itemize}
\item Fragment $w_{1,2} = aa$ można tylko na jeden sposób podzielić na dwie części: $a$ oraz $a$. 
Z poprzedniego kroku wiemy, że $a$ dało się już wyprowadzić z nieterminala $A$.
Istnieje produkcja $A \rightarrow AA$, więc do komórki (1,2) wpisujemy $A$.
\item $w_{2,3} = ab$. Dzielimy na pół. Z poprzeniej iteracji wiemy, że da się wyprowadzić słowo $a$ oraz $b$ z kolejno terminala $A$ oraz $B$. Istnieje produkcja $S\rightarrow AB$, więc do komórki (2,3) wpisujemy $S$.
\item Analogicznie wypełniamy resztę przyprzekątnej. 
Zauważmy jednak, że np. przy $w_{4,5}$ istnieją nietermianle, z których da się wyprowadzić $b$ oraz $a$, ale nie istnieje produkcja, która wyprowadza te nieterminale (nie ma produkcji postaci: $ X \rightarrow BA$). 
Zatem do komórki (4,5) nic nie wpisujemy:
\end{itemize}

\begin{table}[h]
\centering
\label{my-label}
\begin{tabular}{lllllll}
                       & 1                          & 2                          & 3                          & 4                          & 5                          & 6                          \\ \cline{2-7} 
\multicolumn{1}{l|}{1} & \multicolumn{1}{l|}{\{A\}} & \multicolumn{1}{l|}{\{A\}} & \multicolumn{1}{l|}{}      & \multicolumn{1}{l|}{}      & \multicolumn{1}{l|}{}      & \multicolumn{1}{l|}{}      \\ \cline{2-7} 
\multicolumn{1}{l|}{2} & \multicolumn{1}{l|}{}      & \multicolumn{1}{l|}{\{A\}} & \multicolumn{1}{l|}{\{S\}} & \multicolumn{1}{l|}{}      & \multicolumn{1}{l|}{}      & \multicolumn{1}{l|}{}      \\ \cline{2-7} 
\multicolumn{1}{l|}{3} & \multicolumn{1}{l|}{}      & \multicolumn{1}{l|}{}      & \multicolumn{1}{l|}{\{B\}} & \multicolumn{1}{l|}{\{B\}} & \multicolumn{1}{l|}{}      & \multicolumn{1}{l|}{}      \\ \cline{2-7} 
\multicolumn{1}{l|}{4} & \multicolumn{1}{l|}{}      & \multicolumn{1}{l|}{}      & \multicolumn{1}{l|}{}      & \multicolumn{1}{l|}{\{B\}} & \multicolumn{1}{l|}{-}     & \multicolumn{1}{l|}{}      \\ \cline{2-7} 
\multicolumn{1}{l|}{5} & \multicolumn{1}{l|}{}      & \multicolumn{1}{l|}{}      & \multicolumn{1}{l|}{}      & \multicolumn{1}{l|}{}      & \multicolumn{1}{l|}{\{A\}} & \multicolumn{1}{l|}{\{S\}} \\ \cline{2-7} 
\multicolumn{1}{l|}{6} & \multicolumn{1}{l|}{}      & \multicolumn{1}{l|}{}      & \multicolumn{1}{l|}{}      & \multicolumn{1}{l|}{}      & \multicolumn{1}{l|}{}      & \multicolumn{1}{l|}{\{B\}} \\ \cline{2-7} 
\end{tabular}
\end{table}

Teraz będziemy rozważać fragmenty długości $3$: $aab, abb, bba, bab$. 
\begin{itemize}
\item 
\end{itemize}


\section{Pokrycie wierzchołkowe}
\sectionauthor{Krzysztof Starzyk}

\label{sec:vertex-cover}

MPK Wrocław chciałoby dokonać inspekcji wszystkich torów tramwajowych w mieście (wystarczy odwiedzić przystanek by dokonać inspekcji incydentnych) torów. Podwykonawca (TOREX) żąda opłaty za każdy odwiedzony przystanek z osobna więc MPK chciałoby zminimalizować liczbę przystanków które będzie musiała odwiedzić ekipa TOREXu (zachowując przy tym podstawowe zadanie jakim jest inspekcja WSZYSTKICH torów). 
MPK Wrocław cannot into informatyka więc nie rozwiąże tego problemu (rozwiąże go podwyżką cen biletów). My co prawda nie potrafimy w czasie wielomianowym dostarczyć żądanego rozwiązania, ale wiemy jak się do tego zabrać.

Bardziej formalnie:  

\begin{definition}
  \textbf{Pokryciem wierzchołkowym} (\texttt{dal.} PW) grafu $G = (V,E)$ nazywamy zbiór $V'$ t. że:
  $V' \subseteq V \wedge (\forall e\in E, \exists v\in V':  v \in e)$.
\end{definition}
\textbf{Problem pokrycia wierzchołkowego} (\texttt{dal.} PPW) będziemy rozpatrywać na dwa sposoby: optymalizacyjnym (jakie jest najmniejsze PW) i decyzyjnym (czy istnieje PW rozmiaru $k$); posiadając wszelakie zastosowania praktyczne jest oczywiście problemem NP-zupełnym.
  
Jeżeli przystanki tramwajowe potraktujemy jako wierzchołki a tory między  nimi jako krawędzie, to PW nazwiemy taki podzbiór przystanków że wszystkie tory mają przynajmniej jeden koniec kończący się na przystanku z naszego podzbioru. 

Przybliżone rozwiązanie:
W dość prosty sposób możemy uzyskać przybliżone rozwiązanie (co najmniej(?) dwukrotnie gorsze od optymalnego w sensie mocy otrzymanego zbioru). Idea jest następująca: dla każdej krawędzi $e$ w $E'$ ($=E$) weźmy dwa wierzchołki które $e$ łączy, dodajmy je do zbioru rozwiązań i usuńmy z $E'$ wszystkie krawędzie incydentne do nich. 

\begin{algorithm}[h]
  \DontPrintSemicolon
  \SetAlgorithmName{}{}

  \KwData{ $G = (V, E)$}
  \KwResult{ $S$ - pewne \texttt{PW} dla grafu $G$  }
  $E' = E$
  \While{$E' != \{\}$}
  {
  $e'$ - dowolna krawędź łącząca wierzchołki $(u,v)$
  $S = S \bigcup u$ 
  $S = S \bigcup v$
  $E' = E' \backslash$ (wszystkie krawędzie incydentne do $u$ i $v$) 
  }
  \caption{Przybliżone rozwiązanie \texttt{PPW}}
  \label{problem-pokrycia-wierzcholkowego}  
\end{algorithm}




\section{Algorytm znajdowania dwóch najbliższych punktów}

Teraz zajmiemy się problemem znalezienia pary najmniej odległych punktów w zadanym zbiorze $Q = \{(x_1, y_1), \ldots, (x_n, y_n)\}$.
Interesować nas będzie odległość euklidesowa, czyli szukamy takich indeksów $i, j$, że $d(p_i, p_j) = \min\{d(p_k, p_l) \quad | \quad 1 \leq p < l \leq n\}$, gdzie $d(p_k, p_l) = \sqrt{(x_k - x_l)^2 + (y_k - y_l)^2}$

\subsection{Podejście siłowe}

W podejściu siłowym potrzebujemy wyznaczyć i porównać wszystkie odległości pomiędzy punktami. 
Jest ich ${|Q|}\choose{2}$, czyli w taki sposób problem można rozwiązać w czasie $O(n^2)$. 
W następnym rozdziale pokażemy jak zrobić to szybciej.

\subsection{Podejście Dziel i Zwyciężaj}

Wiemy już, że problem ten można naiwnie rozwiązać w czasie $O(n^2)$.
Zastanówmy się jak wykorzystując strategię Dziel i Zwyciężaj zrobić to szybciej.
Już po chwili zastanowienia widać, że nie jest to takie trywialne, ponieważ po podziale zbioru na 2 części i znalezieniu dla nich rozwiązań, i tak musielibyśmy sprawdzić wszystkie odległości pomiędzy tymi zbiorami. 
Ale czy na pewno?
\\\\
Nasz algorytm będzie wywoływał się rekurencyjnie, więc w celu uniknięcia wielokrotnego sortowania, wykorzystamy dwie tablice $X$ i $Y$, które zawierać będą wszystkie punkty z Q posortowane odpowiednio po $x$'owej i $y$'owej współrzędnej.  
W tym miejscu istotnym jest, aby punkty ze wszystkich tablic były odpowiednio \textit{połączone} między sobą.  
Dzięki temu przy podziale zbioru na dwie części: $Q_L$ i $Q_R$ odtworzenie tablic $X_L$, $Y_L$ i $X_R$, $Y_R$ będzie możliwe w czasie $O(|Q|)$.  
Wystarczy przeglądać po kolei elementy z tablic $X$ oraz $Y$ i przerzucać je do mniejszych odpowiedników, w zależności czy punkt jest w części $L$ czy $R$.
\\\\
\begin{algorithm}[H]
 	\DontPrintSemicolon
  	\SetAlgorithmName{Algorytm}{}\\
  	\KwData{ $Q$ - zbiór punktów,  $X$, $Y$ }\\
  	\KwResult{ najmniejsza odległość między punktami }\\
    	\If{$|Q|  <  2$}  { \Return $\infty$ }
        \If {$|Q| = 2$} {\Return $Q[1] - Q[2]$}
		\texttt{wykorzystując tablicę $X$ znajdź prostą $l$ dzielącą zbiór Q na dwa prawie równoliczne zbiory}\;
		\texttt{podziel Q na zbiory $Q_L$ i $Q_R$ względem prostej $l$ odpowiednio po jej lewej i prawej stronie}\;
        \texttt{wyznacz tablice $X_L$, $Y_L$ i $X_R$, $Y_R$}\;
        $d_L$ $\leftarrow$ \texttt{NAJMNIEJ-ODLEGLA-PARA($Q_L$, $X_L$, $Y_L$)}\;
        $d_R$ $\leftarrow$ \texttt{NAJMNIEJ-ODLEGLA-PARA($Q_R$, $X_R$, $Y_R$)}\;
         $d$ $\leftarrow$ $\min{(d_L, d_R)}$\;
         $Y'$ $\leftarrow$ \texttt{punkty z $Y$ odległe o co najwyżej $d$ od prostej $l$}\;
         \For {$i = 1$ to  $|Y'|$ } {
         	\For {$j = 1$ to  $\min{(7, |Y'| - i)}$ } {
            	\If {$P[i] - P[i+j] < d$} {
                	$d$ $\leftarrow$ $|P[i] - P[i+j]|$
                }
            }
         }
        \Return $d$
  	\caption{Implementacja procedury NAJMNIEJ-ODLEGLA-PARA}
\end{algorithm}


\section{Kopce dwumianowe w wersji leniwej}
\sectionauthor{Marcin Bartkowiak}

\label{sec:leniwedwumianowe}

Kopce dwumianowe w wersji leniwej różnią się strukturalnie od wersji gorliwej, tym, że w danym momencie możemy mieć więcej niż jeden kopiec $B_k$ na liście.
\subsection{Różnice w implementacji}

\subsubsection{insert}
Stwórz drzewo składające się wyłącznie z danego elementu a następnie wywołaj \texttt{meld} z kopcem właściwym.
\subsubsection{meld}
Operacja \texttt{meld} polega na połączeniu list drzew dwóch kopców.
\subsubsection{extract-min}
W wersji leniwej, podobnie jak w Kopcach Fibonacciego
to tutaj będziemy wykonywać całą pracę związaną z utrzymaniem struktury kopców.

Idea algorytmu:
\begin{enumerate}
 \item Usuń min z listy wierzchołków, a następnie dodaj do listy wierzchołków jego dzieci.
 \item Stwórz pustą tablicę $B$ wielkości największemu stopniowi drzewa niezbędnego, w kopcu o poprawnej strukturze trzymającym wszystkie elementy($\lceil \log n \rceil$)
 \item Iterując po każdym drzewie w kopcu sprawdź, czy to nie minimum(i ustaw wskaźnik minimum jeśli jest),
      a następnie sprawdź w tablicy $B$ jest element o indeksie jego wielkości.
      Jeśli nie ma wstaw go do tablicy. Jeśli istnieje połącz dane drzewa i rekurencyjnie wstaw nowe drzewo do tablicy.
\end{enumerate}

\subsubsection{Pozostałe operacje}
Pozostałe operacje implementujemy identycznie jak w gorliwych kopcach dwumianowych.

\subsection{Analiza złożoności}
Zdefiniujmy funkcję potencjału $\Phi = \#drzew~w~kopcu$
\subsubsection{meld}
\texttt{Meld} w oczywisty sposób nie zmienia sumy potencjałów kopców, jedynym kosztem będzie przepięcie wskaźników, więc złożoność tej funkcji to $\Theta(1)$
\subsubsection{insert}
Dodając drzewo zwiększamy potencjał o jeden.
Następnie będziemy musieli wykonać \texttt{meld}, które kosztuje jedną operacje.
\[
  \Delta(\Phi) + 1 = 1 + 1 = 2 \in \Theta(1)
\]
\subsubsection{extract-min}
Na początku będziemy musieli wstawić wszystkie dzieci od minimalnego elementu; zajmie to $O(\log n)$.

Oznaczmy $T$ jako wszystkich drzew po tej operacji.

Dominującym kosztem rzeczywistym łączenia będzie iteracja po wszystkich drzewach(w czasie $\Theta(T)$).

Niech $\Delta(\Phi)$ oznacza różnicę potencjałów między kopcem po dodaniu dzieci minimalnego elementu, a kopcem po złączenie drzew tego samego stopnia.
Koszt zamortyzowany wyrażać się więc będzie wzorem.
\[
  \Delta(\Phi) + O(\log n) + \Theta(T) = O(\log(n)) - T + O(\log(n)) + \Theta(T) = O(\log(n))
\]



\section{Szybka Transformata Fouriera}
\sectionauthor{Wiktor Garbarek}

%\newthoerem{remark_small}{Mała uwaga}
\newtheorem{observation_small}{Malutka obserwacja}
\newtheorem{observation_bigger}{Nieco większa obserwacja}


\label{sec:fft}

Zacznijmy od problemu mnożenia wielomianów. Mamy dwa wielomiany stopnia co najwyżej, dla późniejszej wygody w zapisie, $n-1$, powiedzmy $A(x) = \sum_{i=0}^{n-1} a_i\cdot x^i$ i analogicznie $B(x)$, gdzie $a_i$ oraz $b_i$ są rzeczywiste.
Szukamy zatem wielomianu $$C(x) = \sum_{i=0}^{2n-2} c_ix^i$$, gdzie $c_i = \sum_{j=0} a_jb_{i-j} = \sum_{p + q = i} a_pb_q $ (oba te zapisy $c_i$ są równoważne, jeśli założymy, że $ 0 \leq p,q$).
Na pierwszy rzut oka możemy zauważyć, że istnieje algorytm obliczający $C(x)$ w czasie $\Theta(n^2)$. Czy da się lepiej? Ano owszem, bo inaczej nie byłoby tego rozdziału.\\

%\begin{remark_small}
    Mała uwaga, bardzo podobne działanie na wektorach $a,b\in \mathbb{R}^n$ matematycy nazywają splotem.
    $$c = a * b \Leftrightarrow c_i = \sum_{j=1}^{n}a_jb_{i-j}$$.
%\end{remark_small}

Zacznijmy od paru obserwacji
\begin{observation_small} (Interpolacja)
    Każdy wielomian stopnia $n-1$ o współczynnikach rzeczywistych da się przedstawić jednoznacznie jako $n$-elementowa lista par $(x_i, W(x_i))$, gdzie ciąg $x_i$ jest dowolnym ciągiem parami różnych liczb rzeczywistych.
\end{observation_small}
\begin{proof}
    Prosty dowód pozostawiamy czytelnikowi.
\end{proof}

\begin{observation_bigger}
    Jeśli mamy wielomiany $A(x)$ oraz $B(x)$ w powyższej postaci obliczonej dla jednego (wspólnego) ciągu $x_i$ (długości $2n$, to ważne!, ale nas to nie boli), to "łatwo" (tj. w czasie liniowym) możemy znaleźć ich iloczyn.
\end{observation_bigger}
\begin{proof}
    Wystarczy powiedzieć, że $C(x)$ to ciąg $(x_i, A(x_i)B(x_i))$.
\end{proof}
Super! W takim razie, jedyne co pozostaje nam, to znaleźć algorytm, który zamieni wielomian w postaci listy współczynników na próbki naszego wielomianu w $2n$ punktach, obliczymy prosto iloczyn tych wielomianów i jeszcze musimy teraz wrócić do domu obliczając współczynniki wielomianu interpolacyjnego.
No i tutaj jest cały problem, bo trywialny algorytm obliczy każdą tą zamianę postaci w czasie $\Theta(n^2)$ (Mamy $2n$ argumentów, dla każdego z nich obliczamy wartość wielomianu w czasie $\Theta(n)$)
Spróbujmy zastosować tutaj metodę divide and conquer. Najpierw jednak zauważmy ciekawą zależność.

\begin{observation_small}
Dla każdego wielomianu zachodzi następująca równość $$A(x) = A^e(x^2) + xA^o(x^2)$$, gdzie $A^e(x) = \sum_{i=0}^{n/2 - 1}a_{2i}x^i$ oraz $A^o(x) = \sum_{i=0}^{n/2 - 1} a_{2i+1}x^i$
('e' oznacza \textit{even}, elementy na parzystych indeksach, a 'o' oznacza \textit{odd})
\end{observation_small}
Niektórzy mogą pomyśleć, że to już koniec. Prosto bierzemy sobie wielomian $A(x)$, jakiś dowolny zbiór $X$, obliczamy rekurencyjnie wartość dla wielomianów $A^e$ i $A^o$ przy wykorzystaniu zbioru $X' = \{x^2 : x \in X\}$.
Nic bardziej mylnego, zauważmy bowiem, że gdy będziemy łączyć wyniki, to tak naprawdę za każdym razem wykonamy $\Theta(|X|)$ kroków, a przy dowolnie wybranym zbiorze, ten zbiór $X'$ wcale nie musi (i co najbardziej prawdopodobne, nie będzie) zmniejszać swojego rozmiaru
(Nota bene taki algorytm dalej ma złożoność $\Theta(n^2)$).
Ale jednak jest nadzieja! Możemy go wybrać dowolnie, więc wystarczy znaleźć $X$ taki, że $|X'| = \frac{|X|}{2}$. Brzmi niewykonalnie? Jednak wcale nie jest. Popatrzmy sobie na pierwiastki jedynki na płaszczyźnie zespolonej. Spójrzmy na przykład dla pierwiastków ósmego stopnia.
Gdy weźmiemy dowolną liczbę z tego zbioru i podniesiemy ją do kwadratu, to wtedy jej moduł się nie zmienia, a jedyne co robimy to podwajamy kąt między jej promieniem wodzącym, a dodatnią półosią OX. W takim razie zamiast rozpatrywać kąty 0, 45, 90, 135, 180, 225, 270, 315 będziemy rozpatrywać kąty 0, 90, 180, 270, 360, 450, 540, 630. Jednakże obrót o 450 stopni, to tak naprawdę obrót o 90, 540 to tak naprawdę 180 stopni itd.
Czyli widzimy, że podnosząc do kwadratu wszystkie elementy w tym zbiorze zmniejszyliśmy jego liczność do połowy. (I znaleźliśmy też pierwiastki czwartego stopnia z jedynki).

Reasumując obliczamy w takim razie $$\{W(e^{\frac{2k\pi i}{2n}}) = \sum_{j=0}^{n-1}a_je^{\frac{2k\pi ij}{2n}} | k = 0,...,2n-1\}$$
Żeby jednak nasze indeksowanie ładnie wyglądało, to możemy pomyśleć, że tak naprawdę $a_n, a_{n+1}, ..., a_{2n-1}$ są zerami, a wtedy możemy zapisać to przekształcenie w ten sposób.
Mając wektor $\vec{a} = [a_0, ..., a_{N-1}]$, szukamy wektora $\vec{y} = [y_0, ..., y_{N-1}]$ takiego, że
$$y_k = \sum_{j=0}^{N-1}a_je^{\frac{2k\pi ij}{N}}$$. Oto jest właśnie (już prawie!) Dyskretna Transformata Fouriera.
\begin{definition}
    Dyskretną Transformatą Fouriera nazywamy następującą funkcję $dft: \mathbb{C}^N \rightarrow \mathbb{C}^N$
    gdzie, jeśli $\vec{y} = dft(\vec{a})$ to $$y_k = \sum_{j=0}^{N-1}a_je^{\frac{-2k\pi ij}{N}}$$
\end{definition}
Niby jedyną różnicą (i aż jedyną!) jest minus - jaki to ma wpływ na wynik w praktyce, pozostawiamy czytelnikowi do sprawdzenia.
Algorytm ten zawdzięczamy Cooley'owi oraz Tukey'owi (1965), jednakże idea ta była znana już 160 lat wcześniej Gaussowi.

Poruszmy jeszcze jedną ważną rzecz - przede wszystkim w powyższych rozważaniach niemo wymagaliśmy, by długość naszego wektora była potęgą dwójki. Jeśli chodzi natomiast o problem mnożenia wielomianów, w żaden sposób nas to nie boli, ponieważ zawsze możemy dopełnić wektor jakąś ilością zer do najbliższej potęgi dwójki, a skoro wydłużył on się co najwyżej dwukrotnie, to ta idea nie zmienia złożoności całego algorytmu.
Nie obchodzą nas też w żaden sposób wartości pośrednie (czytaj: zamiana wielomianu z wektora współczynników, na wektor punktów do interpolacji)
Jednak gdyby zależało nam na policzeniu powyższego działania (tj. nie mnożenia wielomianów, a już obliczenia DFT dla dowolnego wektora) pojawia się duży problem. Przede wszystkim całą zabawę psuje liczba $\exp{\frac{\mathunderscore}{N}}$, której nie pozbędziemy się w żaden trywialny sposób.

Wracając jednak do algorytmu, zapiszemy go jeszcze w pseudokodzie.
\begin{algorithm}[H]
  \DontPrintSemicolon
  \SetAlgorithmName{Algorytm}{Cooley-Tukey}

  \KwData{ $x = [x_0, x_1, ... x_{N-1}] \in \mathbb{C}^n$ - wektor liczb zespolonych długości N}

  \KwResult{$y \in \mathbb{C}^n$, taki, że $y_k = \sum_{n=0}^{N-1}x_n \cdot \exp{\frac{-2\pi i n k}{N}}$ dla $k = 0,1,...,N-1$}

  \For{$i \leftarrow 1$ to $n$}
  {
     \For{$j \leftarrow 1$ to $p$}
     {
	$C[i][j] \leftarrow 0$\;
	\For{$r \leftarrow 1$ to $m$}
	{
	  $C[i][j] \leftarrow C[i][j] + A[i][r] \cdot B[r][j]$\;
	}
     }
  }

  \caption{Algorytm Cooleya-Tukeya}
  \label{alg-fft-cooley-tukey}
\end{algorithm}


\section{B-drzewa}
\sectionauthor{Jeszcze Nikt}

\label{sec:bdrzewa}

Todo, todo, todo...

\section{Sortowanie ciągów różnej długości}
\sectionauthor{Jeszcze Nikt}

\label{sec:sortowanieciagow}

Todo, todo, todo...

% vim: set ts=2 sw=2 :
\section{Algorytm Shift-And}
\sectionauthor{Mateusz Urbańczyk}

\label{sec:shiftand}

Rozważmy następujący problem (ang. \textit{string searching}). Dla zadanych dwóch ciągów
znaków, tekstu i wzorca, odpowiedzieć na pytanie, czy wzorzec zawiera się w tekście.
Mówiąc inaczej, czy tekst zwiera spójny podciąg znaków, który jest równy wzorcowi.
Zdefiniujmy problem bardziej formalnie. \\

Niech $\Sigma$ będzie skończonym alfabetem. Dla zadanego $\mathcal{T},\mathcal{P} \in \Sigma^*$
takiego, że $|\mathcal{P}| \le |\mathcal{T}|$ (zwykle $|\mathcal{P}| \ll |\mathcal{T}|$), chcemy
odpowiedzieć na pytanie, czy

\begin{equation}
    \label{eq:stringSearchingDefinition}
    \mathlarger{\exists_{0 \leq j \leq n - m}} \enspace
    \textit{\textsf{t. że}} \enspace \mathcal{T}[j..j+m] = \mathcal{P}
\end{equation}

gdzie $m = |\mathcal{P}|$ oraz $n = |\mathcal{T}|$. Posłużymy się tymi oznaczeniami także w dalszej
części opisu. Algorytm \textit{Shift-And}, który zostanie omówiony, będzie obliczał wszystkie
prefiksy $\mathcal{P}$ które są suffiksami $\mathcal{T}[0..j]$ dla $j=0..n$. Wynik trzymany będzie
w masce bitowej $\mathcal{D} = d_m..d_1$ ($\forall_{1 \leq i \leq m} d_{i} \in \{0, 1\} $), która dla danej iteracji $j$ będzie spełniać niezmiennik:

\begin{equation}
 \label{eq:invariant}
 \mathcal{D}[i] =
  \begin{cases}
      1 & \textit{\textsf{jeśli}} \enspace \mathcal{P}[0..i] = \mathcal{T}[j-i..j]
            \quad\quad dla \enspace i=0..m \\
      0 & w.p.p
  \end{cases}
\end{equation}

Ponadto, zdefiniujmy sobie również tablicę $\mathcal{B}$, która dla każdego znaku będzie
trzymać maski bitowe wystąpień we wzorcu:

$$
    \forall x \in \mathcal{L} = \{c: c \in \mathcal{P} \wedge c \notin \mathcal{L}\}
$$
$$
    \mathcal{B}[x][i] =
     \begin{cases}
         1 & \textit{\textsf{jeśli}} \enspace \mathcal{P}[i] = c \quad\quad dla \enspace i=0..m \\
         0 & w.p.p
     \end{cases}
$$

Przejdźmy teraz do faktycznego pomysłu na algorytm. \textbf{Idea:} przechodząc od lewej do prawej,
dla każdego znaku w tekście szukamy najdłuższego prefiksu we wzorcu, który również jest
suffiksem w aktualnym oknie, gdzie okno jest ciągiem znaków z $\mathcal{T}$ o długości $m$,
kończące się na aktualnie rozważanym znaku. Prefiksy będziemy znajdować wykorzystując następujące
operacje bitowe na tablicy $\mathcal{D}$ zachowujące niezmiennik (\ref{eq:invariant}):

\begin{equation}
    \label{eq:resultUpdate}
    \mathcal{D'} \leftarrow ((\mathcal{D} \ll 1 ) \enspace | \enspace 1)
        \enspace \& \enspace \mathcal{B}[\mathcal{T}[j]]
\end{equation}

\begin{lemma}
    Dla $\mathcal{D}$ w iteracji $j$ spełniona jest równoważność:
    $$
        \mathcal{D'}[i+1] = 1 \Leftrightarrow{}
            \mathcal{D}[i] = 1 \wedge{} \mathcal{T}[j+1] = \mathcal{P}[i+1]
    $$
    gdzie $\mathcal{D'}$ to wynik w następnej iteracji algorytmu.
\end{lemma}
\begin{proof}
    Jeżeli lemat jest prawdziwy, to zachowany będzie również niezmiennik algorytmu. \\
    $(\Rightarrow)$
    Weźmy dowolne $i$, $j$ i rozważmy $\mathcal{D'}$. Dowód przeprowadzimy nie wprost.
    Załóżmy, że
        $\mathcal{D'}[i+1] = 1$
    $\wedge$
        ($\mathcal{D}[i] \neq 1 \vee{} \mathcal{T}[j+1] \neq \mathcal{P}[i+1]$)
    i rozważmy przypadki:
    \begin{itemize}
        \item $\mathcal{D}[i] \neq 1$. To implikuje, że $\mathcal{D}[i]$ = 0.
           W kolejnej iteracji wykonamy
           $\mathcal{D} \ll 1, zatem \enspace \mathcal{D'}[i+1] = 0$. Sprzeczność
        \item $\mathcal{T}[j+1] \neq \mathcal{P}[i+1]$.
            Zauważmy, że wtedy $\mathcal{B}[\mathcal{T}[j+1]][i+1] = 0$,
            a stąd bit $\mathcal{D'}[i+1]$ zostanie wyzerowany po operacji \textit{AND}. Sprzeczność.

    \end{itemize}
    $(\Leftarrow)$
    Analogicznie. Zostawiam jako proste ćwiczenie dla czytelnika.
\end{proof}

\begin{observation}
Jeżeli po wykonaniu iteracji $\mathcal{D}[m-1] = 1$, to nasz wzorzec występuje w tekście.
\end{observation}

Dokładny algorytm wynika wprost z powyższej obserwacji oraz z (\ref{eq:resultUpdate}):

\begin{algorithm}[h]
  \DontPrintSemicolon
  \SetAlgorithmName{Algorytm}{}

  \KwData{\enspace $\mathcal{T}, \mathcal{P} \in \Sigma, |\mathcal{P}| \le |\mathcal{T}|$}

  \KwResult{\enspace ewaluacja wyrażenia (\ref{eq:stringSearchingDefinition})}

  \For{$c \in \Sigma$} {
      $\mathcal{B}[c] \leftarrow 0$
  }
  \For{$i=1..m$} {
      $
        \mathcal{B}[\mathcal{P}[i]] \leftarrow
          \mathcal{B}[\mathcal{P}[i]] \enspace | \enspace (1 \ll (i-1))
      $
  }
  $\mathcal{D} \leftarrow 0$ \\
  \For{j=1..n} {
    $
      \mathcal{D} \leftarrow ((\mathcal{D} \ll 1 ) \enspace | \enspace 1)
        \enspace \& \enspace \mathcal{B}[\mathcal{T}[j]]
    $ \\
    \If{$\mathcal{D} \enspace \& \enspace (1 \ll (m-1))$}{
          \Return true
    }
  }
  \Return false

  \caption{Algorym Shift-And}
  \label{alg-wiesniakow}
\end{algorithm}




\section{Algorytm szeregowania}
\sectionauthor{Piotr Kowalczyk}

\label{sec:szeregowanie}

Ten rozdział jest poświęcony prostemu problemowi szeregowania zadań z terminami dla pojedynczego procesora, który da się rozwiązać algorytmem zachłannym (co jest wyjątkowe, ponieważ większość problemów szeregowania jest NP-trudna).

Problem: system z jednym procesorem ma do wykonania $n$ zadań.
Każde z zadań wykonuje się przez jedną jednostkę czasu.
Dla każdego zadania znane są: deadline $d_i \in \mathbb{N}$ oraz zysk $g_i \in \mathbb{R}_{+}$.
Zadanie musi być wykonane przed upływem deadline'u, w przeciwnym wypadku zysk za to zadanie wynosi 0.
Naszym zadaniem jest wyznaczyć wykonywalny podzbiór zbioru zadań, który maksymalizuje sumę zysków.

W takim razie, jaki zbiór zadań jest wykonywalny?

\begin{definition}
\textbf{Wykonalnym ciągiem zadań} nazywamy ciąg $\left\langle i_1, i_2, \dots, i_n \right\rangle$ taki, że $\forall_{k \in \{1, 2, \dots, n\}} k \leq d_{i_k}$. \\
\textbf{Wykonalnym zbiorem zadań} nazywamy zbiór, którego wszystkie elementy można ustawić w ciąg wykonalny.
\end{definition}

Teraz możemy już przedstawić \textbf{strategię zachłanną}: zaczynamy od pustego zbioru zadań $S$.
W każdym kroku znajdujemy zadanie $z$ o największym zysku spośród jeszcze nierozważonych i jeśli zbiór $S \cup \{z\}$ jest wykonalny, to dodajemy zadanie $z$ do $S$.

\begin{proof}
Załóżmy, że algorytm zachłanny (bazujący na powyższej strategii) wybrał zbiór zadań $I$ oraz że istnieje zbiór optymalny $J$ taki, iż $I \neq J$.
Pokażemy, że suma zysków z wykonania zadań jest taka sama dla tych zbiorów.
Niech $\pi_I$, $\pi_J$ będą wykonywalnymi ciągami zadań odpowiednio z $I$ oraz $J$.
Najpierw, wykonując przestawienia, otrzymamy wykonywalne ciągi $\pi ^{\prime}_I$, $\pi^{\prime}_J$, w których wszystkie zadania wspólne dla zbiorów $I$ oraz $J$ (tj. takie, które należą do $I \cap J$) wykonują się w tym samym czasie.

Niech $a \in I \cap J$ będzie zadaniem umieszczonym na różnych pozycjach, tj. na pozycji $i$ w $\pi_I$ oraz pozycji $j$ w $\pi_J$, gdzie $i \neq j$.
Załóżmy BSO, że $i < j$.
Niech $b$ będzie zadaniem, które jest na pozycji $j$ w ciągu $\pi_I$.
Zamieńmy miejscami zadania $a$ oraz $b$ w $\pi_I$.
Niech otrzymany ciąg zwie się $\pi^{\prime\prime}_I$.
Ciąg $\pi^{\prime\prime}_I$ jest wykonalny, ponieważ $d_a \geq j$ (dlatego, że $a$ jest na pozycji $j$ w $\pi_J$) oraz $d_b > i$ (dlatego, że $b$ jest w $\pi_I$ na pozycji $j$).
Liczba zadań z $I \cap J$, które są na różnych pozycjach w ciągach $\pi^{\prime\prime}_I$, $\pi_J$ jest o co najmniej jeden mniejsza, niż w ciągach $\pi_I$, $\pi_J$.
Iterując postępowanie otrzymujemy tezę.

Teraz pokażemy, że ciągi $\pi_I \prime$, $\pi_J \prime$ mają na każdej pozycji zaplanowane zadania o tym samym zysku.

Rozważmy dowolną pozycję $i$. Jeśli ciągi $\pi^{\prime}_I$, $\pi^{\prime}_J$ mają na pozycji $i$ to samo zadanie, to zysk z tego zadania jest taki sam dla zbiorów $I$ oraz $J$.
W przeciwnym przypadku niech $a$, $b$ będą zadaniami z pozycji $i$ z, odpowiednio, $\pi^{\prime}_I$ i $\pi^{\prime}_J$.
Zauważmy, że oba ciągi mają jakieś zadanie na tej pozycji (gdyby $\pi^{\prime}_I$ miał tę pozycję wolną, algorytm włożyłby tam zadanie $b$, a gdyby $\pi^{\prime}_J$ miał tę pozycję wolną, to zbiór $J \cap \{a\}$ też jest wykonywalny i lepszy niż $J$, czyli $J$ nieoptymalny - sprzeczność).

Wystarczy teraz pokazać, że $g_a = g_b$. \\
Załóżmy, że $g_a > g_b$.
Stąd mamy $J \setminus \{b\} \cup \{a\}$ daje większy zysk niż $J$ - sprzeczność. \\
Załóżmy, że $g_a < g_b$.
Zauważmy, że teraz algorytm rozpatruje najpierw zadanie $b$, a później $a$.
Zauważmy też, że zbiór $I \setminus  \{a\} \cup  \{b\}$ jest wykonalny, a więc każdy jego podzbiór jest wykonalny, w szczególności ten podzbiór, który był skonstruowany przez algorytm w momencie rozpatrywania zadania $b$.
Stąd wynika, że zadanie $b$ byłoby dołączone do rozwiązania przez algorytm - sprzeczność.

Stąd wynika, że ciągi $\pi^{\prime}_I$, $\pi^{\prime}_J$ mają na każdej pozycji zaplanowane zadania o tym samym zysku, czyli sumy zysków obu ciągów są równe.
\end{proof}

Pozostaje jeszcze tylko pytanie, jak ustalić, czy zbiór złożony z $k$ zadań jest wykonywalny. Można to robić, sprawdzając wykoknywalność wszystkich $k!$ ciągów, ale wystarczy sprawdzić jeden ciąg.

\begin{lemma}
Niech $I$ będzie dowolnym zbiorem zadań, a $k$ będzie mocą $I$.
Niech $\sigma = (s_1, s_2, \dots, s_k)$ będzie taką permutacją zadań ze zbioru $I$, że $d_{s_1} \leq d_{s_2} \leq \dots \leq d_{s_k}$.
Wówczas $J$ jest zbiorem wykonywalnym wtedy i tylko wtedy, gdy $\sigma$ jest ciągiem wykonywalnym.
\end{lemma}

\begin{proof}
Dowód lematu jako oczywisty pomijamy.
\end{proof}

Teraz możemy w łatwy (i dość nieefektywny sposób) zaimplementować ww. strategię, na początku sortując zadania malejąco według zysków (tak, że $g_1 \geq g_2 \geq \dots \geq g_n$), potem tworząc pusty ciąg $\sigma$ (w szczególności posortowany rosnąco według deadline'ów), a następnie, dodając zadanie numer $i$ do ciągu, używać procedury podobnej do \texttt{insert} z algorytmu sortowania przez wstawianie, zachowywać w $\sigma$ porządek (rosnący, według deadline'ów).
Taka implementacja działa w czasie $\Omega(n^2)$.


\section{Algorytm Borůvki}
\sectionauthor{Michał Bronikowski}

\label{sec:boruvka}

\paragraph{} Algorytm Borůvki jest przykładem algorytmu zachłannego znajdującego minimalne drzewo rozpinające. Algorytm został opracowany przez 
Czeskiego matematyka Otakara Borůvke w 1926 roku. Miał pomóc zoptymalizować zasięg sieci elektrycznej w Czechach. 

\subsection{Działanie}
Niech $G = (V, E)$ będzie grafem ważonym, w którym wszystkie krawędzie mają różne wagi.
W pierwszym kroku algorytm Borůvki wybiera dla wszystkich wierzchołków $ v \in V$ najlżejszą krawędź $e \in E$,
która jest z nimi incydentna i dołącza  wierzchołki wraz z wybranymi krawędziami do lasu $F$. W następnym kroku 
tworzymy graf $G'$ będący kopią $G$, z tą różnicą, że wszystkie spójne składowe z $F$ zostały ze sobą ``połączone'' w pojedyńcze wierzchołki zwane      
superwierzchołkami. Powyższe czynności powtarzamy zamieniając $G'$ na $G$, dopóki nie otrzymamy jednej spójnej składowej, będzie to nasze $MST$.

\subsection{Pseudokod}

\begin{algorithm}[H]
  \DontPrintSemicolon
  \SetAlgorithmName{Algorytm}{}\\
  \KwData{ $\mathcal{G}$ - Graf, w którym szukamy MST }\\
  \KwResult{ $\mathcal{T}$, takie że $\mathcal{T}$ jest $MST$ $\mathcal{G}$ }\\
$\mathcal{T} \leftarrow \emptyset$  \\
$\mathcal{F} \leftarrow \emptyset$  \\
  \While{$|\mathcal{G}(V)| > 1$}
  {
     Do $\mathcal{F}$ dołącz wszystkie wierzchołki z $\mathcal{G}$ wraz z incedentnymi krwędziami o najniższej wadze\\
     W $\mathcal{F}$ wszystkie spójne składowe zamień na pojedyńcze wierzchołki\\
     $\mathcal{G'} \leftarrow \mathcal{G} \setminus F$ \\
     $\mathcal{G} \leftarrow \mathcal{G'}$
  }
  \caption{Algorytm Borůvki}
  \label{alg-boruvka}
\end{algorithm}

\subsection{Dowód poprawnośći}
W każdym kroku algorytmu budujemy rozwiązanie będące $MST$, poprzez dodanie do niego spójnych składowych z $F$ w postaci jednego wierzchołka, a z
$Cut\ Property$ wiemy, że te wierzchołki należą do $MST$

\subsection{Złożoność czasowa}
W pojedyńczym kroku algorytmu wykonujemy maksymalnie $m$ operacji, gdzie $m$ jest ilością krawędzi w $G$.
W każdym kroku algorytmu ilość wierzchołków zmniejsza się przynbajmniej o połowę.
W związku z tym złożoność algorymu Borůvki wynosi $O(m\log{}n)$, gdzie $n$ oznacza ilość wierzchołków w $G$.


\section{Drzewce}
\sectionauthor{Bartłomiej Betka}

\label{sec:drzewce}

Podobnie jak drzewa AVL oraz drzewa czerwono-czarne drzewce rozwiązują problem potencjalnego niezrównoważenia "zwykłego" BST.
Osiągają one jednak ten cel w zupełnie inny sposób.
Zamiast kontrolować wysokość drzewa i jego poddrzew drzewce opierają się na randomizacji, która, jak zaraz pokażemy, zapewnia asymptotycznie równie dobre oczekiwane własności. 

\begin{definition} 
Drzewiec to drzewo binarne, którego każdy węzeł posiada zarówno klucz, jak i priorytet.
Względem kluczy drzewiec jest drzewem przeszukiwań binarnych, a względem priorytetów jest on kopcem.
\end{definition}

\paragraph{Podstawowe operacje, które implementują drzewce:}
\begin{itemize}
    \item \textbf{Find} Wygląda dokładnie tak, jak w drzewie BST (względem kluczy).
    \item \textbf{Insert} Najpierw wstawiamy wartość (wraz z losowym kluczem) jak do drzewa BST, następnie rotacjami/zamianą z ojcem przywracamy porządek kopcowy.
    \item \textbf{Delete} Znajdujemy klucz, rotacjami spychamy go do liścia, następnie usuwamy ten liść (co nie zaburza ani porządku BST, ani kopca).
    \item Pomocniczo potrzebne są również operacje rotacji służące do przywracania porządku kopcowego.
\end{itemize}

Na końcu rozdziału znajduje się pseudokod, który pokazuje, jak można zaimplementować powyższe metody i pozwala przekonać się, że jest to znacznie łatwiejsze niż w przypadku analogicznych operacji na drzewach AVL czy drzewach czerwono-czarnych.

Teraz udowodnimy kilka własności drzewców.\footnote{Przy analizie drzewców zakładamy, że wszystkie priorytety są różne.}

\begin{theorem}
\label{unique treap}
 Dla każdego zbioru par $(klucz, priorytet)$ istnieje dokładnie jeden drzewiec.
\end{theorem}
\begin{proof}
Oczywiście para, w której znajduje się najwyższy priorytet, musi znajdować się w korzeniu drzewa.
W jego lewym poddrzewie znajdować się muszą wszystkie pary o mniejszych kluczach, a w prawym wszystkie pary o większych kluczach, które indukcyjnie tworzą drzewce według tej samej zasady.
\end{proof}

Z twierdzenia tego i jego dowodu wprost wynika, że drzewiec tworzy BST o takim kształcie, jakby kolejne klucze były dodawane w kolejności rosnących priorytetów.

\begin{theorem}
Oczekiwana wysokość drzewca jest $O(log\ n)$.
\end{theorem}
\begin{proof}
Aby udowodnić nasze twierdzenie pokażemy, że wartość oczekiwana głębokości dowolnego węzła (ilość jego przodków) jest $O(log\ n)$.

Oznaczmy $N_k$ - węzeł o kluczu $k$ i wyznaczmy zmienną losową:

$$
X_{ij} =
  \begin{cases} 
  1 &  N_j\text{ jest przodkiem }N_i\\
  0 & $wpp.$
  \end{cases}
$$

Bez straty ogólności załóżmy teraz, że klucze są kolejnymi dodatnimi liczbami naturalnymi ($[1,n]$) i spróbujmy obliczyć $P(X_{ij})$.

Rozważmy węzły o kluczach z $[i, j]$ (węzły z kluczami spoza tego zakresu nie mają znaczenia dla obliczania $P(X_{ij})$) oraz ich priorytety. Mamy trzy możliwe przypadki:
\begin{enumerate}
\item Węzeł $N_i$  ma najniższy priorytet.
W tej sytuacji $N_j$ nie może, oczywiście, być przodkiem $N_i$.
\item Element o kluczu $k$ różnym od $i$ i $j$ ma najniższy priorytet.
W tym przypadku $N_k$ znajdzie się w korzeniu drzewca zawierającego zarówno $N_i$, jak i $N_j$, a ponieważ $i < k< j \lor j < k < i$ to $N_i$ i $N_j$ trafią do różnych poddrzew $N_k$.
\item Element o kluczu $j$ ma najwyższy priorytet.
Tylko w tym wypadku $N_j$ jest przodkiem $N_i$, ponieważ między nimi nie ma żadnego węzła o niższym priorytecie, który rozdzieliłby je do osobnych poddrzew jak w przypadku powyżej.
\end{enumerate}

Interesuje nas zatem $(|j - i|)!$ spośród $(|j - i| + 1)!$ permutacji, czyli $P(X_{ij}) = \frac{1}{|j-i| + 1}$.

A stąd oczekiwana wysokość $N_i$ to:
$$
\mathbf{E}[\text{wysokość }N_i]=
\sum_{j=1, j\neq i}^{n} \frac{1}{|j-i| + 1}=
\overbrace{
\sum_{j=1}^{i-1}\frac{1}{i-j+1}+
\sum_{j=i+1}^{n}\frac{1}{j-i+1}
}^\text{szeregi harmoniczne bez jedynek}<
2ln\ n=
O(log\ n)
$$
\end{proof}

\newpage

\begin{theorem}
Oczekiwana ilość rotacji przy \texttt{insert}/\texttt{delete} < 2.
\end{theorem}
\begin{proof}
Nasz dowód rozpoczniemy od pomocniczej definicji:
\begin{definition}
\textbf{Lewym/Prawym Kręgosłupem} drzewa nazywamy ścieżkę od korzenia do węzła z najmniejszym/największym kluczem (złożoną wyłącznie z lewych/prawych krawędzi).
\end{definition}
Rozważmy teraz długość kręgosłupów dzieci dodawanego węzła w czasie operacji \texttt{insert}.
Nie trudno zauważyć, że początkowo oba są długości $0$ oraz że każda rotacja w lewo wydłuża prawy kręgosłup lewego dziecka o $1$.
Analogicznie każda rotacja w prawo wydłuża lewy kręgosłup prawego drzewa o $1$.
Wynika stąd, że ilość rotacji potrzebnych przy wstawianiu węzła jest równa długości lewego kręgosłupa prawego dziecka i prawego kręgosłupa lewego dziecka nowo wstawionego węzła (po zakończeniu procedury).

Spróbujmy teraz określić wartości oczekiwane tych parametrów.
Weźmy dwa różne węzły $x$ i $y$ należące do drzewca. Oznaczmy $i=y.key$ i $k=x.key$, i wyznaczmy zmienną losową:

$$
X_{ik} =
  \begin{cases}
  1 &  y \in$ prawego kręgosłupa lewego poddrzewa $x\\
  0 & $wpp.$
  \end{cases}
$$

Pokażemy teraz, że $X_{ik} = 1 \iff y.priority > x.priority \land y.key < x.key \land (\forall z\ y.key < z.key < x.key \implies y.priority < z.priority)$

Implikacja w prawo wynika wprost z definicji drzewca.
Rozważmy zatem implikację w drugą stronę.
Ponieważ $y.priority > x.priority \land y.key < x.key$ to $y$ należy do lewego poddrzewa $x$.
W takim razie gdyby $y$ nie należał do prawego kręgosłupa lewego poddrzewa $x$, to musiałby istnieć $z$ t.że $y.key < z.key < x.key$ i $y.priority > z.priority$, co daje sprzeczność.

Bez straty ogólności załóżmy, że klucze są kolejnymi dodatnimi liczbami naturalnymi ($[1,n]$) i spróbujmy obliczyć $P(X_{ik})$.

Rozważmy teraz węzły o kluczach z $[i, k]$. Jeśli chcemy, żeby $\forall z\in[i+1,k-1]\ x.priority < y.priority < z.priority$, to spośród wszystkich $(k-i+1)!$ permutacji interesują nas te, w których $y.priority$ i $x.priority$ są kolejno na dwóch pierwszych pozycjach. Takich permutacji jest  $(k-i-1)!$.
Stąd $P(X_{ik}) = \frac{(k -i - 1)!}{(k -i + 1)!} =  \frac{1}{(k-i+1)(k-i)}$.

Teraz policzmy:
$$
\mathbf{E}[\sum_{i=1}^{k}X_{ik}]=
\sum_{i=1}^{k}\mathbf{E}[X_{ik}]=
\sum_{i=1}^{k}\frac{1}{(k-i+1)(k-i)}=
\sum_{i=1}^{k}\frac{1}{i(i+1)}=
\sum_{i=1}^{k}(\frac{1}{i}-\frac{1}{i+1})=
1-\frac{1}{k}
$$

Po przeprowadzeniu analogicznego rozumowania dla lewego kręgosłupa prawego poddrzewa (klucze z $[k,n]$) otrzymujemy wartość oczekiwaną $1 - \frac{1}{n-k+1}$.

To razem daje nam oczekiwaną ilość rotacji równą $1 - \frac{1}{k} + 1 - \frac{1}{n-k+1} = 2 -  \frac{1}{k} - \frac{1}{n-k+1} < 2$.

Identyczna liczba rotacji dla \texttt{delete} wynika wprost z powyższego oraz Twierdzenia \ref{unique treap}.
\end{proof}

Powyższe twierdzenia pokazują, że wysokość drzewców, a tym samym operacje na nich, nie tylko mają dobrą złożoność asymptotyczną, ale też, że (oczekiwany) narzut związany z rotacjami jest ograniczony przez niewielką stałą.

\newpage

\begin{algorithm}
  \DontPrintSemicolon
  \SetAlgorithmName{Algorytm}{}

  \KwData{ $root$ }

  $new\_root \leftarrow root.right$\;
  $root.right \leftarrow new\_root.left$\;
  $new\_root.left \leftarrow root$\;
  \Return $new\_root$\;
  
  \caption{\texttt{rotateLeft (rotacja w prawo jest analogiczna)}}
  \label{treap-rotate-left}
\end{algorithm}

\begin{algorithm}
  \DontPrintSemicolon
  \SetAlgorithmName{Algorytm}{}

  \KwData{ $root$, $value$ }
  \If{$root = null$}
  {
    \Return $Node(value, random())$\;
  }
  \ElseIf{$root.key = value$}
  {
    \Return $root$\;
  }
  \ElseIf{$root.key > value$}
  {
    $root.left \leftarrow insert(root.left, value)$\;
    \If{$root.left.priority < root.priority$}
    {
      $root \leftarrow rotateLeft(root)$\;
    }
  }
  \ElseIf{$root.key < value$}
  {
    $root.right \leftarrow insert(root.right, value)$\;
    \If{$root.right.priority < root.priority$}
    {
      $root \leftarrow rotateRight(root)$\;
    }
  }
  
  \Return $root$\;
  \caption{\texttt{insert}}
  \label{treap-insert}
\end{algorithm}

\begin{algorithm}
  \DontPrintSemicolon
  \SetAlgorithmName{Algorytm}{}

  \KwData{ $root$, $value$ }
  
  \If{$root = null$}
    {
      \Return $null$\;
    }
  \ElseIf{$root.key > value$}
  {
     $root.left \leftarrow delete(root->left, value)$\;
  }
  \ElseIf{$root.key < value$}
  {
     $root.right \leftarrow delete(root->right, value)$\;
  }
  \ElseIf{$root.key = value$}
  {
  	\If{$root.left = null$}
  	{
  	  \Return $root.right$\;
  	}
  	\ElseIf{$root.right = null$}
  	{
  	  \Return $root.left$\;
  	}
  	\ElseIf{$root.left.priority < root.right.priority$}
  	{
  	  $root \leftarrow rotateLeft(root)$\;
  	  $root.left \leftarrow delete(root.left, value)$\;
  	}
  	\Else
  	{
  	  $root \leftarrow rotateRight(root)$\;
  	  $root.right \leftarrow delete(root.right, value)$\;
  	}
  }
  \Return $root$\;
  \caption{\texttt{delete}}
  \label{treap-delete}
\end{algorithm}


\chapter{W ogóle nie zaczęte}

\section{Złożoność obliczeniowa}
\sectionauthor{Jeszcze Nikt}

\label{sec:zlozonosc}

Todo, todo, todo...

\section{Model obliczeń}
\sectionauthor{Jeszcze Nikt}

\label{sec:modelobliczen}

Todo, todo, todo...

\section{Programowanie dynamiczne na drzewach}
\sectionauthor{Jeszcze Nikt}

\label{sec:dynamicznenadrzewach}

Todo, todo, todo...

\section{Drzewa Splay}
\sectionauthor{Jeszcze Nikt}

\label{sec:splay}

Todo, todo, todo...

\section{Zbiory rozłączne}
\sectionauthor{Jeszcze Nikt}

\label{sec:unionfind}

Todo, todo, todo...

\section{Drzewa przedziałowe}
\sectionauthor{Jeszcze Nikt}

\label{sec:przedzialowe}

Todo, todo, todo...

\section{Tablice hashujące}
\sectionauthor{Jeszcze Nikt}

\label{sec:hashowanie}

Todo, todo, todo...

\section{Wyszukiwanie wzorca z wykorzystaniem automatów skończonych}
\sectionauthor{Jeszcze Nikt}

\label{sec:wzorzecautomat}

Todo, todo, todo...

\section{Algorytm Knutha-Morrisa-Pratta}
\sectionauthor{Jeszcze Nikt}

\label{sec:kmp}

Todo, todo, todo...

\section{Algorytm Karpa-Rabina}
\sectionauthor{Jeszcze Nikt}

\label{sec:kr}

Todo, todo, todo...

\section{Drzewa czerwono-czarne}
\sectionauthor{Jeszcze Nikt}

\label{sec:czerwonoczarne}

Todo, todo, todo...

\section{Słownik statyczny}
\sectionauthor{Jeszcze Nikt}

\label{sec:statyczny}

Todo, todo, todo...

\section{Geometria obliczeniowa}
\sectionauthor{Jeszcze Nikt}

\label{sec:geometria}

Todo, todo, todo...

\section{Kopce Fibonacciego}
\sectionauthor{Jeszcze Nikt}

\label{sec:kopcefib}

Todo, todo, todo...

\section{Drzewo van Emde Boasa}
\sectionauthor{Jeszcze Nikt}

\label{sec:boss}

Todo, todo, todo...

\section{Sortowanie kubełkowe}
\sectionauthor{Jeszcze Nikt}

\label{sec:bucket}

Todo, todo, todo...

\section{Model drzew decyzyjnych}
\sectionauthor{Jeszcze Nikt}

\label{sec:decyzyjne}

Todo, todo, todo...

\section{Dolna granica znajdowania min-maksa}
\sectionauthor{Jeszcze Nikt}

\label{sec:minmax}

Todo, todo, todo...

\section{Dolne granice}
\sectionauthor{Jeszcze Nikt}

\label{sec:dolnegranice}

Todo, todo, todo...

\section{Optymalna kolejność mnożenia macierzy}
\sectionauthor{Jeszcze Nikt}

\label{sec:optymalnemnozenie}

Todo, todo, todo...

\section{Drzewa rozpinające drabiny}
\sectionauthor{Jeszcze Nikt}

\label{sec:drabiny}

Todo, todo, todo...

\section{Cykl Hamiltona}
\sectionauthor{Jeszcze Nikt}

\label{sec:hamilton}

Todo, todo, todo...

\section{Problem spełnialności formuł logicznych}
\sectionauthor{Jeszcze Nikt}

\label{sec:3sat}

Todo, todo, todo...

\section{Trójwymiarowe skojarzenia}
\sectionauthor{Jeszcze Nikt}

\label{sec:3skojarzenie}

Todo, todo, todo...

\section{Stokrotki}
\sectionauthor{Jeszcze Nikt}

\label{sec:stokrotki}

Todo, todo, todo...

\section{Otoczka wypukła}
\sectionauthor{Jeszcze Nikt}

\label{sec:otoczka}

Todo, todo, todo...

\section{Najdłuższy wspólny podciąg}
\sectionauthor{Jeszcze Nikt}

\label{sec:lcs}

Todo, todo, todo...

\section{Algorytm Karatsuby}
\sectionauthor{Jeszcze Nikt}

\label{sec:karatsuba}

Todo, todo, todo...

\section{Algorytm Prima}
\sectionauthor{Jeszcze Nikt}

\label{sec:prim}

Todo, todo, todo...

\section{Algorytm Kruskala}
\sectionauthor{Jeszcze Nikt}

\label{sec:kruskal}

Todo, todo, todo...

\section{Statystyki pozycyjne}
\sectionauthor{Jeszcze Nikt}

\label{sec:statystykipozycyjne}

Todo, todo, todo...

\section{Algorytm magicznych piątek}
\sectionauthor{Jeszcze Nikt}

\label{sec:magicznepiatki}

Todo, todo, todo...

\section{Drzewa AVL}
\sectionauthor{Jeszcze Nikt}

\label{sec:avl}

Todo, todo, todo...

\section{Izomorfizm drzew}
\sectionauthor{Jeszcze Nikt}

\label{sec:izomorfizm}

Todo, todo, todo...

%% Dodatki

\begin{appendices}

\pagestyle{empty}
\input{porownanie.tex}

\chapter{Zadania na programowanie dynamiczne i algorytmy zachłanne}

Na egzaminie z Algorytmów i Struktur Danych, na drugiej części egzaminu lubią pojawiać się zadania na programowanie dynamiczne oraz algorytmy zachłanne.
Z doświadczenia wiem, że najlepszą metodą na nauczenie się rozwiązywania tego typu zadań jest zrobienie dużej ilości zadań tego rodzaju.
W niniejszym dodatku umieszczam mały zbiór zadań w nadziei, że pomoże on Czytelnikowi lepiej radzić sobie z zadaniami tego typu.

\section*{Optymalne mnożenie macierzy}

O mnożeniu macierzy mowa była już w rozdziale \ref{sec:strassen}.
Przedstawiliśmy tam algorytm mnożenia macierzy, który dla macierzy o rozmiarze $n \times m$ oraz $m \times p$ działa w czasie $\Theta(n \cdot m \cdot p)$.
Ponadto powinniśmy pamiętać z Algebry, że mnożenie macierzy jest przemienne, czyli, że $(A \cdot B) \cdot C = A \cdot (B \cdot C)$.
Lub mówiąc inaczej - ustawienie nawiasów w ciągu iloczynów macierzy nie wpływa na wynik mnożenia.
Jednakże ustawienie nawiasów wpływa na coś innego - na sumaryczną ilość operacji jaką wykona algorytm mnożący wszystkie macierze.
Dla przykładu niech macierz $A$ będzie rozmiaru $5 \times 10$, macierz $B$ rozmiaru $10 \times 20$, a $C$ rozmiaru $20 \times 35$.
Jeśli algorytm pierw przemnoży macierz $A$ przez $B$ a następnie wynik tego mnożenia przez $C$, wykona $4500$ operacji.
Jeśli natomiast najpierw przemnoży $B$ przez $C$ a następnie $A$ przez wynik poprzedniego mnożenia, wtedy wykona $8750$ operacji.
Mając dany ciąg $a_1, a_2, \cdots a_{n+1}$ należy wypisać takie nawiasowanie macierzy $A_1, A_2, \cdots A_n$ przy którym algorytm wykona jak najmniejszą ilość operacji.
Zakładamy przy tym, ż macierz $A_i$ jest rozmiaru $a_i \times a_{i+1}$.

\section*{Wydawanie reszty}

Przyjmijmy, że mamy następujące nominały: $50$ groszy, $25$ groszy, $10$ groszy, $5$ groszy oraz $1$ grosz.
Chcemy wydać resztę używając tych nominałów.
Możemy to zrobić na różne sposoby.
Dla przykładu, gdy chcemy wydać $11$ groszy, możemy to zrobić używając $10$-groszówki i $1$-groszówki, dwóch $5$-groszówek oraz $1$-groszówki, jednej $5$ groszówki oraz sześciu $1$-groszówek lub jedenastu $1$-groszówek.
Ułóż algorytm, który dla zadanej reszty liczy ile jest sposobów na wydanie reszty za pomocą wspomnianych nominałów.

\section*{Łamanie patyka}

Dany jest długi patyk, który należy połamać w określonych miejscach.
Jak powszechnie wiadomo (?) im dłuższy patyk, tym  trudniej go złamać.
Firma profesjonalnie zajmująca się łamaniem patyków, każe sobie płacić proporcjonalnie dużo od długości patyka.
W zależności od tego w jakiej kolejności każemy firmie łamać patyk, może zależeć to ile za takie łamanie zapłacimy.
Dla przykładu powiedzmy, że mamy patyk o długości $10$ m i musimy go złamać w $2$, $4$ i $7$ metrach.
Jeśli firma wpierw złamie go w $2$-gim metrze, następnie w $4$-tym a na końcu w $7$-mym to zapłacimy za tą niewątpliwą przyjemność $10 + 8 + 6 = 24$.
Jeśli natomiast najpierw poprosilibyśmy, aby firma złamała go w $4$-tym metrze, a następnie w $2$-gim i $7$-mym to zapłacilibyśmy $10 + 4 + 6 = 20$.
Mając daną długość patyka oraz miejsca przełamań policz najtańszą kolejność łamania patyka.

\section*{Wąskie liczby}

Liczbę w systemie $k$-tkowym nazywamy wąską, jeśli każde dwie sąsiednie cyfry różnią się od siebie nie więcej niż o jeden.
Dla danych liczb $k$ oraz $n$ policz ile jest $n$-cyfrowych liczb w systemie $k$-tkowym, które są wąskie.

\section*{Mądre słonie}

Niektórym wydaje się, że im większy jest słoń, tym jest mądrzejszy.
Aby zaprzeczyć temu twierdzeniu chcesz znaleźć w podanej bazie danych słoni jak najdłuższy ciąg w którym waga słoni rośnie, a IQ maleje.

\section*{Odwracanie naleśników}

Dany jest talerz z naleśnikami.
Naleśniki mają różne średnice i leżą na talerzu na wspólnym stosie.
Chcemy posortować naleśniki, przy czym jedyną dopuszczalną operacją jest włożenie łopatki pod wybrany naleśnik a następnie odwrócenie go wraz z wszystkimi naleśnikami leżącymi wyżej.
Należy znaleźć ciąg operacji odwracania naleśników, który doprowadzi do posortowania naleśników według średnic.
Możesz założyć, że naleśniki są z serem.

\section*{Upośledzony Gustaw}

Gustaw potrafi liczyć.
Teraz uczy się jak liczby zapisuje się na papierze.
Ponieważ Gustaw jest bardzo dobrym uczniem, zna już cyfry $1$, $2$, $3$ oraz $4$.
Nie zorientował się on jeszcze, że $4$ to jest inna cyfra niż $1$, więc myśli, że ``$4$'' to inny sposób aby zapisać ``$1$''.
Pomimo tego wymyślił bardzo ``ciekawą'' grę.
Mając daną liczbę $n$ zastanawia się ile zna liczb, których cyfry sumują się do $n$.
Jeśli na przykład $n$ jest równe $2$ to Gustaw zna pięć takich liczb: $11$, $14$, $41$ oraz $2$.
Opracuj algorytm, który dla zadanaj liczby $n$ policzy ile liczb zna Gustaw, które sumują się do $n$.

\section*{Gra w klasy}

W bogatej dzielnicy Sosnowca dzieci znalazły sobie nową zabawą.
Na chodniku malują szachownicę o rozmiarach $n \times n$.
Następnie na każde pole szachownicy kładą różną ilość groszy.
Kolejnie jedno dziecko staje w dolnym lewym rogu szachownicy i zbiera wszystkie monety leżące na tym polu.
Następnie wybiera pomiędzy dwoma kierunkami: prawo lub góra i skacze na wybrane przez siebie pole
(o ile jest w stanie na nie doskoczyć; powiedzmy, że jest w każdym momencie skoczyć na pole nie dalsze niż o $k$).
Pole to musi zawierać więcej groszy niż pole na którym stał uprzednio.
Z tego pola zbiera wszystkie monety i zabawę kontynuuje dopóty ma możliwe ruchy.
Opracuj algorytm, który mając daną szachownicę obliczy ile dany gracz jest w stanie zebrać groszy.

\section*{Pokrycie przedziałowe}

Dane są przedziały liczbowe postaci $(l_i, r_i)$.
Należy wybrać jak najmniejszą liczbę przedziałów tak, aby ich suma zawierała przedział $(0, M)$.

\section*{Problem chińskich pałeczków}

Dlaczego łyżkę kojarzy się z jesienią?
Bo je się nią.
Ale nie w Chinach.
Tam je się pałeczkami.
Profesor L. wymyślił właśnie nowy system jedzenia w którym używa się $3$ pałeczek.
Jednej normalnej pary oraz jednego długiego patyka, którym zjada duże kawałki dziabiąc nim.
Oczywiście dobrze by było, aby dwie małe pałeczki były o zbliżonym rozmiarze, natomiast rozmiar największej pałeczki nie ma znaczenia.
Formalnie - jeśli $A \leq B \leq C$ to rozmiar pałeczek to niedobroć zestawu definiujemy jako $(B-A)^2$.
Profesor L. organizuje właśnie imprezę na którą przyjdzie $K$ gości.
Chce on zaprezentować swoim gościom swój nowy system.
Planuje teraz wybrać ze swojej kolekcji $N$ pałeczek $K$ zestawów, tak aby zminimalizować sumę niedobroci zestawów.

\section*{Konstruowanie BST}

Mamy zadaną wysokość $H$ oraz liczbę elementów $N$.
Chcemy znaleźć taką permutację liczb $1, 2, \cdots N$, że wstawiając do początkowo pustego drzewa BST otrzymamy drzewo o wysokości nie większej niż $H$.
Jeśli jest więcej takich permutacji to chcemy znaleźć leksykograficznie najmniejszą.

\section*{Optymalne BST}

Mamy zadany zbiór par $\{(e_i, p_i)\}$.
Chcemy z nich utworzyć drzewo w ten sposób, aby tworzyło ono drzewo BST po wartościach $e_i$.
Ponadto dla każdego drzewa definiujemy jego wagę jako $\sum p_i \cdot d_i$ gdzie $d_i$ to odległość $i$-tego wierzchołka od korzenia.
Należy znaleźć drzewo BST złożone z zadanych elementów o minimalnej wadze.

\section*{Willy Wonka}

Willy Wonka chce rozdać trójce dzieci cukierki.
Cukierków jest $N$ ($N \leq 30$) i każdy z nich ma swoją wagę w gramach ($w_i \leq W \leq 20$).
Willy chce rozdać wszystkie cukierki i chce aby dzieciaki dostały mniej więcej po równo - tj. chce zminimalizować różnicę w wadze między dzieciakiem, który dostał najcięższy worek cukierków a dzieciakiem, który dostał worek najlżejszy.

\section*{Ustawianie domina}

Chcemy ustawić szereg domina złożony z $N$ kostek domina.
Powiedzmy, że nasz szereg wygląda tak: ``DD\_\_DxDDD\_D''.
Chcemy teraz wstawić kolejną kostkę w pole oznaczone jako ``x''.
Ponieważ mylić się jest rzeczą ludzką - prawdopodobieństwo, że kostka przwróci się w lewo wynosi $p_l$ a w prawo $p_r$.
Zakładamy, że $0 < p_l + p_r \leq 0.5$.
Gdy kostka się przewróci - przewraca blok kostek znajdujący się obok niego.
W przykładzie - jeśli kostka którą stawiamy przewróci się w lewo - przewróci dodatkowo jedną kostkę.
Jeśli w prawo - przewróci dodatkowo aż trzy kostki.
Kostki te będziemy musieli postawić na nowo.
Ile wynosi oczekiwana ilość kostek, które musimy postawić w optymalnej strategii ustawienia?

\section*{Hazard}

Z sześciu symboli rozlosowywane są $3$ (z powtórzeniami).
Stawiamy $a \leq L$ złotych na jeden symbol (gdzie $L$ to dostępny limit na zakład).
Jeśli zostanie on wylosowany to odzyskujemy swoje $a$ złotych plus dodatkowo zyskujemy $a$ złotych za każde wystąpienie naszego symbolu wśród $3$ wylosowanych.
Jacek opracował Strategię Na Pewno Wygrywającą.
Wygląda ona w ten sposób, że Jacek stawia w pierwszym zakładzie $1$ złoty.
Jeśli przegra to w następnym zakładzie stawia dwa razy więcej niż w zakładzie poprzednim.
Łatwo można zauważyć, że gdy któryś zakład wygramy - wychodzimy na plus.
Wtedy zaczynamy naszą strategię od nowa - stawiając ponownie złotówkę na zakład.
Problemem jest limit na zakład.
Gdy Jacek go osiągnie, zaczyna strategię od nowa - mając nadzieję, że jeszcze się odkuje.
Jakie jest prawdopodobieństwo, że między $K$-tym zakładem, a $M$-tym chociaż raz będziemy na plusie?

\section*{Złote monety}

W Bajtocji istnieje dziwny system monetarny.
Każda moneta ma wybity nominał, który jest liczbą naturalną.
Każdą monetę o nominale $A$ można w Banku Bajtockim rozmienić na monety $\lfloor A / 2 \rfloor$, $\lfloor A / 3 \rfloor$ oraz $\lfloor A / 4 \rfloor$.
Ponadto monety można zamieniać na polskie złotówki przy kursie jeden do jednego (nie można dokonywać zamian w przeciwną stronę).
Mamy jedną monetę o nominale $A$.
Ile złotówek możemy otrzymać za nią?

\section*{Deski i gwoździe}

Dany jest zbiór przedziałów $\{(s_i, k_i)\}$.
Pytamy się o najmnijeszy zbiór punktów $\{p_j\}$ taki, że dla każdego $i$ istenieje takie $j$, że $p_j \in (s_i, k_i)$.

\section*{Sumowanie}

Dany jest zbiór liczb naturalnych $\{a_i\}$, który należy zsumować.
Koszt zsumowania liczb $a$ oraz $b$ wynosi $a + b$.
W jakiej kolejności należy sumować elementy, aby zminimalizować koszt sumowania?

\section*{Rzeźbiarz}

Do pracowni rzeźbiarskiej właśnie dotarła marmurowa płyta o rozmiarach $H \times W$.
Płytę tą można ciąć na dwa prostokąty wzdłóż jednego z jego boków.
W pracowni potrzebne są określone rozmiary płyt $\{ (h_i, w_i) \}$.
Jeden rozmiar można wykorzystać wielokrotnie.
Pracownia chce teraz tak pociąć dostarczoną płytę, aby zminimalizować ilość ``odpadków''.

\section*{Dzikie kodowanie}

Każdej literze przyporządkowujemy dodatnią liczbę całkowitą, określającą jej pozycję w alfabecie.
Słowa kodujemy zapisując każdą literę za pomocą przypisanej jej liczby.
Na przykład słowo ``BEAN'' zakodowalibyśmy jako ``25114''.
Problem jest z dekodowaniem słowa.
Dla przykładu - ``25114'' można zdekodować na $6$ sposobów.
Dla danej liczby należy policzyć na ile sposobów mozna ją zdekodować.

\section*{Transport przez rzekę}

Mamy dany statek, który może na raz przetransportować $n$ samochodów z jednego brzegu rzeki na drugi.
Transport trwa $k$ jednostek czasu w jedna stronę, po którym statek musi powrócić na pierwszy brzeg.
Mamy danych $m$ samochodów oraz $s_i$ - czas przyjazdu każdego samochodu nad rzekę.
Pytamy się o to w jaki sposób należy transportować samochody, aby ostatni samochód przekroczył rzekę jak najwcześniej.
Ile conajmniej kursów musi zrobić statek, aby osiągnąć taki czas?

\section*{Gra na grafie}

Dany mamy graf pełny z pętelkami.
Każda krawędź ma jeden z trzech kolorów.
Na grafie znajdują się $3$ pionki.
Możemy przesunąć pionek z wierzchołka $u$ na wierzchołek $v$ jeśli krawędź między tymi wierzchołkami jest tego samego koloru co krawędź łączącą pozycje dwóch pozostałych pionków.
Pytamy się o minimalną liczbę ruchów jakie trzeba wykonać, aby przesunąć wszystkie pionki na jedno pole.

\section*{Sklep z kwiatami}

Mamy dany zbiór kwiatów oraz rząd donniczek.
Musimy umieścić kwiaty w donniczkach zachowując następujące zasady:
\begin{itemize}
 \item donniczek nie można przestawiać
 \item jeden rodzaj kwiatu może wystąpić w wielu donniczkach
 \item kwiaty muszą być ustawione w kolejności alfabetycznej (narcyzy muszą znaleźć się na lewo od róż)
\end{itemize}
Nie wszystkie kwiaty tak samo ładnie wyglądają w różnych donniczkach.
Bardziej formalnie - mamy zadaną macierz $\{m_{i,j}\}$ określającą ``ładność'' wyglądu $i$-tego gatunku kwiatów w $j$-tej donniczce.
Pytamy się o najładniejsze ustawienie kwiatów w donniczkach.

TODO: Dodać zadania z średnie.pdf

\end{appendices}

\end{document}
