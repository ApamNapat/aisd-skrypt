\section{Szybka Transformata Fouriera}
\sectionauthor{Wiktor Garbarek}

%\newthoerem{remark_small}{Mała uwaga}
\newtheorem{observation_small}{Malutka obserwacja}
\newtheorem{observation_bigger}{Nieco większa obserwacja}


\label{sec:fft}

Zacznijmy od problemu mnożenia wielomianów. Mamy dwa wielomiany stopnia co najwyżej, dla późniejszej wygody w zapisie, $n-1$, powiedzmy $A(x) = \sum_{i=0}^{n-1} a_i\cdot x^i$ i analogicznie $B(x)$, gdzie $a_i$ oraz $b_i$ są rzeczywiste.
Szukamy zatem wielomianu $$C(x) = \sum_{i=0}^{2n-2} c_ix^i$$ gdzie $$c_i = \sum_{j=0} a_jb_{i-j} = \sum_{p + q = i} a_pb_q $$ Oba te zapisy $c_i$ są równoważne, jeśli założymy, że $ p,q \geq 0$.
Na pierwszy rzut oka możemy zauważyć, że istnieje algorytm obliczający $C(x)$ w czasie $\Theta(n^2)$. Czy da się lepiej? Owszem, bo inaczej nie byłoby tego rozdziału.\\

\begin{definition}
Takie działanie na wektorach $a,b\in \mathbb{R}^n$ matematycy nazywają splotem.
$$c = a * b \Leftrightarrow c_i = \sum_{j=1}^{n}a_jb_{i-j}$$
\end{definition}

Zacznijmy od paru obserwacji
\begin{theorem} (Lagrange)
    Każdy wielomian stopnia $n-1$ o współczynnikach rzeczywistych da się przedstawić jednoznacznie jako $n$-elementowa lista par $(x_i, W(x_i))$, gdzie ciąg $x_i$ jest dowolnym ciągiem parami różnych liczb rzeczywistych.
\end{theorem}
\begin{proof}
    Prosty dowód pozostawiamy czytelnikowi.
\end{proof}

\begin{observation_bigger}
    Ustalmy jakiś ciąg $\{x_i\}_{i=0}^{2n-1}$.
    Jeśli mamy wielomiany $A(x)$ oraz $B(x)$ w powyższej postaci obliczonej dla wspomnianego ciągu, to "łatwo" (tj. w czasie liniowym) możemy znaleźć ich iloczyn w tej postaci.
\end{observation_bigger}
\begin{proof}
    Wystarczy powiedzieć, że $C(x)$ to ciąg $(x_i, A(x_i)B(x_i))$. \\
    Bardzo ważną uwagą jest to, że $\{x_i\}$ jest długości co najmniej $2n$, bowiem wielomian wynikowy może mieć stopień co najwyżej $n - 1 + n - 1 = 2n - 2$, więc potrzebujemy przede wszystkim co najmniej $2n$ punktów, by wyrazić wielomian C (na mocy twierdzenia Lagrange'a).
\end{proof}

Super! W takim razie, jedyne co pozostaje nam, to znaleźć algorytm, który zamieni wielomian w postaci listy współczynników na próbki naszego wielomianu w $2n$ punktach, obliczymy prosto iloczyn tych wielomianów i jeszcze musimy teraz wrócić do domu obliczając współczynniki wielomianu interpolacyjnego.
No i tutaj jest cały problem, bo trywialny algorytm obliczy każdą tą zamianę postaci w czasie $\Theta(n^2)$ (Mamy $2n$ argumentów, dla każdego z nich obliczamy wartość wielomianu w czasie $\Theta(n)$)
Spróbujmy zastosować tutaj metodę divide and conquer. Najpierw jednak zauważmy ciekawą zależność.

\begin{observation_small}
Dla każdego wielomianu zachodzi następująca równość $$A(x) = A^e(x^2) + xA^o(x^2)$$ gdzie $$A^e(x) = \sum_{i=0}^{n/2 - 1}a_{2i}x^i$$ oraz $$A^o(x) = \sum_{i=0}^{n/2 - 1} a_{2i+1}x^i$$
Indeks 'e' oznacza \textit{even} (elementy na parzystych indeksach), a 'o' oznacza \textit{odd}.
\end{observation_small}
Niektórzy mogą pomyśleć, że to już koniec. Prosto bierzemy sobie wielomian $A(x)$, jakiś dowolny zbiór $X$, obliczamy rekurencyjnie wartość dla wielomianów $A^e$ i $A^o$ przy wykorzystaniu zbioru $X' = \{x^2 : x \in X\}$.
Nic bardziej mylnego, zauważmy bowiem, że gdy będziemy łączyć wyniki, to tak naprawdę za każdym razem wykonamy $\Theta(|X|)$ kroków, a przy dowolnie wybranym zbiorze, ten zbiór $X'$ wcale nie musi (i co najbardziej prawdopodobne, nie będzie) zmniejszać swojego rozmiaru\footnote{Nota bene taki algorytm dalej ma złożoność $\Theta(n^2)$}. \\
Ale jednak jest nadzieja! Możemy go wybrać dowolnie, więc wystarczy znaleźć $X$ taki, że $|X'| = \frac{|X|}{2}$. Brzmi niewykonalnie? Jednak wcale nie jest. Popatrzmy sobie na pierwiastki jedynki na płaszczyźnie zespolonej. Spójrzmy na przykład dla pierwiastków ósmego stopnia.
Gdy weźmiemy dowolną liczbę z tego zbioru i podniesiemy ją do kwadratu, to wtedy jej moduł się nie zmienia, a jedyne co robimy to podwajamy kąt między jej promieniem wodzącym, a dodatnią półosią OX.
W takim razie zamiast rozpatrywać kąty $[0, 45, 90, 135, 180, 225, 270, 315]$ będziemy rozpatrywać kąty $[0, 90, 180, 270, 360, 450, 540, 630]$.
Jednakże obrót o 450 stopni, to tak naprawdę obrót o 90, 540 to tak naprawdę 180 stopni itd.
Czyli widzimy, że podnosząc do kwadratu wszystkie elementy w tym zbiorze zmniejszyliśmy jego liczność do połowy. I znaleźliśmy też w ten sposób pierwiastki czwartego stopnia z jedynki.

Reasumując, obliczamy w takim razie $$\{W(e^{\frac{2k\pi i}{2n}}) = \sum_{j=0}^{n-1}a_je^{\frac{2k\pi ij}{2n}} | k = 0,...,2n-1\}$$
Żeby jednak nasze indeksowanie ładnie wyglądało, to możemy pomyśleć, że tak naprawdę $a_n, a_{n+1}, ..., a_{2n-1}$ są zerami, a wtedy możemy zapisać to przekształcenie w ten sposób.
Mając wektor $\vec{a} = [a_0, ..., a_{N-1}]$, szukamy wektora $\vec{y} = [y_0, ..., y_{N-1}]$ takiego, że
$$y_k = \sum_{j=0}^{N-1}a_je^{\frac{2k\pi ij}{N}}$$ Oto jest właśnie... no właśnie, żeby być ścisłym, coś w rodzaju odwrotnej dyskretnej transformaty Fouriera (IDFT (ang. \textit{Inverse Discrete Fourier Transform}),
a algorytm jej obliczania nazywa się, a jakżeby inaczej, IFFT (ang. \textit{Inverse Fast Fourier Transform})),
jednak tutaj każdy element powinien zostać przemnożony przez $\frac{1}{N}$ by dostać IDFT, zaraz podamy ścisłe definicje.
\begin{definition}
    Dyskretną Transformatą Fouriera (DFT, a algorytm jej obliczania to FFT) nazywamy następującą funkcję $dft: \mathbb{C}^N \rightarrow \mathbb{C}^N$
    gdzie, jeśli $\vec{y} = dft(\vec{a})$ to $$y_k = \sum_{j=0}^{N-1}a_je^{-\frac{2k\pi ij}{N}}$$
\end{definition}

\begin{definition}
    Odwrotną Dyskretną Transformatą Fouriera (IDFT, a algorytm jej obliczania to IFFT) nazywamy następującą funkcję $idft: \mathbb{C}^N \rightarrow \mathbb{C}^N$
    gdzie, jeśli $\vec{y} = idft(\vec{a})$ to $$y_k = \frac{1}{N}\sum_{j=0}^{N-1}a_je^{\frac{2k\pi ij}{N}}$$
\end{definition}

\begin{observation_small}
    Zachodzą następujące równości: $\text{idft}(\text{dft}(\vec{x})) = \vec{x}$ oraz $\text{dft}(\text{idft}(\vec{x})) = \vec{x}$
\end{observation_small}
\begin{proof}
    Rozpisując dokładnie nasza teza sprowadza się do udowodnienia następującej równości $$a_k = \frac{1}{N}\sum_{j=0}^{N-1}\Big(\sum_{t=0}^{N-1} a_t e^{-\frac{2j\pi i t}{N}} \Big) e^{\frac{2k\pi i j}{N}}$$
    Włączając nasze pierwiastki z jedynki do wewnętrznej sumy, zamieniając ze sobą wskaźniki sumowania i upraszaczając wykładnik otrzymujemy
    $$a_k = \frac{1}{N}\sum_{t=0}^{N-1}\sum_{j=0}^{N-1} a_t e^{\frac{2\pi i j(k-t)}{N}}$$
    Możemy zauważyć, że dla każdego $t$ nasze $k-t$ jest ustalone. W takim razie dla każdy wyraz $a_t$ jest mnożony przez $\Omega_t = \sum_{m=0}^{N-1}\omega_m^{k-t}$, gdzie $\omega_m$ to $m$-ty pierwiastek stopnia N z jedynki.
    W takim razie, na mocy wzorów Newtona-Girarda'a oraz wzorów Viete'a dla wielomianu $z^N - 1 = 0$, $\Omega_t = N$ tylko wtedy, gdy $k-t = 0$ (bo wtedy po prostu sumujemy niezerowe liczby podniesione do zerowej potęgi, czyli mamy $N$ jedynek), oraz $\Omega_t = 0$ w przeciwnych przypadkach.
    (Dokładne rozpisanie tego byłoby żmudnym zajęciem, więc polecam rozdział \textit{Expressing power sums in terms of elementary symmetric polynomials} na stronie \url{https://en.wikipedia.org/wiki/Newton%27s_identities},
    tu możemy zauważyć, że wszystkie wyrazy $e_i$ według notacji w tym artykule są zerami na mocy wspomnianych wzorów Viete'a).
    Zauważmy też fakt, że rozumowanie też jest prawidłowe dla $k-t < 0$.
\end{proof}
Algorytm ten zawdzięczamy Cooley'owi oraz Tukey'owi (1965), jednakże idea ta była znana już 160 lat wcześniej Gaussowi.

Poruszmy jeszcze jedną ważną rzecz - przede wszystkim w powyższych rozważaniach niemo wymagaliśmy, by długość naszego wektora była potęgą dwójki.
Jeśli chodzi natomiast o problem mnożenia wielomianów, w żaden sposób nas to nie boli, ponieważ zawsze możemy dopełnić wektor jakąś ilością zer do najbliższej potęgi dwójki,
a skoro wydłużył on się co najwyżej dwukrotnie, to ta idea nie zmienia złożoności całego algorytmu.
Nie obchodzą nas też w żaden sposób wartości pośrednie (czytaj: zamiana wielomianu z wektora współczynników, na wektor punktów do interpolacji)
Jednak gdyby zależało nam na policzeniu powyższego działania (tj. nie mnożenia wielomianów, a już obliczenia DFT dla dowolnego wektora) pojawia się duży problem.
Przede wszystkim całą zabawę psuje liczba $\exp{\frac{*}{N}}$, której nie pozbędziemy się w żaden trywialny sposób.

Wracając jednak do algorytmu, zapiszemy jego uogólnioną wersję w pseudokodzie.\\
\begin{algorithm}[H]
  \DontPrintSemicolon
  \SetAlgorithmName{Algorytm (FFT)}{Cooley-Tukey}

  \KwData{\\ $a = [a_0, a_1, ... a_{N-1}] \in \mathbb{C}^N$ - wektor liczb zespolonych długości N \\ $X = [x_0, ..., x_{N-1}] \in \mathbb{C}^N$ spełniający równość $|\{x^2 | x \in X\}| = \frac{|X|}{2}$ dla każdego wywołania rekurencyjnego.}

  \KwResult{$y \in \mathbb{C}^n$, taki, że $y_k = \sum_{n=0}^{N-1}a_n \cdot x_k^n$}
  \\
  \If{$|X| = 1 \wedge |a| = 1$}{
    return [$a_0*x_0$]
    }

  $X' = [x^2$ for $x$ in $X]$\\
  $y^e = fft([a_0, a_2, ..., a_{N - 2}], X')$\\
  $y^o = fft([a_1, a_3, ..., a_{N - 1}], X')$ \\

  return [$y_i^e + x_i*y_i^o$ for $i \leftarrow 0$ to $|X| - 1$]

  \caption{Algorytm Cooleya-Tukeya}
  \label{alg-fft-cooley-tukey}
\end{algorithm}
Jego złożoność możemy wyrazić poprzez zależność $T(n, |X|) = 2*T(n/2, |X|/2) + O(|X|)$,
a więc upraszczając $T(n') = 2T(n'/2) + O(n')$, czyli złożoność czasowa naszego algorytmu to $O(n\log{n})$.
