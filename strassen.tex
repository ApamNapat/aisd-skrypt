\section{Algorytm Strassena}
\sectionauthor{Krzysztof Piecuch}

\label{sec:strassen}

Z mnożeniem macierzy mieliście już prawdopodobnie do czynienia na Algebrze.
\comment{Symbol mnożenia macierzy jest inny niż w rozdziale o liczbach Fibonacciego.}
Mając dane macierze $A$ (o rozmiarze $n \times m$) oraz $B$ (o rozmiarze $m \times p$) nad ciałem liczb rzeczywistych, chcemy policzyć ich iloczyn:
\[
 A \cdot B = C
\]
gdzie elementy macierzy $C$ (o rozmiarze $n \times p$) zadane są wzorem:
\[
 c_{i,j} = \sum_{r=1}^{m} a_{i,r} \cdot b_{r,j}
\]
Przykładowo:
\[
  \begin{bmatrix}
    1 & 0 & 2 \\
    0 & 3 & 1 \\
  \end{bmatrix}
\cdot
  \begin{bmatrix}
    3 & 1 \\
    2 & 1 \\
    1 & 0
  \end{bmatrix}
=
  \begin{bmatrix}
     (1 \cdot 3  +  0 \cdot 2  +  2 \cdot 1) & (1 \cdot 1   +   0 \cdot 1   +   2 \cdot 0) \\
     (0 \cdot 3  +  3 \cdot 2  +  1 \cdot 1) & (0 \cdot 1   +   3 \cdot 1   +   1 \cdot 0) \\
  \end{bmatrix}
=
  \begin{bmatrix}
    5 & 1 \\
    7 & 3 \\
  \end{bmatrix}
\]
Korzystając prosto z definicji możemy napisać następujący algorytm mnożenia dwóch macierzy:

\begin{algorithm}[H]
  \DontPrintSemicolon
  \SetAlgorithmName{Algorytm}{}
  
  \KwData{ $A$, $B$ - macierze o rozmiarach $n \times m$ oraz $m \times p$ }
  
  \KwResult{ $C = A \cdot B$ }
  
  \For{$i \leftarrow 1$ to $n$}
  {
     \For{$j \leftarrow 1$ to $p$}
     {
	$C[i][j] \leftarrow 0$\;
	\For{$r \leftarrow 1$ to $m$}
	{
	  $C[i][j] \leftarrow C[i][j] + A[i][r] \cdot B[r][j]$\;
	}
     }
  }
  
  \caption{Naiwny algorytm mnożenia macierzy}
  \label{alg-mnozenie-macierzy}
\end{algorithm}
Powyższy algorytm działa w czasie $\Theta(n \cdot p \cdot m)$ ($\Theta(n^3)$ dla macierzy kwadratowych o boku $n$).
Korzystając ze sprytnej sztuczki, jesteśmy w stanie zmniejszyć złożoność naszego algorytmu.

Zacznijmy od założenia, że rozmiar macierzy jest postaci $2^k \times 2^k$.
Jeśli macierze nie są takiej postaci, to możemy uzupełnić brakujące wiersze i kolumny zerami.
Następnie podzielmy macierze na cztery równe części:
\[
  A = 
  \begin{bmatrix}
    A_{1,1} & A_{1,2} \\
    A_{2,1} & A_{2,2} \\
  \end{bmatrix} \enspace  
  B = 
  \begin{bmatrix}
    B_{1,1} & B_{1,2} \\
    B_{2,1} & B_{2,2} \\
  \end{bmatrix} \enspace
  C = 
  \begin{bmatrix}
    C_{1,1} & C_{1,2} \\
    C_{2,1} & C_{2,2} \\
  \end{bmatrix}
\]
Każda z części jest rozmiaru $2^{k-1} \times 2^{k-1}$.
Ponadto wzór na każdą część macierzy $C$ wyraża się wzorem:
\[
 C_{i,j} = A_{i,1} \cdot B_{1,j} + A_{i,2} \cdot B_{2,j}
\]

Czy wzór ten umożliwia nam ułożenie efektywnego algorytmu mnożenia macierzy?
Nie.
W algorytmie mamy do policzenia $4$ podmacierze macierzy $C$.
Każda podmacierz wymaga $2$ mnożeń oraz jednego dodawania.
Dodawanie macierzy możemy w prosty sposób zrealizować w czasie $\Theta(n^2)$.
Mnożenie podmacierzy możemy wykonać rekurencyjnie.
Taki algorytm będzie działał w czasie $T(n) = 8\cdot T(n/2) + \Theta(n^2)$ czyli $\Theta(n^3)$.
Osiągnęliśmy tą samą złożoność czasową jak w przypadku algorytmu liczącego iloczyn wprost z definicji.

Algorytm Strassena osiąga lepszą złożoność asymptotyczną przez pozbycie się jednego z mnożeń.
Algorytm ten liczy następujące macierze:
\begin{align*}
 M_1 &= (A_{1,1} + A_{2,2}) \cdot (B_{1,1} + B_{2,2}) \\
 M_2 &= (A_{2,1} + A_{2,2}) \cdot B_{1,1} \\
 M_3 &= A_{1,1} \cdot (B_{1,2} - B_{2,2}) \\
 M_4 &= A_{2,2} \cdot (B_{2,1} - B_{1,1}) \\
 M_5 &= (A_{1,1} + A_{1,2}) \cdot B_{2,2} \\
 M_6 &= (A_{2,1} - A_{1,1}) \cdot (B_{1,1} + B_{1,2}) \\
 M_7 &= (A_{1,2} - A_{2,2}) \cdot (B_{2,1} + B_{2,2})
\end{align*}
Do policzenia każdej z tych macierzy potrzebujemy jednego mnożenia i co najwyżej dwóch dodawań/odejmowań.
Podmacierze macierzy $C$ możemy policzyć teraz w następujący sposób:
\begin{align*}
 C_{1,1} &= M_1 + M_4 - M_5 + M_7 \\
 C_{1,2} &= M_3 + M_5 \\
 C_{2,1} &= M_2 + M_4 \\
 C_{2,2} &= M_1 - M_2 + M_3 + M_6
\end{align*}
Wykonując proste przekształcenia arytmetyczne, możemy dowieść poprawności powyższych równań.

Używając powyższych wzorów, możemy skonstruować algorytm rekurencyjny.
Będzie on dzielić macierze $A$ oraz $B$ o rozmiarze $2^k \times 2^k$ na cztery równe części.
Następnie policzy on macierze $M_i$.
Tam, gdzie będzie musiał dodawać/odejmować użyje on algorytmu działającego w czasie $\Theta(n^2)$.
Tam, gdzie będzie musiał mnożyć - wywoła się on rekurencyjnie.
Na podstawie macierzy $M_i$ policzy macierz $C$.
Ponieważ wykona dokładnie $7$ mnożeń oraz stałą ilość dodawań, jego złożoność obliczeniowa będzie wyrażała się wzorem rekurencyjnym $T(n) = 7\cdot T(n/2) + \Theta(n^2)$.
Korzystając z twierdzenia o rekurencji uniwersalnej otrzymujemy złożoność $\Theta(n^{\log_2 7})$ czyli około $\Theta(n^{2.81})$.