\section{Sortowanie bitoniczne}

\label{sec:bitoniczne}

W tym rozdziale przedstawimy algorytm sortowania bitonicznego.
Jest to algorytm działający w czasie $\Theta(n \log^2 n)$ czyli gorszym niż inne, znane algorytmy sortujące, takie jak sortowanie przez scalanie albo sortowanie szybkie.
Zaletą sortowania bitonicznego jest to, że może zostać uruchomiony równolegle na wielu procesorach.
Ponadto, dzięki temu, że algorytm zawsze porównuje te same elementy bez względu na dane wejściowe, istnieje prosta implementacja fizyczna tego algorytmu (np. w postaci tzw. sieci sortujących).
Algorytm będzie zakładał, że rozmiar danych $n$ jest potęgą dwójki.
Gdyby tak nie było, moglibyśmy wypełnić tablicę do posortowania nieskończonościami, tak aby uzupełnić rozmiar danych do potęgi dwójki.
Rozmiar danych zwiększyłby się wtedy nie więcej niż dwukrotnie, zatem złożoność asymptotyczna pozostałaby taka sama.

Sortowanie bitoniczne posługuje się tzw. ciągami bitonicznymi, które sobie teraz zdefiniujemy.
\begin{definition}
 \textbf{Ciągiem bitonicznym właściwym} nazywamy każdy ciąg powstały przez sklejenie ciągu niemalejącego z ciągiem nierosnącym.
\end{definition}
Dla przykładu ciąg $2$, $2$, $5$, $100$, $72$, $69$, $42$, $17$ jest ciągiem bitonicznym właściwym, gdyż powstał przez sklejenie ciągu niemalejącego $2$, $2$, $5$ oraz ciągu nierosnącego $100$, $72$, $69$, $42$, $17$.
Ciąg $1$, $0$, $1$, $0$ nie jest ciągiem bitonicznym właściwym, gdyż nie istnieją taki ciąg niemalejący i taki ciąg nierosnący, które w wyniku sklejenia dałyby podany ciąg.
\begin{definition}
 \textbf{Ciągiem bitonicznym} nazywamy każdy ciąg powstały przez rotację cykliczną ciągu bitonicznego właściwego.
\end{definition}
Ciąg $69$, $42$, $17$, $2$, $2$, $5$, $100$, $72$ jest ciągiem bitonicznym, gdyż powstał przez rotację cykliczną ciągu bitonicznego właściwego $2$, $2$, $5$, $100$, $72$, $69$, $42$, $17$.

Istnieje prosty algorytm sprawdzający, czy ciąg jest bitoniczny.
Należy znaleźć element największy oraz najmniejszy.
Następnie od elementu najmniejszego należy przejść cyklicznie w prawo (tj. w sytuacji gdy natrafimy na koniec ciągu, wracamy do początku) aż napotkamy element największy.
Elementy, które przeszliśmy w ten sposób powinny tworzyć ciąg niemalejący.
Analogicznie, idziemy od elementy największego cyklicznie w prawo aż do elementu najmniejszego.
Elementy, które odwiedziliśmy powinny tworzyć ciąg nierosnący.
W sytuacji w której mamy wiele elementów najmniejszych (największych), powinny one ze sobą sąsiadować (w sensie cyklicznym) i nie ma znaczenia, który z nich wybierzemy.
Dla przykładu w ciągu  $69$, $42$, $17$, $2$, $2$, $5$, $100$, $72$ idąc od elementu najmniejszego do największego tworzymy ciąg $2$, $2$, $5$, $100$ i jest to ciąg niemalejący.
Idąc od elementu największego do najmniejszego otrzymujemy ciąg $100$, $72$, $69$, $42$, $17$, $2$ i jest to ciąg nierosnący.

Jedyną procedurą, która będzie przestawiała elementy w tablicy, będzie procedura \texttt{bitonic\_compare} (Algorytm \ref{bitonic-compare}).
Jako dane wejściowe otrzymuje ona tablicę \texttt{A}, wielkość tablicy \texttt{n} oraz wartość logiczną \texttt{up}, która określa, czy ciąg będzie sortowany rosnąco czy malejąco.
\begin{algorithm}[h]
  \DontPrintSemicolon
  \SetAlgorithmName{Algorytm}{}

  \KwData{ A[0..n-1], up }
  
  \For{$i \leftarrow 0$ to $n-1$}
  {
    \If{$(A[i] > A[i+n/2]) = up$}
    {
      $A[i] \leftrightarrow A[i+n/2]$\;
    }
  }
  \caption{Procedura \texttt{bitonic\_compare}}
  \label{bitonic-compare}
\end{algorithm}
Procedura dzieli zadaną na wejściu tablicę na dwie równe części.
Następnie porównuje pierwszy element z pierwszej części z pierwszym elementem z drugiej części.
Jeśli te elementy nie znajdują się w pożądanym porządku, to je przestawia.
Następnie powtarza tą czynność z kolejnymi elementami.

Dla przykładu, jeśli procedurę uruchomimy z tablicą \texttt{A = [2, 8, 7, 1, 4, 3, 5, 6]}, wartością \texttt{n = 8} oraz \texttt{up = true}, w wyniku otrzymamy tablicę \texttt{A = [2, 3, 5, 1, 4, 8, 7, 6]}.
W pierwszym kroku wartość $2$ zostanie porównana z wartością $4$.
Ponieważ chcemy otrzymać porządek rosnący (wartość zmiennej \texttt{up} jest ustawiona na \texttt{true}), to zostawiamy tą parę w spokoju.
W następnym kroku porównujemy wartość $8$ z wartością $3$.
Te wartości są w złym porządku, dlatego algorytm zamienia je miejscami.
Dalej porównujemy $7$ z $5$ i zamieniamy je miejscami i w końcu porównujemy $1$ z $6$ i te wartości zostawiamy w spokoju, gdyż są w dobrym porządku.

Procedura \texttt{bitonic\_compare} ma bardzo ważną własność, którą teraz udowodnimy.

\begin{theorem}
 Jeżeli elementy tablicy \texttt{A[0..n-1]} tworzą ciąg bitoniczny, to po zakończeniu procedury \texttt{bitonic\_compare} elementy tablicy \texttt{A[0..n/2-1]} oraz tablicy \texttt{A[n/2..n-1]} będą tworzyły ciągi bitoniczne.
 Ponadto jeśli wartość zmiennej \texttt{up} jest ustawiona na \texttt{true} to każdy element tablicy \texttt{A[0..n/2-1]} będzie niewiększy od każdego elementu tablicy \texttt{A[n/2..n-1]}.
 W przeciwnym przypadku będzie niemniejszy.
 \label{bitonic-theorem}
\end{theorem}

Weźmy dla przykładu ciąg bitoniczny $69$, $42$, $17$, $2$, $2$, $5$, $100$, $72$.
Po przejściu procedury \texttt{bitonic\_compare} z ustawioną zmienną \texttt{up} na wartość \texttt{true} otrzymamy ciąg $2$, $5$, $17$, $2$, $69$, $42$, $100$, $72$.
Ciągi $2$, $5$, $17$, $2$ oraz $69$, $42$, $100$, $72$ są ciągami bitonicznymi.
Ponadto każdy element ciągu $2$, $5$, $17$, $2$ jest niewiększy od każdego elementu ciągu $69$, $42$, $100$, $72$.

Przejdźmy do dowodu powyższego twierdzenia.
Przyda nam się do tego poniższy lemat:

\begin{lemma}[zasada zero-jeden]
 Twierdzenie \ref{bitonic-theorem}. jest prawdziwe dla dowolnych tablic wtedy i tylko wtedy, gdy jest prawdziwe dla tablic zero-jedynkowych.
 \label{zero-one-lemma}
\end{lemma}

\begin{proof}
 Jeśli twierdzenie jest prawdziwe dla każdej tablicy to w szczególności jest prawdziwe dla tablic złożonych z zer i jedynek.
 Dowód w drugą stronę jest dużo ciekawszy.
 
 Weźmy dowolną funkcję niemalejącą $f$.
 To znaczy funkcję $f: \mathbb{R} \rightarrow \mathbb{R}$ taką, że $\forall_{a, b \in \mathbb{R}} \enspace a \leq b \Rightarrow f(a) \leq f(b)$.
 Dla tablicy \texttt{T} przez $f(T)$ będziemy rozumieli tablicę powstałą przez zaaplikowanie funkcji $f$ do każdego elementu tablicy \texttt{T}.
 Niech \texttt{A} oznacza tablicę wejściową do procedury \texttt{bitonic\_compare} i niech \texttt{B} oznacza tablicę wyjściową.
 Udowodnimy, że karmiąc procedurę \texttt{bitonic\_compare} tablicą $f(A)$ otrzymamy tablicę $f(B)$.
 W kroku $i$-tym procedura rozważa przestawienie elementów $t_i$ oraz $t_{i+n/2}$.
 Jeśli $f(a_i) = f(a_{i+n/2})$ to nie ma znaczenia czy elementy zostaną przestawione.
 Z kolei jeśli $f(a_i) < f(a_{i+n/2})$ to $a_i < a_{i+n/2}$ zatem jeśli procedura przestawi elementy $f(a_i)$ oraz $f(a_{i+n/2})$ to również przestawi elementy $a_i$ oraz $a_{i+n/2}$.
 Analogicznie gdy $f(a_i) > f(a_{i+n/2})$.
 Zatem istotnie: dla każdej funkcji niemalejącej $f$, procedura \texttt{bitonic\_compare} otrzymując na wejściu tablicę $f(A)$ zwróci na wyjściu tablicę $f(B)$.
 
 Wróćmy do dowodu lematu.
 Dowód nie wprost.
 Załóżmy, że twierdzenie jest prawdziwe dla wszystkich tablic zero-jedynkowych i nie jest prawdziwe dla pewnej tablicy \texttt{T[0..n-1]}.
 Niech \texttt{S[0..n-1]} oznacza zawartość tablicy po zakończeniu procedury \texttt{bitonic\_compare}.
 Jeśli twierdzenie nie jest prawdziwe, oznacza to, że albo któraś z tablic \texttt{S[0..n/2-1]}, \texttt{S[n/2..n-1]} nie jest bitoniczna albo, że element pierwszej z nich jest większy od któregoś elementu z drugiej tablicy.
 Rozważmy dwa przypadki.
 
 Załóżmy, że tablica \texttt{S[0..n/2-1]} nie jest bitoniczna (przypadek kiedy druga z tablic nie jest bitoniczna, jest analogiczny).
 Załóżmy, że ciąg powstały przez przejście od najmniejszego elementu w tej tablicy do największego nie tworzy ciągu niemalejącego (przypadek gdy ciąg powstały przez przejście od największego elementu do najmniejszego nie tworzy ciągu nierosnącego jest analogiczny).
 Zatem istenieje w tablicy element \texttt{S[i]} większy od elementu \texttt{S[i+1]}.
 Rozważmy następują funkcję:
 \[   
  f(a) = 
     \begin{cases}
       1 &\quad\textsf{jeśli} \enspace a \leq S[i]\\
       0 &\quad\textsf{wpp.}
     \end{cases}
 \]
 Zwróćmy uwagę, że w takiej sytuacji twierdzenie nie byłoby prawdziwe dla tablicy $f(T)$, zatem dla tablicy zero-jedynkowej.
 Gdyż ponownie - element $f(S[i]) = 1$ byłby większy od elementu $f(S[i+1]) = 0$.
 
  \begin{center}
  \begin{tabular}{ccccccccr}
  1 & 2 & 3 & 5          & 4            & 6 & 7 & 1 & \texttt{S[]} \\
  0 & 0 & 0 & 1          & 0            & 1 & 1 & 0 & $f(S)$ \\
    &   &   & $i$ & $i+1$ &   &   &   &
  \end{tabular}
  \end{center}
 
 Drugi przypadek.
 Załóżmy, że zmienna \texttt{up} ustawiona jest na \texttt{true} (przypadek drugi jest analogiczny).
 Załóżmy, że element \texttt{S[i]} jest mniejszy od elementu \texttt{S[j]} gdzie $j < n/2$ oraz $i \geq n/2$.
 Rozważmy funkcję:
  \[   
  f(a) = 
     \begin{cases}
       1 &\quad\textsf{jeśli} \enspace a \leq S[j]\\
       0 &\quad\textsf{wpp.}
     \end{cases}
 \]
 Wtedy twierdzenie nie byłoby prawdziwe dla tablicy $f(T)$ (zero-jedynkowej).
 Ponownie - element $f(S[i]) = 0$ byłby mniejszy od elementu $f(S[j]) = 1$.
 
  \begin{center}
  \begin{tabular}{ccccccccr}
  1 & 2 & 100        & 2 & 102 & 99         & 103 & 107 & \texttt{S[]} \\
  0 & 0 & 1          & 0 & 1   & 0          & 1   & 1   & $f(S)$ \\
    &   & $j$ &   &     & $i$ &     &     &
  \end{tabular}
  \end{center}
 
 
\end{proof}

Do pełni szczęścia potrzebujemy udowodnić, że Twierdzenie \ref{bitonic-theorem} jest prawdziwe dla wszystkich ciągów zero-jedynkowych.

\begin{lemma}
 Twierdzenie \ref{bitonic-theorem}. jest prawdziwe dla wszystkich ciągów zero-jedynkowych.
 \label{zero-one-cases-lemma}
\end{lemma}

\begin{proof}
 Zakładać będziemy, że zmienna \texttt{up} jest ustawiona na \texttt{true} (dowód dla sytuacji przeciwnej jest analogiczny).
 \comment{Ten dowód jest nudny.}
 Istnieje sześć rodzai bitonicznych ciągów zero-jedynkowych : $0^n$, $0^k1^l$, $0^k1^l0^m$, $1^n$, $1^k0^l$, $1^k0^l1^m$ z czego trzy ostatnie są symetryczne do trzech pierwszych (więc zostaną pominięte w dowodzie).
 Rozważmy wszystkie interesujące nas przypadki:
 \begin{itemize}
  \item $0^n$.
   Po wykonaniu procedury \texttt{bitonic\_compare} otrzymamy $0^{n/2}$ oraz $0^{n/2}$.
   Oba ciągi są bitoniczne i każdy element z pierwszego ciągu jest niewiększy od każdego elementu z ciągu drugiego.
  \item $0^k1^l$ oraz $k < n/2$.
   Wtedy po wykonaniu procedury \texttt{bitonic\_compare} otrzymamy ciągi $0^k1^{l-n/2}$ oraz $1^{n/2}$.
   Oba ciągi są bitoniczne i każdy element z pierwszego ciągu jest niewiększy od każdego elementu z ciągu drugiego.
  \item $0^k1^l$ oraz $k > n/2$.
   Otrzymamy ciągi $0^{n/2}$ oraz $0^{k-n/2}1^l$.
   Znowu - oba są bitoniczne i każdy element z pierwszego jest niewiększy od każdego z drugiego.
  \item $0^k1^l0^m$ oraz $k > n/2$.
   Wtedy otrzymujemy ciągi $0^{n/2}$ oraz $0^{k-n/2}1^l0^m$.
   Spełniają one tezę twierdzenia.
  \item $0^k1^l0^m$ oraz $m > n/2$.
   Ciągi, które otrzymamy wyglądają tak: $0^{n/2}$ oraz $0^k1^l0^{m-n/2}$.
   Są to ciągi, które nas cieszą.
  \item $0^k1^l0^m$ oraz $l > n/2$.
   Dostaniemy wtedy ciągi $0^k1^{l-n/2}0^m$ oraz $1^{n/2}$.
   Są to ciągi, które spełniają naszą tezę.
  \item $0^k1^l0^m$ oraz $k, l, m < n/2$.
   Ciągi, które uzyskamy to $0^{n/2}$ oraz $1^{n/2-m}0^{n/2-l}$ $1^{n/2-k}$.
   Spełniają one naszą tezę.
 \end{itemize}
\end{proof}

Na mocy Lematów \ref{zero-one-lemma} i \ref{zero-one-cases-lemma} Twierdzenie \ref{bitonic-theorem} jest prawdziwe dla wszystkich tablic \texttt{T[0..n-1]}.
Mając tak piękne twierdzenie, możemy napisać prosty algorytm sortujący ciągi bitoniczne (Algorytm \ref{bitonic-merge}).

\begin{algorithm}[h]
  \DontPrintSemicolon
  \SetAlgorithmName{Algorytm}{}
  
  \KwData{ A - tablica bitoniczna, n, up }
  
  \KwResult{ A - tablica posortowana }
  
   \If{$n > 1$}
   {
     bitonic\_compare(A[$0$..$n-1$], $n$,   up)\;
     bitonic\_merge(A[$0$..$n/2-1$], $n/2$, up)\;
     bitonic\_merge(A[$n/2$..$n-1$], $n/2$, up)\;
   }
  \caption{Procedura \texttt{bitonic\_merge}}
  \label{bitonic-merge}
\end{algorithm}

Algorytm zaczyna od wywołania procedury \texttt{bitonic\_compare}.
Dzięki niej, wszystkie elementy mniejsze wrzucane są do pierwszej połowy tablicy, a elementy większe do drugiej połowy.
Ponadto \texttt{bitonic\_compare} gwarantuje, że obie podtablice pozostają bitoniczne (jakie to piękne!).
Możemy zatem wykonać całą procedurę ponownie na obu podtablicach rekurencyjnie.

Złożoność algorytmu wyraża się wzorem rekurencyjnym $T(n) = 2 \cdot T(n/2) + \Theta(n)$.
Rozwiązując rekurencję otrzymujemy, że złożoność algorytmu to $\Theta(n \log n)$.

Mamy algorytm sortujący ciągi bitoniczne.
Jak uzyskać algorytm sortujący dowolne ciągi?
Zrealizujemy to w najprostszy możliwy sposób!
Posortujemy (rekurencyjnie) pierwszą połowę tablicy rosnąco, drugą połowę tablicy malejąco (dlatego potrzebna nam była zmienna \texttt{up}!) i uzyskamy w ten sposób ciąg bitoniczny.
Teraz wystarczy już uruchomić algorytm sortujący ciągi bitoniczne i voilà.

\begin{algorithm}[h]
  \DontPrintSemicolon
  \SetAlgorithmName{Algorytm}{}
  
  \KwData{ A, n, up }
  
  \KwResult{ A - tablica posortowana }
  
   \If{$n > 1$}
   {
     bitonic\_sort(A[$0$..$n/2-1$], $n/2$, true)\;
     bitonic\_sort(A[$n/2$..$n-1$], $n/2$, false)\;
     bitonic\_merge(A[$0$..$n-1$], $n$, up)\;
   }
  \caption{Procedura \texttt{bitonic\_sort}}
  \label{bitonic-sort}
\end{algorithm}

Złożoność algorytmu wyraża się wzorem rekurencyjnym $T(n) = 2 \cdot T(n/2) + \Theta(n \log n)$.
Rozwiązaniem tej rekurencji jest $\Theta(n \log^2 n)$.