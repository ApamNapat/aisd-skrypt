\section{B-drzewa}
\sectionauthor{Karol Rodziński}

\label{sec:bdrzewa}

Specyfika pamięci dyskowej polega na tym, że czas dostępu do niej jest znacznie (o kilka rzędów wielkości) dłuższy niż do pamięci wewnętrznej (RAM).
Drzewa BST, nawet w wersjach zrównoważonych, nie nadają się do przechowywania danych na dysku.
Kiedy dane z dysku są odczytywane, zapisywane, musimy posługiwać się porcjami danych zwanymi blokami (stronami).
Możemy uogólnić, iż na czas dostępu do danych składa się: czas wyszukania, opóźnienie rotacyjne, czas transferu. 

W takim przypadku lepiej jest sięgąć po większe ilości danych na raz aniżeli ciągle przemieszczać się w celu odczytu mniejszych porcji danych, co prowadzi do wniosku, że dane należy zorganizować w taki sposób, żeby odczytów było możliwie jak najmniej.

Odpowiedzią na ten problem mają być B-drzewa, które są bardzo często wykorzystywane do implementowania systemów plików czy też systemów bazodanowych.

\comment{Rudolf Bayer i Ed McCreight stworzyli strukturę B-drzew podczas pracy w Boeing Research Labs w 1971 r. Do dzisiaj jednak nie są w stanie powiedzieć, co oznacza B w nazwie tej struktury: Boeing, balanced, broad, bushy, Bayer etc. McCreight powiedział, że "im więcej myślisz o tym, co oznacza B w tej nazwie, tym bardziej rozumiesz tę strukturę."}

\begin{definition}
Formalnie B-Drzewo rzędu $K$ ($K \geq 3$) to drzewiasta struktura danych, która spełnia następujace warunki:
\begin{itemize}
\item Korzeń jest liściem albo ma od 2 do $K$ synów
\item Wszystkie liście znajdują się na tym samym poziomie 
\item Każdy węzeł, który nie jest liściem ani korzeniem, ma od $\lceil\frac{K}{2}\rceil$ do $K$ synów.
Węzeł mający $L$ synów zawiera $L-1$ kluczy
\item Każdy liść zawiera od $\lceil\frac{K}{2}\rceil - 1$ do $K-1$ kluczy
\end{itemize}
\end{definition}
Zgodnie z podanymi warunkami B-drzewo jest zawsze przynajmniej do połowy wypełnione, ma kilka poziomów i jest całkowicie zrównoważone.
\begin{figure}[!h]
\centering
\textbf{Przykład B-drzewa}\par
\medskip
\begin{tikzpicture}
    \tikzstyle{bplus}=[rectangle split, rectangle split horizontal,rectangle split ignore empty parts,draw, fill=white]
    \tikzstyle{every node}=[bplus]
    \tikzstyle{level 1}=[sibling distance=60mm]
    \tikzstyle{level 2}=[sibling distance=20mm]

    \node {31 \nodepart{two}  $\times$} [->]
      child {node {14 \nodepart{two} $\times$}
        child {node {8 \nodepart{two} 12}}
        child {node {19 \nodepart{two} $\times$}}
      }
      child {node {48 \nodepart{two} 70}
        child {node {35 \nodepart{two} 37}}
        child {node {55 \nodepart{two} $\times$}}
        child {node {75 \nodepart{two} 99}}
      };
\end{tikzpicture}
\bigskip
\captionsetup{justification=centering}
\caption{Jest tak zwane 2-3 Drzewo tzn. węzły w tym drzewie mają 1-2 klucze i 2-3 dzieci}
\end{figure}
\newpage
Drugi warunek pozwala nam na posiadanie stosunkowo niewielkiej wysokości naszego B-drzewa.
Zajmijmy się najgorszym i najlepszym przypadkiem.
\begin{itemize}
\item Najgorszy przypadek
\end{itemize}
Sytuacja najbardziej pesymistyczną zdarzy się wtedy, gdy nasze B-drzewo będzie miało najmniejszą dozwoloną liczbę wskaźników w węzłach wewnętrznych
\begin{proof}
Niech $q$ będzie minimalną liczbą dzieci węzła wewnętrznego drzewa.
Z podanych wcześniej warunków, wiemy, że jest to: $\lceil\frac{K}{2}\rceil$.
W takim przypadku zauważmy, iż:
\begin{align*}
\texttt{1 poziom $\mapsto$ 1 klucz} \\
\texttt{2 poziom $\mapsto$ $2(q-1)$ kluczy} \\
\texttt{3 poziom $\mapsto$ $2q(q-1)$ kluczy} \\
\texttt{...} \\
\texttt{h poziom $\mapsto$ ${2q}^{h}-2(q-1)$} \\
\end{align*}
Zsumujmy wyrażenie powyżej, aby otrzymać liczbę kluczy w B-drzewie, otrzymamy:
\begin{align*}
1 + (\sum_{i=0}^{h-2}2{q}^{i})(q-1) = 1 + 2(q-1)(\frac{{q}^{h-1}}{q-1}) = -1 + 2{q}^{h-1}
\end{align*}
Zatem związek między liczbą kluczy a wysokością to:
\begin{align*}
n \geq -1 + 2{q}^{h-1}
\end{align*}
Po zlogarytmowaniu otrzymamy:
\begin{align*}
h \leq \log_{q}{\frac{n+1}{2}} + 1
\end{align*}
\end{proof}
\begin{itemize}
\item Najlepszy przypadek
\end{itemize}
\begin{proof}
Niech $h$ oznacza wysokość B-drzewa. Niech $n$ oznacza liczbę węzłów w drzewie i niech $m$ oznacza maksymalną liczbę dzieci węzła. Z warunków wymienionych wcześniej wiemy, że taki węzeł może mieć maksymalnie $m-1$ kluczy. Można pokazać poprzez indukcję, że takie B-drzewo o wysokości $h$, że każdy jego węzeł ma maksymalną liczbę dzieci, posiada ${m}^{h}-1$ kluczy, stąd w najlepszym przypadku wysokość B-drzewa to $\lceil\log_{m}{(n+1)}\rceil - 1$
\end{proof}
\newpage
TO-DO: metody dodawania, usuwania etc.