\section{Model afinicznych drzew decyzyjnych}
\sectionauthor{Krzysztof Piecuch}

\label{sec:elementuniqueness}

Zdefiniujmy następujący problem (ang. element uniqueness).
Mając daną tablicę \texttt{T[0..n-1]} liczb rzeczywistych, odpowiedzieć na pytanie czy istnieją w tablicy dwa elementy, które są sobie równe.
Pierwsze rozwiązanie jakie przychodzi wielu ludziom do głowy, to posortować tablicę \texttt{T} a następnie sprawdzić sąsiednie elementy.
Algorytm ten rozwiązuje nasz problem w czasie $\Theta(n \log n)$.
Pytanie - czy da się szybciej?
W niniejszym rozdziale udowodnimy, że w modelu afinicznych drzew decyzyjnych problemu nie da się rozwiązać lepiej.

W modelu afinicznych drzew decyzyjnych, w każdym zapytaniu możemy wybrać sobie $n+1$ liczb: $c$, $a_0$, $a_1$, $\dots$, $a_{n-1}$, a następnie zapytać czy
\[
 c + \sum_{i=0}^{n-1} a_i t_i \geq 0
\]
gdzie $t_i$ to elementy tablicy \texttt{T}.
Gdybyśmy użyli terminologii algebraicznej, to powiedzielibyśmy, że $t$ jest punktem w przestrzeni $\mathbb{R}^n$, lewa strona powyższej nierówności to przekształcenie afiniczne,
a zbiór wszystkich punktów z $\mathbb{R}^n$, które spełniają tą nierówność to półprzestrzeń afiniczna.
Jeśli na Algebrze nie wyrobiliście sobie jeszcze intuicji, to w $\mathbb{R}^2$ półprzestrzeń afiniczną otrzymujemy przez narysowanie dowolnej prostej i wzięcie wszystkich elementów z jednej ze stron.
Podobnie w $\mathbb{R}^3$ półprzestrzeń afiniczną otrzymujemy poprzez narysowanie dowolnej płaszczyzny, a następnie wzięcia wszystkich elementów z jednej ze stron.
W wyższych wymiarach wygląda to analogicznie.
\tikzboxwithcaption{tikz/uniquenesstree.tikz}{
\label{uniqueness-tree-example}
Przykład afinicznego drzewa decyzyjnego.
W wierzchołkach wewnętrznych mamy zapytanie $(c^j, a_i^j)$.
W zależności od tego czy $c^k + \sum_{i=0}^{n-1} a^k_i t_i \geq 0$ czy też nie, idziemy odpowiednio w lewo lub w prawo.
Liście zawierają odpowiedź naszego algorytmu.
}

Algorytm używający tego typu porównań można zapisać za pomocą drzewa (rys. \ref{uniqueness-tree-example}).
Zaczynamy z korzenia tego drzewa.
W każdym wierzchołku wewnętrznym zadajemy zapytanie.
W zależności od tego czy odpowiedź na pytanie była pozytywna czy negatywna, idziemy w drzewie w lewo lub w prawo.
Gdy dojdziemy do liścia w drzewie otrzymujemy nasze rozwiązanie (tak lub nie).
Takie drzewo będziemy nazywać afinicznym drzewem decyzyjnym.

Aby było nam łatwiej zdefiniować główny lemat naszego rozdziału, zdefiniujemy sobie dwa pojęcia.
\begin{definition}
 Mówimy, że punkt $t \in \mathbb{R}^n$ \textbf{osiąga} liść $l$ w afinicznym drzewie decyzyjnym, jeśli algorytm uruchomiony dla punktu $t$ dochodzi do liścia $l$.
\end{definition}
\begin{definition}
 Mówimy, że podzbiór $C \subseteq \mathbb{R}^n$ jest \textbf{zbiorem wypukłym}, jeśli dla dowolnych punktów $u, v \in C$ oraz dowolnej liczby rzeczywistej $0 \leq \alpha \leq 1$ punkt $\alpha \cdot u + (1 - \alpha) v$ także należy do $C$.
\end{definition}
Pierwsza z definicji pozwala nam mówić o elementach, które trafiają do tego samego liścia, a druga to sformalizowane pojęcie wypukłości znane z liceum.
Uzbrojeni w nowe definicje, możemy przejść do obiecanego lematu:

\begin{lemma}
 Zbiór punktów osiągających liść $l$ w afinicznym drzewie decyzyjnym, jest zbiorem wypukłym.
 \label{uniqueness-lemma}
\end{lemma}

\begin{proof}
 Weźmy dowolne afiniczne drzewo decyzyjne i wybierzmy w nim dowolny liść $l$.
 Dobrze wiemy, że istnieje dokładnie jedna ścieżka prosta z korzenia do tego liścia.
 Weźmy dowolny wierzchołek wewnętrzny $w$ na tej ścieżce i dowolne punkty $u$ oraz $v$, które osiągają liść $l$.
 W końcu weźmy dowolną liczbę rzeczywistą $0 \leq \alpha \leq 1$.
 Załóżmy ponadto, że ścieżka z korzenia do liścia $l$ w wierzchołku $w$ skręca w lewo (zatem zapytanie zadane w wierzchołku $w$ punkty $u$ oraz $v$ otrzymały odpowiedź twierdzącą).
 Przypadek przeciwny jest analogiczny.
 Wiemy zatem, że
\[
 c^w + \sum_{i=0}^{n-1} a_i^w u_i \geq 0
\]
 oraz, że
\[
 c^w + \sum_{i=0}^{n-1} a_i^w v_i \geq 0
\]
 Ponieważ $\alpha \geq 0$ możemy przemnożyć pierwsze równanie przez $\alpha$:
\[
 \alpha c^w + \sum_{i=0}^{n-1} a_i^w \alpha u_i \geq 0
\]
 a ponieważ $1 - \alpha \geq 0$ możemy drugie równanie przemnożyć przez $1 - \alpha$:
\[
 (1 - \alpha) c^w + \sum_{i=0}^{n-1} a_i^w (1 - \alpha) v_i \geq 0
\]
Teraz sumując oba równania otrzymujemy:
\[
 c^w + \sum_{i=0}^{n-1} a_i^w (\alpha u_i + (1 - \alpha) v_i) \geq 0
\]
Zatem punkt $\alpha \cdot u + (1 - \alpha) v$ również w wierzchołku $w$ skręci w tą samą stronę co punkty $u$ oraz $v$.
Ponieważ wybraliśmy dowolny wierzchołek $w$, to punkt $\alpha \cdot u + (1 - \alpha) v$ osiągnie liść $l$.
Stąd zbiór wszystkich punktów, które osiągają liść $l$ w afinicznym drzewie decyzyjnym, jest zbiorem wypukłym.
\end{proof}

Wyobraźmy sobie, że na płaszczyznę $\mathbb{R}^2$ kładziemy półpłaszczyzny.
Wtedy przecięcie dowolnej liczby półpłaszczyzn jest zbiorem wypukłym.
Podobnie jeśli w przestrzeni $\mathbb{R}^3$ wyznaczymy sobie półprzestrzenie, to ich przecięcie będzie tworzyło zbiór wypukły.
Lemat \ref{uniqueness-lemma} mówi, że tak samo się dzieje w każdej przestrzeni $\mathbb{R}^n$.

Ten lemat za chwilę okaże się dla nas kluczowy, gdyż za jego pomocą udowodnimy, że jeśli afiniczne drzewo decyzyjne poprawnie rozwiązuje problem element uniqueness to musi posiadać co najmniej $n!$ liści.
Oznacza to, że wysokość takiego drzewa musi wynosić co najmniej $O(n \log n)$.

\begin{lemma}
 \label{permutation-lemma}
 Niech $\{r_0, r_1, \ldots, r_{n-1}\}$ będzie $n$ elementowym zbiorem liczb rzeczywistych i niech $(a_0, a_1, \ldots, a_{n-1})$ oraz $(b_0, b_1, \ldots, b_{n-1})$ będą dwoma różnymi permutacjami liczb z tego zbioru.
 W każdym afinicznym drzewie decyzyjnym poprawnie rozwiązującym problem element uniqueness, punkty $a$ oraz $b$ osiągają różne liście w drzewie.
\end{lemma}

\begin{proof}
 Dowód nie wprost.
 Załóżmy, że $a$ oraz $b$ osiągają ten sam liść w drzewie.
 Liść ten musi odpowiadać przecząco na zadany problem, gdyż ani $a$ ani $b$ nie zawierają dwóch tych samych elementów.
 Ponieważ $a$ oraz $b$ składają się z tych samych liczb rzeczywistych i różnie się od siebie permutacją, to muszą istnieć takie indeksy $i$ oraz $j$, że $a_i > a_j$ oraz $b_i < b_j$.
 Weźmy następującą wartość $\alpha$:
 \[
  \alpha = \frac{b_j - b_i}{(a_i - a_j) + (b_j - b_i)}
 \]
 Wykonując proste przekształcenia arytmetyczne, możemy przekonać się, że $0 < \alpha < 1$.
 Oznacza to, na mocy lematu \ref{uniqueness-lemma}, że punkt $\alpha a + (1 - \alpha)b$ również osiąga ten sam liść co punkty $a$ i $b$.
 Ponieważ jednak zachodzi:
 \[
  \alpha a_i + (1 - \alpha) b_i = \alpha a_j + (1 - \alpha) b_j
 \]
 (o czym można się przekonać wykonując proste przekształcenia arytmetyczne), odpowiedź algorytmu dla tego punktu powinna być twierdząca.
 Zatem afiniczne drzewo decyzyjne dla tego punktu zwraca złą odpowiedź.
 Sprzeczność z założeniem, że drzewo rozwiązywało problem poprawnie.
\end{proof}

Weźmy dowolny $n$ elementowy zbiór liczb rzeczywistych.
Na mocy Lematu \ref{permutation-lemma} każda permutacja tych liczb musi osiągać inny liść w afinicznym drzewie decyzyjnym poprawnie rozwiązującym problem element uniqueness.
Oznacza to, że liczba liści w takim drzewie musi wynosić przynajmniej $n!$.
Zatem wysokość takiego drzewa musi wynosić co najmniej $\Omega(n \log n)$.