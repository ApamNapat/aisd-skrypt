\section{Drzewce}
\sectionauthor{Bartłomiej Betka}

\label{sec:drzewce}

Podobnie jak drzewa AVL oraz drzewa czerwono-czarne drzewce rozwiązują problem potencjalnego niezrównoważenia "zwykłego" BST.
Osiągają one jednak ten cel w zupełnie inny sposób.
Zamiast kontrolować wysokość drzewa i jego poddrzew drzewce opierają się na randomizacji, która, jak zaraz pokażemy, zapewnia asymptotycznie równie dobre oczekiwane własności. 

\begin{definition} 
Drzewiec to drzewo binarne, którego każdy węzeł posiada zarówno klucz jak i priorytet.
Względem kluczy drzewiec jest drzewem przeszukiwań binarnych a względem priorytetów jest on kopcem.
\end{definition}

\paragraph{Podstawowe operacje, które implementują drzewce:}
\begin{itemize}
    \item \textbf{Find} Wygląda dokładnie tak, jak w drzewie BST (względem kluczy).
    \item \textbf{Insert} Najpierw wstawiamy wartość (wraz z losowym kluczem) jak do drzewa BST, następnie rotacjami/zamianą z ojcem przywracamy porządek kopcowy.
    \item \textbf{Delete} Znajdujemy klucz, rotacjami spychamy go do liścia, następnie usuwamy ten liść (co nie zaburza ani porządku BST ani kopca).
    \item Pomocniczo potrzebne są również operacje rotacji służące do przywracania porządku kopcowego.
\end{itemize}

Na końcu rozdziału znajduje się pseudo-kod, który pokazuje jak można zaimplementować powyższe metody i pozwala przekonać się, że jest to znacznie łatwiejsze niż w przypadku analogicznych operacji na drzewach AVL czy drzewach czerwono-czarnych.

Teraz udowodnimy kilka własności drzewców.\footnote{Przy analizie drzewców zakładamy, że wszystkie priorytety sa różne.}

\begin{theorem}
 Dla każdego zbioru par $(klucz, priorytet)$ istnieje dokładnie jeden drzewiec
 \label{xyz}
\end{theorem}
\begin{proof}
Oczywiście para, w której znajduje się najwyższy priorytet musi znajdować się w korzeniu drzewa.
W jego lewym poddrzewie znajdować się muszą wszystkie pary o mniejszych kluczach, a w prawym wszystkie pary o większych kluczach, które, indukcyjnie, tworzą drzewce według tej samej zasady.
\end{proof}

Z twierdzenia tego i jego dowodu wprost wynika, że drzewiec tworzy BST o takim kształcie, jakby kolejne klucze były dodawane w kolejności rosnących priorytetów.

\begin{theorem}
 Asymptotyczna wysokość drzewca wynosi $O(log\ n)$
 \label{xyz}
\end{theorem}
\begin{proof}

\end{proof}

\begin{theorem}
 Oczekiwana ilość rotacji przy insert/delete < 2
 \label{xyz}
\end{theorem}
\begin{proof}

\end{proof}



\begin{algorithm}
  \DontPrintSemicolon
  \SetAlgorithmName{Algorytm}{}

  \KwData{ root, value }
  
  \If{$root = null$}
    {
      \Return $null$\;
    }
  \ElseIf{$root.key > value$}
  {
     root.left $\leftarrow$ delete(root->left, value)\;
  }
  \ElseIf{$root.key < value$}
  {
     root.right $\leftarrow$ delete(root->right, value)\;
  }
  \ElseIf{$root.key = value$}
  {
  	\If{$root.left = null$}
  	{
  	  \Return $root.right$\;
  	}
  	\ElseIf{$root.right = null$}
  	{
  	  \Return $root.left$\;
  	}
  	\ElseIf{$root.left.priority < root.right.priority$}
  	{
  	  $root \leftarrow rotateLeft(root)$\;
  	  $root.left \leftarrow delete(root.left, value)$\;
  	}
  	\Else
  	{
  	  $root \leftarrow rotateRight(root)$\;
  	  $root.right \leftarrow delete(root.right, value)$\;
  	}
  }
  \Return $root$\;
  \caption{\texttt{delete}}
  \label{treap-delete}
\end{algorithm}

\begin{algorithm}
  \DontPrintSemicolon
  \SetAlgorithmName{Algorytm}{}

  \KwData{ root, value }
  \If{$root = null$}
  {
    \Return $Node(value, random())$\;
  }
  \ElseIf{$root.key = value$}
  {
    \Return $root$\;
  }
  \ElseIf{$root.key > value$}
  {
    $root.left \leftarrow insert(root.left, value)$\;
    \If{$root.left.priority < root.priority$}
    {
      $root \leftarrow rotateLeft(root)$\;
    }
  }
  \ElseIf{$root.key < value$}
  {
    $root.right \leftarrow insert(root.right, value)$\;
    \If{$root.right.priority < root.priority$}
    {
      $root \leftarrow rotateRight(root)$\;
    }
  }
  
  \Return root\;
  \caption{\texttt{insert}}
  \label{treap-insert}
\end{algorithm}

\begin{algorithm}
  \DontPrintSemicolon
  \SetAlgorithmName{Algorytm}{}

  \KwData{ root }

  $new\_root \leftarrow root.right$\;
  $root.right \leftarrow new\_root.left$\;
  $new\_root.left \leftarrow root$\;
  \Return new\_root\;
  
  \caption{\texttt{rotateLeft (rotacja w prawo jest analogiczna)}}
  \label{treap-rotate-left}
\end{algorithm}
