\chapter{Zadania na programowanie dynamiczne i algorytmy zachłanne}

Na egzaminie z Algorytmów i Struktur Danych, na drugiej części egzaminu lubią pojawiać się zadania na programowanie dynamiczne oraz algorytmy zachłanne.
Z doświadczenia wiem, że najlepszą metodą na nauczenie się rozwiązywania tego typu zadań jest zrobienie dużej ilości zadań tego rodzaju.
W niniejszym dodatku umieszczam mały zbiór zadań w nadziei, że pomoże on Czytelnikowi lepiej radzić sobie z zadaniami tego typu.

\section*{Optymalne mnożenie macierzy}

O mnożeniu macierzy mowa była już w rozdziale \ref{sec:strassen}.
Przedstawiliśmy tam algorytm mnożenia macierzy, który dla macierzy o rozmiarze $n \times m$ oraz $m \times p$ działa w czasie $\Theta(n \cdot m \cdot p)$.
Ponadto powinniśmy pamiętać z Algebry, że mnożenie macierzy jest przemienne, czyli, że $(A \cdot B) \cdot C = A \cdot (B \cdot C)$.
Lub mówiąc inaczej - ustawienie nawiasów w ciągu iloczynów macierzy nie wpływa na wynik mnożenia.
Jednakże ustawienie nawiasów wpływa na coś innego - na sumaryczną ilość operacji jaką wykona algorytm mnożący wszystkie macierze.
Dla przykładu niech macierz $A$ będzie rozmiaru $5 \times 10$, macierz $B$ rozmiaru $10 \times 20$, a $C$ rozmiaru $20 \times 35$.
Jeśli algorytm pierw przemnoży macierz $A$ przez $B$ a następnie wynik tego mnożenia przez $C$, wykona $4500$ operacji.
Jeśli natomiast najpierw przemnoży $B$ przez $C$ a następnie $A$ przez wynik poprzedniego mnożenia, wtedy wykona $8750$ operacji.
Mając dany ciąg $a_1, a_2, \cdots a_{n+1}$ należy wypisać takie nawiasowanie macierzy $A_1, A_2, \cdots A_n$ przy którym algorytm wykona jak najmniejszą ilość operacji.
Zakładamy przy tym, ż macierz $A_i$ jest rozmiaru $a_i \times a_{i+1}$.

\section*{Wydawanie reszty}

Przyjmijmy, że mamy następujące nominały: $50$ groszy, $25$ groszy, $10$ groszy, $5$ groszy oraz $1$ grosz.
Chcemy wydać resztę używając tych nominałów.
Możemy to zrobić na różne sposoby.
Dla przykładu, gdy chcemy wydać $11$ groszy, możemy to zrobić używając $10$-groszówki i $1$-groszówki, dwóch $5$-groszówek oraz $1$-groszówki, jednej $5$ groszówki oraz sześciu $1$-groszówek lub jedenastu $1$-groszówek.
Ułóż algorytm, który dla zadanej reszty liczy ile jest sposobów na wydanie reszty za pomocą wspomnianych nominałów.

\section*{Łamanie patyka}

Dany jest długi patyk, który należy połamać w określonych miejscach.
Jak powszechnie wiadomo (?) im dłuższy patyk, tym  trudniej go złamać.
Firma profesjonalnie zajmująca się łamaniem patyków, każe sobie płacić proporcjonalnie dużo od długości patyka.
W zależności od tego w jakiej kolejności każemy firmie łamać patyk, może zależeć to ile za takie łamanie zapłacimy.
Dla przykładu powiedzmy, że mamy patyk o długości $10$ m i musimy go złamać w $2$, $4$ i $7$ metrach.
Jeśli firma wpierw złamie go w $2$-gim metrze, następnie w $4$-tym a na końcu w $7$-mym to zapłacimy za tą niewątpliwą przyjemność $10 + 8 + 6 = 24$.
Jeśli natomiast najpierw poprosilibyśmy, aby firma złamała go w $4$-tym metrze, a następnie w $2$-gim i $7$-mym to zapłacilibyśmy $10 + 4 + 6 = 20$.
Mając daną długość patyka oraz miejsca przełamań policz najtańszą kolejność łamania patyka.

\section*{Wąskie liczby}

Liczbę w systemie $k$-tkowym nazywamy wąską, jeśli każde dwie sąsiednie cyfry różnią się od siebie nie więcej niż o jeden.
Dla danych liczb $k$ oraz $n$ policz ile jest $n$-cyfrowych liczb w systemie $k$-tkowym, które są wąskie.

\section*{Mądre słonie}

Niektórym wydaje się, że im większy jest słoń, tym jest mądrzejszy.
Aby zaprzeczyć temu twierdzeniu chcesz znaleźć w podanej bazie danych słoni jak najdłuższy ciąg w którym waga słoni rośnie, a IQ maleje.

\section*{Odwracanie naleśników}

Dany jest talerz z naleśnikami.
Naleśniki mają różne średnice i leżą na talerzu na wspólnym stosie.
Chcemy posortować naleśniki, przy czym jedyną dopuszczalną operacją jest włożenie łopatki pod wybrany naleśnik a następnie odwrócenie go wraz z wszystkimi naleśnikami leżącymi wyżej.
Należy znaleźć ciąg operacji odwracania naleśników, który doprowadzi do posortowania naleśników według średnic.
Możesz założyć, że naleśniki są z serem.

\section*{Upośledzony Gustaw}

Gustaw potrafi liczyć.
Teraz uczy się jak liczby zapisuje się na papierze.
Ponieważ Gustaw jest bardzo dobrym uczniem, zna już cyfry $1$, $2$, $3$ oraz $4$.
Nie zorientował się on jeszcze, że $4$ to jest inna cyfra niż $1$, więc myśli, że ``$4$'' to inny sposób aby zapisać ``$1$''.
Pomimo tego wymyślił bardzo ``ciekawą'' grę.
Mając daną liczbę $n$ zastanawia się ile zna liczb, których cyfry sumują się do $n$.
Jeśli na przykład $n$ jest równe $2$ to Gustaw zna pięć takich liczb: $11$, $14$, $41$ oraz $2$.
Opracuj algorytm, który dla zadanaj liczby $n$ policzy ile liczb zna Gustaw, które sumują się do $n$.

\section*{Gra w klasy}

W bogatej dzielnicy Sosnowca dzieci znalazły sobie nową zabawą.
Na chodniku malują szachownicę o rozmiarach $n \times n$.
Następnie na każde pole szachownicy kładą różną ilość groszy.
Kolejnie jedno dziecko staje w dolnym lewym rogu szachownicy i zbiera wszystkie monety leżące na tym polu.
Następnie wybiera pomiędzy dwoma kierunkami: prawo lub góra i skacze na wybrane przez siebie pole
(o ile jest w stanie na nie doskoczyć; powiedzmy, że jest w każdym momencie skoczyć na pole nie dalsze niż o $k$).
Pole to musi zawierać więcej groszy niż pole na którym stał uprzednio.
Z tego pola zbiera wszystkie monety i zabawę kontynuuje dopóty ma możliwe ruchy.
Opracuj algorytm, który mając daną szachownicę obliczy ile dany gracz jest w stanie zebrać groszy.

\section*{Pokrycie przedziałowe}

Dane są przedziały liczbowe postaci $(l_i, r_i)$.
Należy wybrać jak najmniejszą liczbę przedziałów tak, aby ich suma zawierała przedział $(0, M)$.

\section*{Problem chińskich pałeczków}

Dlaczego łyżkę kojarzy się z jesienią?
Bo je się nią.
Ale nie w Chinach.
Tam je się pałeczkami.
Profesor L. wymyślił właśnie nowy system jedzenia w którym używa się $3$ pałeczek.
Jednej normalnej pary oraz jednego długiego patyka, którym zjada duże kawałki dziabiąc nim.
Oczywiście dobrze by było, aby dwie małe pałeczki były o zbliżonym rozmiarze, natomiast rozmiar największej pałeczki nie ma znaczenia.
Formalnie - jeśli $A \leq B \leq C$ to rozmiar pałeczek to niedobroć zestawu definiujemy jako $(B-A)^2$.
Profesor L. organizuje właśnie imprezę na którą przyjdzie $K$ gości.
Chce on zaprezentować swoim gościom swój nowy system.
Planuje teraz wybrać ze swojej kolekcji $N$ pałeczek $K$ zestawów, tak aby zminimalizować sumę niedobroci zestawów.

\section*{Konstruowanie BST}

Mamy zadaną wysokość $H$ oraz liczbę elementów $N$.
Chcemy znaleźć taką permutację liczb $1, 2, \cdots N$, że wstawiając do początkowo pustego drzewa BST otrzymamy drzewo o wysokości nie większej niż $H$.
Jeśli jest więcej takich permutacji to chcemy znaleźć leksykograficznie najmniejszą.

\section*{Optymalne BST}

Mamy zadany zbiór par $\{(e_i, p_i)\}$.
Chcemy z nich utworzyć drzewo w ten sposób, aby tworzyło ono drzewo BST po wartościach $e_i$.
Ponadto dla każdego drzewa definiujemy jego wagę jako $\sum p_i \cdot d_i$ gdzie $d_i$ to odległość $i$-tego wierzchołka od korzenia.
Należy znaleźć drzewo BST złożone z zadanych elementów o minimalnej wadze.

\section*{Willy Wonka}

Willy Wonka chce rozdać trójce dzieci cukierki.
Cukierków jest $N$ ($N \leq 30$) i każdy z nich ma swoją wagę w gramach ($w_i \leq W \leq 20$).
Willy chce rozdać wszystkie cukierki i chce aby dzieciaki dostały mniej więcej po równo - tj. chce zminimalizować różnicę w wadze między dzieciakiem, który dostał najcięższy worek cukierków a dzieciakiem, który dostał worek najlżejszy.

\section*{Ustawianie domina}

Chcemy ustawić szereg domina złożony z $N$ kostek domina.
Powiedzmy, że nasz szereg wygląda tak: ``DD\_\_DxDDD\_D''.
Chcemy teraz wstawić kolejną kostkę w pole oznaczone jako ``x''.
Ponieważ mylić się jest rzeczą ludzką - prawdopodobieństwo, że kostka przwróci się w lewo wynosi $p_l$ a w prawo $p_r$.
Zakładamy, że $0 < p_l + p_r \leq 0.5$.
Gdy kostka się przewróci - przewraca blok kostek znajdujący się obok niego.
W przykładzie - jeśli kostka którą stawiamy przewróci się w lewo - przewróci dodatkowo jedną kostkę.
Jeśli w prawo - przewróci dodatkowo aż trzy kostki.
Kostki te będziemy musieli postawić na nowo.
Ile wynosi oczekiwana ilość kostek, które musimy postawić w optymalnej strategii ustawienia?

\section*{Hazard}

Z sześciu symboli rozlosowywane są $3$ (z powtórzeniami).
Stawiamy $a \leq L$ złotych na jeden symbol (gdzie $L$ to dostępny limit na zakład).
Jeśli zostanie on wylosowany to odzyskujemy swoje $a$ złotych plus dodatkowo zyskujemy $a$ złotych za każde wystąpienie naszego symbolu wśród $3$ wylosowanych.
Jacek opracował Strategię Na Pewno Wygrywającą.
Wygląda ona w ten sposób, że Jacek stawia w pierwszym zakładzie $1$ złoty.
Jeśli przegra to w następnym zakładzie stawia dwa razy więcej niż w zakładzie poprzednim.
Łatwo można zauważyć, że gdy któryś zakład wygramy - wychodzimy na plus.
Wtedy zaczynamy naszą strategię od nowa - stawiając ponownie złotówkę na zakład.
Problemem jest limit na zakład.
Gdy Jacek go osiągnie, zaczyna strategię od nowa - mając nadzieję, że jeszcze się odkuje.
Jakie jest prawdopodobieństwo, że między $K$-tym zakładem, a $M$-tym chociaż raz będziemy na plusie?

\section*{Złote monety}

W Bajtocji istnieje dziwny system monetarny.
Każda moneta ma wybity nominał, który jest liczbą naturalną.
Każdą monetę o nominale $A$ można w Banku Bajtockim rozmienić na monety $\lfloor A / 2 \rfloor$, $\lfloor A / 3 \rfloor$ oraz $\lfloor A / 4 \rfloor$.
Ponadto monety można zamieniać na polskie złotówki przy kursie jeden do jednego (nie można dokonywać zamian w przeciwną stronę).
Mamy jedną monetę o nominale $A$.
Ile złotówek możemy otrzymać za nią?

\section*{Deski i gwoździe}

Dany jest zbiór przedziałów $\{(s_i, k_i)\}$.
Pytamy się o najmnijeszy zbiór punktów $\{p_j\}$ taki, że dla każdego $i$ istenieje takie $j$, że $p_j \in (s_i, k_i)$.

\section*{Sumowanie}

Dany jest zbiór liczb naturalnych $\{a_i\}$, który należy zsumować.
Koszt zsumowania liczb $a$ oraz $b$ wynosi $a + b$.
W jakiej kolejności należy sumować elementy, aby zminimalizować koszt sumowania?

\section*{Rzeźbiarz}

Do pracowni rzeźbiarskiej właśnie dotarła marmurowa płyta o rozmiarach $H \times W$.
Płytę tą można ciąć na dwa prostokąty wzdłóż jednego z jego boków.
W pracowni potrzebne są określone rozmiary płyt $\{ (h_i, w_i) \}$.
Jeden rozmiar można wykorzystać wielokrotnie.
Pracownia chce teraz tak pociąć dostarczoną płytę, aby zminimalizować ilość ``odpadków''.

\section*{Dzikie kodowanie}

Każdej literze przyporządkowujemy dodatnią liczbę całkowitą, określającą jej pozycję w alfabecie.
Słowa kodujemy zapisując każdą literę za pomocą przypisanej jej liczby.
Na przykład słowo ``BEAN'' zakodowalibyśmy jako ``25114''.
Problem jest z dekodowaniem słowa.
Dla przykładu - ``25114'' można zdekodować na $6$ sposobów.
Dla danej liczby należy policzyć na ile sposobów mozna ją zdekodować.

\section*{Transport przez rzekę}

Mamy dany statek, który może na raz przetransportować $n$ samochodów z jednego brzegu rzeki na drugi.
Transport trwa $k$ jednostek czasu w jedna stronę, po którym statek musi powrócić na pierwszy brzeg.
Mamy danych $m$ samochodów oraz $s_i$ - czas przyjazdu każdego samochodu nad rzekę.
Pytamy się o to w jaki sposób należy transportować samochody, aby ostatni samochód przekroczył rzekę jak najwcześniej.
Ile conajmniej kursów musi zrobić statek, aby osiągnąć taki czas?

\section*{Gra na grafie}

Dany mamy graf pełny z pętelkami.
Każda krawędź ma jeden z trzech kolorów.
Na grafie znajdują się $3$ pionki.
Możemy przesunąć pionek z wierzchołka $u$ na wierzchołek $v$ jeśli krawędź między tymi wierzchołkami jest tego samego koloru co krawędź łączącą pozycje dwóch pozostałych pionków.
Pytamy się o minimalną liczbę ruchów jakie trzeba wykonać, aby przesunąć wszystkie pionki na jedno pole.

\section*{Sklep z kwiatami}

Mamy dany zbiór kwiatów oraz rząd donniczek.
Musimy umieścić kwiaty w donniczkach zachowując następujące zasady:
\begin{itemize}
 \item donniczek nie można przestawiać
 \item jeden rodzaj kwiatu może wystąpić w wielu donniczkach
 \item kwiaty muszą być ustawione w kolejności alfabetycznej (narcyzy muszą znaleźć się na lewo od róż)
\end{itemize}
Nie wszystkie kwiaty tak samo ładnie wyglądają w różnych donniczkach.
Bardziej formalnie - mamy zadaną macierz $\{m_{i,j}\}$ określającą ``ładność'' wyglądu $i$-tego gatunku kwiatów w $j$-tej donniczce.
Pytamy się o najładniejsze ustawienie kwiatów w donniczkach.

TODO: Dodać zadania z średnie.pdf