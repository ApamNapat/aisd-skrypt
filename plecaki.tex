\section{Problem plecakowy}
\sectionauthor{Marcin Bartkowiak, Krzysztof Piecuch}

\label{sec:plecaki}

Wyobraźmy sobie, że jesteśmy złodziejem.
Pod przykryciem nocy, udało nam się dotrzeć w ustalone miejsce.
Wszystko idzie wyjątkowo gładko.
Wpisujemy kod, który otrzymaliśmy od sprzątaczki i jesteśmy już w środku.
Wtem okazuje się, że zapomnieliśmy listy przedmiotów, które mieliśmy ukraść.
Próbujemy zachować zimną krew i zmaksymalizować nasz zysk.
Możemy ukraść przedmioty różnego rodzaju.
Dokładniej rzecz ujmując - mamy $M$ różnych typów przedmiotów.
Każdy typ przedmiotu ma swoją wielkość $w_i > 0$ i swoją cenę u pasera $v_i > 0$.
Wiemy, że w naszym plecaku zmieszczą się przedmioty o łącznej wielkości nie przekraczającej $W$.
Które przedmioty mamy wybrać (i w jakiej ilości), aby zmaksymalizować nasz zysk i żeby nie przepakować naszego plecaka?
Bardziej formalnie: w jaki sposób wybrać zmienne $x_i$ tak aby zysk $\sum_{i=1}^{M} x_i \cdot v_i$ był jak największy oraz aby był
zachowany warunek $\sum_{i=1}^{M} x_i \cdot w_i \leq W$.
W zależności od tego do jakiego sklepu się włamaliśmy, rozróżniamy następujące rodzaje problemu plecakowego:
\begin{itemize}
  \item Galeria sztuki.
  W tym przypadku mamy dodatkowe ograniczenie: $x_i \in \left \{ 0, 1 \right \}$.
  Każde dzieło sztuki jest unikalne i nie możemy z galerii wynieść dwóch takich samych waz ani obrazów.
  Wersję tą nazywamy czasem dyskretnym $0$/$1$ problemem plecakowym.
  \item Sklep RTV.
  Tym razem ukraść możemy dwa takie same telewizory.
  Ale wciąż ilość przedmiotu danego typu jest ograniczona: $x_i \in \mathbb{N}$, $x_i \leq c_i$.
  Problem ten nazywamy czasem ograniczonym, dyskretnym problemem plecakowym.
  \item Cyberprzestrzeń.
  W tym przypadku możemy ukraść dowolną ilość przedmiotu danego typu (np. klucze aktywacji).
  Wtedy $x_i \in \mathbb{N}$.
  Problem ten można znaleźć w literaturze pod hasłem: nieograniczony, dyskretny problem plecakowy.
  \item Laboratorium chemiczne.
  W tym przypadku, ilość chemikaliów, które kradniemy, jest ograniczona ($0 \leq x_i \leq c_i$).
  Ponieważ chemikalia możemy przelewać, ilość chemikaliów nie musi wyrażać się liczbą naturalną ($x_i \in \mathbb{R}$).
  Jest to ciągły problem plecakowy.
\end{itemize}

Najprostszym pomysłem jaki wpada do głowy, jest policzenie dla każdego przedmiotu stosunku wartości do wielkości ($v_i / w_i$).
Następnie wzięcie w pierwszej kolejności przedmiotów o większej wartości tego ilorazu.
Taki zachłanny algorytm sprawdza się w przypadku ciągłego problemu plecakowego (TODO: dowód).
Złożoność tego algorytmu to $\Theta(n \log n)$, gdyż potrzebujemy posortować tablicę ilorazów.

Nie jest jednak poprawnym algorytmem dla wersji dyskretnych.
Kontrprzykładem będzie sytuacja, w której plecak ma wielkość $W = 10$ i do dyspozycji mamy $M = 3$ przedmioty: $v_1=9$, $v_2=v_3=5$, $w_1=6$, $w_2=w_3=5$.
W każdym z wariantów problemu postępując w sposób zachłanny, wybierzemy w pierwszym kroku przedmiot $1$.
Otrzymamy w ten sposób plecak o wartości $9$, do którego nie jesteśmy w stanie już wstawić kolejnego przedmiotu.
Wybierając przedmioty $2$ i $3$ otrzymujemy plecak o wartości $10$.
Zatem wybranie przedmiotu o największym stosunku wartości do wielkości było w takiej sytuacji błędem.
Poprawnym rozwiązaniem wersji dyskretnych okazuje się podejście dynamiczne.

Zacznijmy od najprostszej wersji - nieograniczonej.
Użyjemy tutaj tablicy \texttt{T[0..W]}.
W komórce \texttt{T[w]} będziemy pamiętać optymalne rozwiązanie dla plecaka wielkości $w$.
Rozwiązanie naszego problemu będzie znajdowało się w komórce \texttt{T[W]}.
Wartości poszczególnych komórek liczymy według wzoru:
\begin{equation*}
  T[w] = \begin{cases}
    0, & \text{jeśli $w = 0$}\\
    max_{w_i \leq w}\{v_i + T[w - w_i]\}, & \text{wpp}
  \end{cases}
\end{equation*}
Gdy plecak jest rozmiaru $0$ nie możemy zapakować do niego żadnego przedmiotu.
Jeśli mamy dostępne miejsce - patrzymy na wszystkie przedmioty, które mieszczą się do plecaka i obliczamy potencjalny zysk przy wzięciu każdego z nich.
Z otrzymanych wartości bierzemy maksimum.
Do policzenia mamy $\Theta(W)$ komórek.
Do policzenia każdej z nich potrzebujemy $O(M)$ operacji w najgorszym przypadku.
Złożoność całego algorytmu można zatem ograniczyć przez $O(M \cdot W)$.

Przejdźmy teraz do nieco trudniejszego przypadku - wersji $0$/$1$.
W tym wariancie problemu, będziemy potrzebowali tablicy dwuwymiarowej \texttt{T[0..W][0..M]}.
W komórce \texttt{T[w][m]} będziemy trzymać optymalne rozwiązanie w sytuacji w którym ograniczamy się do plecaka rozmiaru $w$ oraz rozważamy tylko $m$ pierwszych typów przedmiotów.
\begin{equation*}
  T[w][m] = \begin{cases}
    0, & \text{jeśli $w = 0$ lub $m = 0$}\\
    T[w][m-1], & \text{jeśli $w_m > w$}\\
    max\{v_m + T[w - w_m][m-1], T[w][m-1]\}, & \text{wpp}
  \end{cases}
\end{equation*}
Pierwsza linijka wzoru to przypadek bazowy rekurencji - wtedy gdy plecak jest pusty lub nie mamy do wyboru żadnego przedmiotu.
Teraz gdy wiemy, że nie jesteśmy w przypadku bazowym, bierzemy do ręki przedmiot typu $m$ i kontemplujemy nad tym czy wziąć ten przedmiot czy nie.
Linijka druga wzoru rozwiewa nasz problem w momencie w którym przedmiot ten i tak nie zmieściłby się nam do plecaka (nie bierzemy go w takiej sytuacji; duh...).
Jeśli jednak by się zmieścił to liczymy potencjalną wartość plecaka wraz z tym przedmiotem oraz bez tego przedmiotu.
Jako wynik bierzemy maksimum - o czym mówi linijka trzecia.
Tym razem do policzeniania mamy $\Theta(W \cdot M)$ komórek.
Na każdą komórkę potrzebujemy jednak tylko $\Theta(1)$ operacji.
Zatem złożoność całego algorytmu wynosi $\Theta(W \cdot M)$.

Ostatnim przypadkiem jest wersja ograniczona.
Tak samo jak w przypadku wersji $0$/$1$ będziemy używać tablicy \texttt{T[0..W][0..M]}.
Dokładnie tak samo jak w poprzednim przypadku w komórce \texttt{T[w][m]} będziemy trzymać optymalne rozwiązanie
w sytuacji w którym ograniczamy się do plecaka rozmiaru $w$ oraz rozważamy tylko $m$ pierwszych typów przedmiotów.
Również wzór rekurencyjny będzie wyglądał bardzo podobnie.
\begin{equation*}
  T[w][m] = \begin{cases}
    0, & \text{jeśli $w = 0$ lub $m = 0$}\\
    max_{0 \leq c \leq c_m, c \cdot w_m \leq w}\{c \cdot v_m + T[w - c \cdot w_m][m-1]\}, & \text{wpp}
  \end{cases}
\end{equation*}
\comment{Tu i we wzorze wyżej wyjechałem na margines :C}
Różnica polega na tym, że gdy rozważamy przedmiot typu $m$ to rozważamy każdą jego dostępną ilość, która mieści się w plecaku.
Dla każdej ilości liczymy potencjalną wartość plecaka w którym wzięliśmy daną ilość przedmiotów tego typu.
Z wszystkich wartości bierzemy maksimum jako wynik.
Do policzenia zostaje nam $\Theta(W \cdot M)$ komórek.
Każda komórka \texttt{T[..][m]} wymaga $O(c_m)$ operacji.
Zakładając, że dla każdego typu przedmiotów zachodzi $c_m > 0$, cały algorytm działać będzie w czasie $O(W \cdot \sum_{m \leq M} c_m)$.

