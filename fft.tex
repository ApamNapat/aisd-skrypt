\section{Szybka Transformata Fouriera}
\sectionauthor{Wiktor Garbarek}

%\newthoerem{remark_small}{Mała uwaga}
\newtheorem{observation_small}{Malutka obserwacja}
\newtheorem{observation_bigger}{Nieco większa obserwacja}


\label{sec:fft}

Zacznijmy od problemu mnożenia wielomianów. Mamy dwa wielomiany stopnia co najwyżej, dla późniejszej wygody w zapisie, $n-1$, powiedzmy $A(x) = \sum_{i=0}^{n-1} a_i\cdot x^i$ i analogicznie $B(x)$, gdzie $a_i$ oraz $b_i$ są rzeczywiste.
Szukamy zatem wielomianu $$C(x) = \sum_{i=0}^{2n-2} c_ix^i$$, gdzie $c_i = \sum_{j=0} a_jb_{i-j} = \sum_{p + q = i} a_pb_q $ (oba te zapisy $c_i$ są równoważne, jeśli założymy, że $ 0 \leq p,q$).
Na pierwszy rzut oka możemy zauważyć, że istnieje algorytm obliczający $C(x)$ w czasie $\Theta(n^2)$. Czy da się lepiej? Ano owszem, bo inaczej nie byłoby tego rozdziału.\\

%\begin{remark_small}
    Mała uwaga, bardzo podobne działanie na wektorach $a,b\in \mathbb{R}^n$ matematycy nazywają splotem.
    $$c = a * b \Leftrightarrow c_i = \sum_{j=1}^{n}a_jb_{i-j}$$.
%\end{remark_small}

Zacznijmy od paru obserwacji
\begin{observation_small} (Interpolacja)
    Każdy wielomian stopnia $n-1$ o współczynnikach rzeczywistych da się przedstawić jednoznacznie jako $n$-elementowa lista par $(x_i, W(x_i))$, gdzie ciąg $x_i$ jest dowolnym ciągiem parami różnych liczb rzeczywistych.
\end{observation_small}
\begin{proof}
    Prosty dowód pozostawiamy czytelnikowi.
\end{proof}

\begin{observation_bigger}
    Jeśli mamy wielomiany $A(x)$ oraz $B(x)$ w powyższej postaci obliczonej dla jednego (wspólnego) ciągu $x_i$ (długości $2n$, to ważne!, ale nas to nie boli), to "łatwo" (tj. w czasie liniowym) możemy znaleźć ich iloczyn.
\end{observation_bigger}
\begin{proof}
    Wystarczy powiedzieć, że $C(x)$ to ciąg $(x_i, A(x_i)B(x_i))$.
\end{proof}
Super! W takim razie, jedyne co pozostaje nam, to znaleźć algorytm, który zamieni wielomian w postaci listy współczynników na próbki naszego wielomianu w $2n$ punktach, obliczymy prosto iloczyn tych wielomianów i jeszcze musimy teraz wrócić do domu obliczając współczynniki wielomianu interpolacyjnego.
No i tutaj jest cały problem, bo trywialny algorytm obliczy każdą tą zamianę postaci w czasie $\Theta(n^2)$ (Mamy $2n$ argumentów, dla każdego z nich obliczamy wartość wielomianu w czasie $\Theta(n)$)
Spróbujmy zastosować tutaj metodę divide and conquer. Najpierw jednak zauważmy ciekawą zależność.

\begin{observation_small}
Dla każdego wielomianu zachodzi następująca równość $$A(x) = A^e(x^2) + xA^o(x^2)$$, gdzie $A^e(x) = \sum_{i=0}^{n/2 - 1}a_{2i}x^i$ oraz $A^o(x) = \sum_{i=0}^{n/2 - 1} a_{2i+1}x^i$
('e' oznacza \textit{even}, elementy na parzystych indeksach, a 'o' oznacza \textit{odd})
\end{observation_small}
Niektórzy mogą pomyśleć, że to już koniec. Prosto bierzemy sobie wielomian $A(x)$, jakiś dowolny zbiór $X$, obliczamy rekurencyjnie wartość dla wielomianów $A^e$ i $A^o$ przy wykorzystaniu zbioru $X' = \{x^2 : x \in X\}$.
Nic bardziej mylnego, zauważmy bowiem, że gdy będziemy łączyć wyniki, to tak naprawdę za każdym razem wykonamy $\Theta(|X|)$ kroków, a przy dowolnie wybranym zbiorze, ten zbiór $X'$ wcale nie musi (i co najbardziej prawdopodobne, nie będzie) zmniejszać swojego rozmiaru
(Nota bene taki algorytm dalej ma złożoność $\Theta(n^2)$).
Ale jednak jest nadzieja! Możemy go wybrać dowolnie, więc wystarczy znaleźć $X$ taki, że $|X'| = \frac{|X|}{2}$. Brzmi niewykonalnie? Jednak wcale nie jest. Popatrzmy sobie na pierwiastki jedynki na płaszczyźnie zespolonej. Spójrzmy na przykład dla pierwiastków ósmego stopnia.
Gdy weźmiemy dowolną liczbę z tego zbioru i podniesiemy ją do kwadratu, to wtedy jej moduł się nie zmienia, a jedyne co robimy to podwajamy kąt między jej promieniem wodzącym, a dodatnią półosią OX. W takim razie zamiast rozpatrywać kąty 0, 45, 90, 135, 180, 225, 270, 315 będziemy rozpatrywać kąty 0, 90, 180, 270, 360, 450, 540, 630. Jednakże obrót o 450 stopni, to tak naprawdę obrót o 90, 540 to tak naprawdę 180 stopni itd.
Czyli widzimy, że podnosząc do kwadratu wszystkie elementy w tym zbiorze zmniejszyliśmy jego liczność do połowy. (I znaleźliśmy też pierwiastki czwartego stopnia z jedynki).

Reasumując obliczamy w takim razie $$\{W(e^{\frac{2k\pi i}{2n}}) = \sum_{j=0}^{n-1}a_je^{\frac{2k\pi ij}{2n}} | k = 0,...,2n-1\}$$
Żeby jednak nasze indeksowanie ładnie wyglądało, to możemy pomyśleć, że tak naprawdę $a_n, a_{n+1}, ..., a_{2n-1}$ są zerami, a wtedy możemy zapisać to przekształcenie w ten sposób.
Mając wektor $\vec{a} = [a_0, ..., a_{N-1}]$, szukamy wektora $\vec{y} = [y_0, ..., y_{N-1}]$ takiego, że
$$y_k = \sum_{j=0}^{N-1}a_je^{\frac{2k\pi ij}{N}}$$. Oto jest właśnie (już prawie!) Dyskretna Transformata Fouriera.
\begin{definition}
    Dyskretną Transformatą Fouriera nazywamy następującą funkcję $dft: \mathbb{C}^N \rightarrow \mathbb{C}^N$
    gdzie, jeśli $\vec{y} = dft(\vec{a})$ to $$y_k = \sum_{j=0}^{N-1}a_je^{\frac{-2k\pi ij}{N}}$$
\end{definition}
Niby jedyną różnicą (i aż jedyną!) jest minus - jaki to ma wpływ na wynik w praktyce, pozostawiamy czytelnikowi do sprawdzenia.
Algorytm ten zawdzięczamy Cooley'owi oraz Tukey'owi (1965), jednakże idea ta była znana już 160 lat wcześniej Gaussowi.

Poruszmy jeszcze jedną ważną rzecz - przede wszystkim w powyższych rozważaniach niemo wymagaliśmy, by długość naszego wektora była potęgą dwójki. Jeśli chodzi natomiast o problem mnożenia wielomianów, w żaden sposób nas to nie boli, ponieważ zawsze możemy dopełnić wektor jakąś ilością zer do najbliższej potęgi dwójki, a skoro wydłużył on się co najwyżej dwukrotnie, to ta idea nie zmienia złożoności całego algorytmu.
Nie obchodzą nas też w żaden sposób wartości pośrednie (czytaj: zamiana wielomianu z wektora współczynników, na wektor punktów do interpolacji)
Jednak gdyby zależało nam na policzeniu powyższego działania (tj. nie mnożenia wielomianów, a już obliczenia DFT dla dowolnego wektora) pojawia się duży problem. Przede wszystkim całą zabawę psuje liczba $\exp{\frac{\mathunderscore}{N}}$, której nie pozbędziemy się w żaden trywialny sposób.

Wracając jednak do algorytmu, zapiszemy go jeszcze w pseudokodzie.
\begin{algorithm}[H]
  \DontPrintSemicolon
  \SetAlgorithmName{Algorytm}{Cooley-Tukey}

  \KwData{ $x = [x_0, x_1, ... x_{N-1}] \in \mathbb{C}^n$ - wektor liczb zespolonych długości N}

  \KwResult{$y \in \mathbb{C}^n$, taki, że $y_k = \sum_{n=0}^{N-1}x_n \cdot \exp{\frac{-2\pi i n k}{N}}$ dla $k = 0,1,...,N-1$}

  \For{$i \leftarrow 1$ to $n$}
  {
     \For{$j \leftarrow 1$ to $p$}
     {
	$C[i][j] \leftarrow 0$\;
	\For{$r \leftarrow 1$ to $m$}
	{
	  $C[i][j] \leftarrow C[i][j] + A[i][r] \cdot B[r][j]$\;
	}
     }
  }

  \caption{Algorytm Cooleya-Tukeya}
  \label{alg-fft-cooley-tukey}
\end{algorithm}
