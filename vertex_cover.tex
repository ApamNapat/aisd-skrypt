\section{Pokrycie wierzchołkowe}
\sectionauthor{Krzysztof Starzyk}

MPK Wrocław chciałoby dokonać inspekcji wszystkich torów tramwajowych w mieście (wystarczy odwiedzić przystanek by dokonać inspekcji incydentnych) torów. Podwykonawca (TOREX) żąda opłaty za każdy odwiedzony przystanek z osobna więc MPK chciałoby zminimalizować liczbę przystanków które będzie musiała odwiedzić ekipa TOREXu (zachowując przy tym podstawowe zadanie jakim jest inspekcja WSZYSTKICH torów). 
MPK Wrocław cannot into informatyka więc nie rozwiąże tego problemu (rozwiąże go podwyżką cen biletów). My co prawda nie potrafimy w czasie wielomianowym dostarczyć żądanego rozwiązania, ale wiemy jak się do tego zabrać.

Bardziej formalnie:  

\begin{definition}
  \textbf{Pokryciem wierzchołkowym} (\texttt{dal.} PW) grafu $G = (V,E)$ nazywamy zbiór $V'$ tż.:
  $V' \subseteq V \wedge (\forall e\in E, \exists v\in V':  v \in e)$.
\end{definition}
\textbf{Problem pokrycia wierzchołkowego} (\texttt{dal.} PPW) będziemy rozpatrywać na dwa sposoby: optymalizacyjnym (jakie jest najmniejsze PW) i decyzyjnym (czy istnieje PW rozmiaru $k$); posiadając wszelakie zastosowania praktyczne jest oczywiście problemem NP-zupełnym.
  
Jeżeli przystanki tramwajowe potraktujemy jako wierzchołki a tory między  nimi jako krawędzie, to PW nazwiemy taki podzbiór przystanków że wszystkie tory mają przynajmniej jeden koniec kończący się na przystanku z naszego podzbioru. 

Przybliżone rozwiązanie:
W dość prosty sposób możemy uzyskać przybliżone rozwiązanie (co najmniej(?) dwukrotnie gorsze od optymalnego w sensie mocy otrzymanego zbioru). Idea jest następująca: dla każdej krawędzi $e$ w $E'$ ($=E$) weźmy dwa wierzchołki które $e$ łączy, dodajmy je do zbioru rozwiązań i usuńmy z $E'$ wszystkie krawędzie incydentne do nich. 

\begin{algorithm}[h]
  \DontPrintSemicolon
  \SetAlgorithmName{}{}

  \KwData{ $G = (V, E)$}
  \KwResult{ $S$ - pewne \texttt{PW} dla grafu $G$  }
  $E' = E$
  \While{$E' != \{\}$}
  {
  $e'$ - dowolna krawędź łącząca wierzchołki $(u,v)$
  $S = S \bigcup u$ 
  $S = S \bigcup v$
  $E' = E' \backslash$ (wszystkie krawędzie incydentne do $u$ i $v$) 
  }
  \caption{Przybliżone rozwiązanie \texttt{PPW}}
  \label{problem-pokrycia-wierzcholkowego}  
\end{algorithm}


