\section{Algorytm Dijkstry}
\sectionauthor{Mikołaj Słupiński}

\label{sec:dijkstra}

\paragraph{}Powstało wiele algorytmów pozwalających wyznaczyć najkrótszą ścieżkę w grafie z
krawedziami ważonymi. Wśród nich na szczególną uwagę zasługuje algorytm Dijkstry.

\subsection{Działanie}

\paragraph{}Niech $G = (V, E)$ będzie grafem ważonym. Dodatkowo musimy założyć, że waga $w(u,v) \ge 0$
dla wszystkich krawędzi $(u, v) \in E$.

Niech $S$ będzie takim zbiorem wierzchołków, których najkrótsza odległość od źródła
s została już określona. Algorytm Dijkstry wybiera kolejne wierzchołki $u \in V - S$
z minimalnym oszacowaniem najkrótszej ścieżki, dodaje u do S, i rozluźnia wszystkie ścieżki pozostawiając u.

\paragraph{Pseudokod:}
TODO

Algorytm zachowuje niezmiennik, że $Q = V - S$ na początku każdej iteracji pętli while.

Algorytm Dijkstry stosuje zachłanne podejście zawsze wybierając najbliższy wierzchołek
w $V - S$, który dodaje do zbioru S.

\subsection{Dowód poprawności algorytmu}

\paragraph{}Aby dowieść poprawności algorytmu Dijkstry skorzysamy z następującego niezmiennika pętli:
Na począrku każdej iteracji pętli while $d[v] = \delta(s,v)$ dla każdego wierzchołka $v \in S$.

Wystarczy udowodnić, że dla każdego wierzchołka $u \in V$ mamy $d[u] = \delta(s, u)$ w momencie
kiedy $u$ jest dodane do zbioru $S$. Kiedy już udowodnimy, że $d[u] = \delta(s, u)$,
polegamy na ograniczeniu górnym własności, aby pokazać, że równość jest potem zachowana.

\paragraph{Inicializacja:} Na samym początku, $S$ jest zbiorem pustym, więc niezmiennik jest oczywiście
zachowany.

\paragraph{Utrzymanie:} Chcemy pokazać, że z każdą iteracją zachowana jest równość $d[u] = \delta(s, u)$
dla wszystkich wierzchołków dodanych do zbioru $S$. Załóżmy nie wprost, że u jest
pierwszym takim wierzchołkiem, że $d[u] \ne \delta(s,u)$ w momencie go do zbioru
$S$. Weźmy dowolny $u \ne s$, ponieważ  $s$ jest pierwszym wierzcholkiem dodanym
do $S$ i E$d[s] = \delta(s,s) = 0$. Skoro $u \ne s$, to $S$ nie jest zbiorem pustym
zaraz przed dodaniem $u$ do S. Musi istnieć ścieżka od $s$ do $u$ gdyż inaczej
$d[u] = \delta(s,u) = \inf$ i doszlibyśmy do sprzeczności z naszym założeniem, że
$d[u] \ne \delta(s, u)$. Skoro istnieje conajmniej jedna ścieżka, istnieje też ścieżka
najkrótsza. Przed dodaniem $u$ do $S$, ścieżka p łączy wierzchołek w $S$, powiedzmy s,
z wierzchołkeim w $V - S$, powiedzmy $u$. Rozważmy pierwszy wierzchołek y należący do p
t. że $y \in V - S$ i niech $x \in S$ będzie poprzednikiem y. Możemy podzielić ścieżkę
$p$ na dwie podścieżki, $p_1$ łączącą $s$ z $x$ oraz $p_2$ łączącą $y$ z $u$
(ścieżki te mogą być pozbawione krawędzi).

Chcemy udowodnić, że $d[y] = \delta(s,y)$ gdy $u$ jest dodane do $S$. Aby to zrobić
zauważmy, że $x \in S$. Skoro $u$ jest pierwszym wierzchołkiem, dla którego $d[u] \ne \delta(s,u)$,
to $d[x] \delta(s,x)$ w momencie kiedy $x$ został dodany do S. Krawędź $(x, y)$
została wtedy zrelaksowana, z czego wynika powyższa równość.

Możemy teraz uzyskać sprzeczność pozwalającą nam udowodnić, że $d[u] = \delta(s,u)$.
Skoro $y$ występuje przed $u$ na najkrótszej ścieżce od $s$ do $u$, a wszystkie
wagi krawędzi są nieujemne (w szczególności krawędzi należących do $p_2$). Otrzymujemy
$\delta(s, y) \le \delta(s, u)$, więc
TODO

Ale skoro oba wierzchołki u i y były w $V - S$ gdy ustaliliśmy $u$ mamy $d[u] \le d[y]$.
Zatem, obie nierówności są tak na prawdę równościami, dzięki którym $d[y] = \delta(s, y) = \delta(s, u) = d[u]$.
W rezultacie $d[u] = \delta(s,u)$, co przeczy naszemu wyborowi u. Wnioskujemy, że
$d[u] = \delta(s, u)$ gdy dodamy $u$ do $S$, a własność ta jest zachowana od tego momentu.

\paragraph{Zakończenie:} Na końcu, $Q$ jest puste, co w połączeniu z naszym niezmiennikiem, że $Q = V - S$ implikuje,
że $S = V$, zatem $d[u] = \delta(s,u)$ dla każego wierzchołka $u \in V$.

\subsection{Analiza}

Skoro wiemy już jak działa algorytm Dijkstry oraz wiemy, że działa poprawie należy
zastanowić się z jaką prędkością on działa.
TODO
\subsection{Problemy}
TODO
